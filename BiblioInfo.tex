% Filling in this bibliographic information facilitates the
% processing of this document.
% Insert linebreaks if necessary
% The abstract page is not generated by default. The reason for this is that it
% will be generated by the unit for Publishing and Graphic sevices from the metadata 
% template ("Spikningsmallen"). If you want to include the abstract page generated from this template,
% please edit the \frontmatterMonograph or \frontmatterCS command at the en of this file
%. Because of this, some of the bibliografic data on this page will not be used.

% Abstract and titelpage
\newcommand{\authorSurname}{Mahmud} % Your surname
\newcommand{\authorFirstName}{Nesredin} % Your given name
\newcommand{\authorFirstInitial}{N} % Initial of given name
\newcommand{\authorEmail}{nesredin.mahmud@mdh.se} % Your e-mail address
\newcommand{\dissertationTitle}{Design of Assured and Efficient Safety-critical Systems using Formal Methods and Optimization Techniques} % The title of the dissertation
\newcommand{\dissertationSubtitle}{Subtitle of the dissertation}% The subtitle of the dissertation (if there is any).
\newcommand{\yearOfPublication}{2019} % Year of publication
\newcommand{\placeOfDisputation}{M\"{a}lardalen University} % Place of disputation
\newcommand{\dateOfDisputation}{Thursday, June 13, 2019} % Date of disputation (Day, Month, Year)
\newcommand{\timeOfDisputation}{13:15} % Time of disputation (Swedish time format)
\newcommand{\disputationLanguage}{English} % Language the examination will be conducted in.
\newcommand{\numberOfPages}{51} % The page number of the last page
\newcommand{\placeOfPublication}{Uppsala} % The place of publication
\newcommand{\ISBN}{978-91-554-5436-4} % The ISBN number of the dissertation.
\newcommand{\series}{Uppsala Dissertations from the Faculty of Science and Technology} % The title of the series
\newcommand{\serialNumber}{1214} % The number in the series.
\newcommand{\keywords}{ embedded system, real-time system, safety-critical system, requirements specification, formal method, Simulink, integer-linear programming, metaheuristics} % Key words separated by comma
\newcommand{\department}{Department of Electronic Publishing} % Name of your department
\newcommand{\departmentaddress}{Villavagen 14, SE-752 36 Uppsala,
Sweden} % Address of your department
\newcommand{\ISSN}{1651-6214} % (ISSN for Digital comprehensive summaries of Uppsala dissertations from the faculty of science and technology)
\newcommand{\urn}{urn:nbn:se:uu:diva-3344} % URN number

% List of papers
\newcommand{\listofpapers}
{\cleardoublepage
\pdfbookmark[0]{List of Papers}{LOP}
\chapter*{List of Papers Included in the Thesis}
\noindent This thesis is based on the following papers:
\vspace{1\baselineskip}

    % It is possible to refer to the published papers using labels in the text.
    % Suggested order
    % Author 1 surname, Author 2 first name initial., Author 1 surname, Author 2 first name
    % initial. etc. (Year of publication) Paper main title.
    % Paper subtitle. Name of journal in italics, volume(number):page rage
    % Example
    
  \begin{description}	
   \item[Paper A] ReSA: An ontology-based requirement specification language tailored to automotive systems. Nesredin~Mahmud, Cristina~Seceleanu and Oscar~Ljungkrantz.\textit{In the 10th IEEE International Symposium on Industrial Embedded Systems (SIES)(pp. 1-10). IEEE, 2015}.\label{lbl_resa}
  
   \item[Paper B] ReSA tool: Structured requirements specification and SAT-based consistency-checking. Nesredin~Mahmud, Cristina~Seceleanu and Oscar~Ljungkrantz. \textit{In the 2016 Federated Conference on Computer Science and Information Systems (FedCSIS)(pp. 1737-1746). IEEE, 2016.}\label{lbl_resatool}
   
   \item[Paper C] Specification and semantic analysis of embedded systems requirements: From description logic to temporal logic. Nesredin~Mahmud, Cristina~Seceleanu and Oscar~Ljungkrantz. \textit{In the International Conference on Software Engineering and Formal Methods (SEFM)(pp. 332-348). Springer, Cham, 2017.}\label{lbl_resadl}
      
   \item[Paper D] SIMPPAAL - A Framework For Statistical Model Checking of Industrial Simulink Models. Predrag~Filipovikj, Nesredin~Mahmud, Raluca~Marinescu, Cristina~Seceleanu, Oscar~Ljungkrantz and Henrik~L\"{o}nn. \textit{Submitted to ACM Transactions on Software Engineering and Methodology (TOSEM). Initially submitted in November 2018, revisions required.} 
   \\[3pt]{\footnotesize This article is an extended version of the following conference paper:
   Simulink to UPPAAL Statistical Model Checker: Analyzing Automotive Industrial Systems.
Predrag Filipovikj, Nesredin Mahmud, Raluca Marinescu, Cristina Seceleanu, Oscar Ljungkrantz, Henrik L{\"o}nn. In Proceedings of the 21st International
Symposium on Formal Methods (FM2016), pages 748-756. Limassol, Cyprus. Springer, LNCS, November 2016. }\label{lbl_simulink_ilp}

   \item[Paper E] Power-aware Allocation of Fault-tolerant Multirate AUTOSAR Applications.
   Nesredin~Mahmud, Guillermo~Rodriguez-Navas, Hamid~Faragardi, Saad~Mubeen and Cristina~Seceleanu. \textit{In the 25th Asia-Pacific Software Engineering Conference (APSEC'18). IEEE.} \label{lbl_resa} \label{lbl_softwareallocation_ilp}
    \vspace{1cm}
	\item[Paper F] Optimized Allocation of Fault-tolerant Embedded Software with End-to-end Timing Constraints.	Nesredin~Mahmud, Guillermo~Rodriguez-Navas, Hamid~Faragardi, Saad~Mubeen and Cristina~Seceleanu. \textit{M\"{a}lardalen Real-time Research Center Technical Report (MRTC), ISRN MDH-MRTC-325/2019-1-SE, 2019.}\\
	{\footnotesize Note: The research is submitted as a journal paper to Elsevier Journal of Systems Architecture (JSA). Under Review}. \label{lbl_softwareallocation_pso}
  \end{description}
\section*{\Large List of Papers Not Included in the Thesis}
\vspace{1\baselineskip}
\begin{itemize}	
	\item Evaluating industrial applicability of virtualization on a distributed multicore platform. Nesredin Mahmud, Kristian Sandstr{\"o}m, and Aneta Vulgarakis. \textit{In Proceedings of the 2014 IEEE Emerging Technology and Factory Automation (ETFA), pp. 1-8. IEEE, 2014.}
	\item The multi-resource server for predictable execution on multi-core platforms. Rafia Inam, Nesredin Mahmud, Moris Behnam, Thomas Nolte, and Mikael Sj{\"o}din. \textit{In 2014 IEEE 19th Real-Time and Embedded Technology and Applications Symposium (RTAS), pp. 1-12. IEEE, 2014.}
	\item Simulink to UPPAAL statistical model checker: Analyzing automotive industrial systems. Predrag Filipovikj, Nesredin Mahmud, Raluca Marinescu, Cristina Seceleanu, Oscar Ljungkrantz, and Henrik L{\"o}nn. \textit{In International Symposium on Formal Methods, pp. 748-756. Springer, Cham, 2016.}
\end{itemize}

\vspace{1\baselineskip} \noindent {\small Reprints were made with permission from
the publishers.}}


% Dedication. Dedication is optional. You can enter whatever feels right and use the fonts and images you like as long as they are embedded in the document. The suggested format does not have to be used.
\newcommand{\dedication}%
{\cleardoublepage
\thispagestyle{empty}
\vspace*{\stretch{3}}
\begin{flushright}
		
{\fontfamily{pzc}\Large\selectfont\emph{To My Parents}}

\end{flushright}
\vspace*{\stretch{1}}} % 
% Ack
\newcommand{\ack}%
{\cleardoublepage
	%\thispagestyle{empty}
	{\noindent\LARGE Acknowledgements}\vspace{74pt}

\noindent I would like to express my deepest gratitude to my main supervisor Associate Professor Cristina Seceleanu. Your passion for research is inspiring, and your guidance, encouragements, determination, and patience have been the driving force of my PhD journey. It has been a pleasure to be around with you and get the best mentor one could ever get. I would like to thank my co-supervisor Oscar Ljungkrantz for the lively reception at Volvo Group Trucks Technology (VGTT), Gothenburg, and the wonderful researchers he introduced to me. It is my pleasure also to thank my co-supervisor Guillermo Rodriguez-Navas, for the great team work, valuable feedback, and encouragements. I am grateful to have you all as my supervisors. Your effort has made this work worthwhile.

Moreover, I would like to thank my examining committee Professor Peter Csaba {\"O}lveczky, University of Oslo, Professor Wolfgang Ahrendt, Chalmers University of Technology, Associate Professor Raffaela Mirandola, Politecnico di Milano, for accepting our call and reviewing my thesis; and Associate Professor Antonio Cicchetti for reviewing the thesis proposal. Special gratitude goes to the faculty examiner Professor Joost-Pieter Katoen, RWTH Aachen University, for accepting the opponentship. It is an honour to have you all at my defense. Also, I would like to extend my appreciation to Damir Isovic, Hans Hansson, Thomas Nolte, Moris Behnam, Kristian Sandstr{\"o}m and the IDT administration staff, who facilitated my PhD defense and influenced my academic career. 

I wish to acknowledge with much appreciation the role of Henrik L{\"o}nn, Oscar Ljungkrantz, as well as Saad Mubeen and Hamid Reza Faragardi, who supported my research technically at different levels, including co-authoring joint papers. Your willingness to share and support is highly admired. For the list is too long, I briefly express my respect and love to my colleagues at MDH, and outside such as Tewodros, Abu Hussaine, Bj{\"o}rn. Your friendly support, jokes, and cultural exchanges will always be memorable.

I am thankful to Allah for granting me the well-being and strength to do my studies. I thank my parents for giving me unconditional love and supporting throughout my life, and my brothers, sisters and lifelong friends for the good words and encouragements. Thank you Big Brother~Abdulwhab for being the example. In times of emotional stress, the support of my dear wife, Elham Salih, has been unconditional and unwavering. You are strong, confident and intelligent person and love of my life, and I thank the OOG for that! Our little boy, Mohammed, you should know that we love you so much. So learn the world with open mind and make us proud!

I would like to thank VINNOVA for sponsoring this research via the VeriSpec project, in which the financial support has made this work possible.

\begin{flushright}
Nesredin Mahmud\\
V{\"a}ster{\aa}s, June 13, 2019
\end{flushright}


\vspace*{\stretch{1}}} % 

\newcommand{\nasabstract}%
{
    %\thispagestyle{empty}
	{\noindent\LARGE Abstract}\vspace{84pt}
	
	\noindent 
	Safety-critical systems need to be analyzed rigorously to remove software/specifications errors, that is, their requirements specifications should be unambiguous, comprehensible and consistent, and the software design should conform to the specifications, hence avoiding undesirable system failures. Currently, there is a lack of effective and scalable methods to specify and analyze requirements, and formally analyze the behavioral models of embedded systems. Most embedded systems requirements are expressed in natural language, which is flexible and intuitive but frequently ambiguous and incomprehensible. Besides natural language, template-based requirements specification methods are used frequently to specify requirements (esp. in safety-critical applications). Although, the latter reduce ambiguity and improve the comprehensibility of the specifications, they are usually rigid due to the constrained syntax of the templates, and template selection is challenging. Industrial systems are frequently developed by using modeling and simulation environments such as Simulink, which is also used to generate code automatically for various hardware platforms. Therefore, it is essential to be able to formally analyze Simulink models, to get insight into the behavior of the embedded system, and also prevent potential errors from propagating into the implementation. Analyzing the timing behavior of safety-critical software that is refined by multi-rate periodic tasks with data age constraints across the end-to-end software functionality is not trivial. This is due to the undersampling and oversampling effects caused by the data propagation from higher to lower rates and vice versa, respectively. Furthermore, when such systems are deployed on a distributed architecture, e.g., electrical/electronic vehicular system, besides assuring the timeliness, the reliability of the distributed software should be maximized to counter the higher risk of failures in the distributed computing setting, hence improving the overall predictability of the safety-critical system. However, designing for reliability usually requires additional critical system resources such as power and energy. Hence, to accommodate the growing complexity of software functionality, the design of the safety-critical systems should consider the efficient use of critical system resources such as the power source, while meeting the timing and reliability requirements.
	
	To address the above needs, in this thesis, we propose formal-methods-based approaches and optimization techniques to assure improved quality of requirements specifications and software designs, and to efficiently map software functionality to hardware. The contributions of the thesis are: (i) \resa{} - a domain-specific requirements specification language tailored to embedded systems, based on constrained natural language; (ii) a formal approach to check consistency of \resa{} specifications via Boolean satisfiability problem (SAT) and ontology; (iii) a framework based on statistical model checking to analyze Simulink models via automated transformation into networks of stochastic timed automata; and (iv) a resource-efficient allocation of fault-tolerant software with end-to-end timing and reliability constraints via integer linear programming and hybrid particle-swarm optimization. Our proposed solutions are validated and evaluated on automotive use cases such as the Adjustable Speed Limiter (ASL) and the Brake-by-Wire (BBW) systems from Volvo Group Trucks Technology (VGTT), and on an Engine Management (EM) system benchmark from Bosch.
%\vspace{1\baselineskip}
\vspace*{\stretch{1}}} % 


\newcommand{\samman}%
{	\cleardoublepage
	{\noindent\LARGE Sammanfattning}\vspace{84pt}

\noindent
S{\"a}kerhetskritiska system b{\"o}r analyseras noggrant f{\"o}r att ta bort fel i programvaror och specifikationer, dvs systemens krav m{\"aa}ste vara entydiga, begripliga och konsekventa, och programvarudesignen ska {\"o}verensst{\"a}mma med specifikationerna f{\"o}r att undvika o{\"o}nskade systemfel. F{\"o}r n{\"a}rvarande saknas effektiva och skalbara metoder f{\"o}r att specificera och analysera systemkrav, och f{\"o}r att formellt analysera beteendemodellerna f{\"o}r inbyggda system. De flesta krav f{\"o}r inbyggda system uttrycks i naturligt spr{\"aa}k, vilket {\"a}r flexibelt och intuitivt men ofta tvetydigt och oprecist. F{\"o}rutom naturligt spr{\"aa}k anv{\"a}nds ofta mallbaserade kravspecifikationsmetoder f{\"o}r att specificera krav (speciellt i s{\"a}kerhetskritiska till{\"a}mpningar). {\"A}ven om de senare minskar otydligheten och f{\"o}rb{\"a}ttrar begripligheten, {\"a}r de vanligtvis oflexibla p{\"aa} grund av den begr{\"a}nsade syntaxen i mallarna, och mallvalet {\"a}r sv{\"aa}rt. Industriella system utvecklas ofta genom att anv{\"a}nda modellerings- och simuleringsmilj{\"o}er s{\"aa}som Simulink, som ocks{\"aa} anv{\"a}nds f{\"o}r att generera kod automatiskt f{\"o}r olika h{\"aa}rdvaruplattformar. D{\"a}rf{\"o}r {\"a}r det viktigt att kunna formellt analysera Simulink-modeller, f{\"o}r att f{\"aa} insikt i beteendet hos det inbyggda systemet, och f{\"o}r att f{\"o}rhindra potentiella fel fr{\"aa}n att sprida sig till implementationen. Att analysera tidsperspektivet f{\"o}r s{\"aa}dan s{\"a}kerhetskritisk mjukvara som har tasks med olika periodicitet och som har begr{\"a}nsningar p{\"aa} datas {\"aa}lder, dvs datans f{\"a}rskhet, f{\"o}r end-to-end-programvarufunktionalitet, {\"a}r inte trivialt. Detta orsakas av undersamplings- och {\"o}versamplingseffekter, som uppst{\"aa}r n{\"a}r data g{\"aa}r fr{\"aa}n h{\"o}gre till l{\"a}gre signaleringshastigheter och vice versa. Vidare, n{\"a}r s{\"aa}dana system anv{\"a}nds i en distribuerad arkitektur, t.ex. elektriska / elektroniska fordonssystem, s{\"aa} b{\"o}r, f{\"o}rutom att s{\"a}kerst{\"a}lla tidskraven, {\"a}ven tillf{\"o}rlitligheten hos den distribuerade mjukvaran maximeras f{\"o}r att motverka den h{\"o}gre risken f{\"o}r fel i den distribuerade databehandlingen, f{\"o}r att d{\"a}rigenom f{\"o}rb{\"a}ttra den {\"o}vergripande f{\"o}ruts{\"a}gbarheten f{\"o}r det s{\"a}kerhetskritiska systemet. Design f{\"o}r tillf{\"o}rlitlighet kr{\"a}ver emellertid vanligtvis mer av kritiska systemresurser, s{\"aa}som energi. F{\"o}r att tillgodose nuvarande och framtida mjukvarufunktionalitet b{\"o}r utformningen av det s{\"a}kerhetskritiska systemet ta h{\"a}nsyn till effektiviteten hos kritiska systemresurser, s{\"aa}som energif{\"o}rbrukning, samtidigt som kraven p{\"aa} tid och tillf{\"o}rlitlighet uppfylls.

F{\"o}r att m{\"o}ta ovanst{\"aa}ende behov, f{\"o}resl{\"aa}r vi i denna avhandling formella metoder och optimeringstekniker f{\"o}r att s{\"a}kerst{\"a}lla f{\"o}rb{\"a}ttrad kvalitet p{\"aa} kravspecifikationer och mjukvaruutveckling, och f{\"o}r att effektivt mappa mjukvarufunktionalitet till h{\"aa}rdvara. Avhandlingens bidrag {\"a}r: (i) \textit{ReSA} - ett dom{\"a}nspecifikt spr{\"aa}k f{\"o}r kravspecifikation, skr{\"a}ddarsytt f{\"o}r inbyggda system, baserat p{\"aa} begr{\"a}nsat naturligt spr{\"aa}k; (ii) ett formellt tillv{\"a}gag{\"aa}ngss{\"a}tt f{\"o}r att kontrollera konsistensen av \textit{ReSA}-specifikationer genom SAT och Ontologi; (iii) ett ramverk baserat p{\"aa} statistisk modellkontroll f{\"o}r att analysera Simulink-modeller via automatiserad omvandling till n{\"a}tverk av stokastiska tidsautomater; och (iv) en resurseffektiv f{\"o}rdelning av feltolerant programvara med end-to-end-tidskrav och drifts{\"a}kerhetsbegr{\"a}nsningar genom heltals-linj{\"a}r programmering och hybrid partikel-sv{\"a}rmoptimering. V{\"aa}ra f{\"o}reslagna l{\"o}sningar utv{\"a}rderas i fall som anv{\"a}nds i fordon, s{\"aa}som justerbar hastighetsbegr{\"a}nsare (ASL) och BBW-system fr{\"aa}n Volvo Group Trucks Technology (VGTT), och p{\"aa} ett motorstyrsystem fr{\"aa}n Bosch.


	%\vspace{1\baselineskip}
\vspace*{\stretch{1}}} % 

% Frontmatter commands
% Halftitle page for monographs
\newcommand{\halftitlepage}{%
\thispagestyle{empty}
\begin{center}
	ACTA UNIVERSITATIS UPSALIENSIS

	\emph{\series{}}
	
	\serialNumber
\end{center}
\cleardoublepage}

% Title page for monographs
\newcommand{\thetitlepage}%
{\thispagestyle{empty}
\vspace*{40mm}
\begin{center}
\Large \authorFirstName{} \authorSurname{}

\vspace{6mm}

\Huge{ \dissertationTitle{}}

\vspace{6mm}

\fontsize{14}{16}\fontshape{it}\selectfont\dissertationSubtitle{}			
\end{center}
\begin{figure}[b]
\begin{center}
\ifpdf
\includegraphics{UU_logo_pc_sv_42}
\else
\includegraphics{UU_logo_pc_sv_42.eps}
\fi
\end{center}
\end{figure}}

% Abstract. This is optional. 
\newcommand{\thesisabstract}{%
\begin{abstract}{%
	\newpage
	\thispagestyle{empty}
  \fontsize{9}{11}\selectfont
  \noindent
  \begin{flushleft}{\nohyphens
  Dissertation at M\"{a}lardalen University to be publicly examined in \placeOfDisputation{}, \dateOfDisputation{} at \timeOfDisputation{}
  for the Degree of Doctor of Philosophy. The examination will be conducted in \disputationLanguage{}
}\end{flushleft}

  \noindent\textbf{Abstract}
%  \vspace{-9pt}
%  \begin{flushleft}{\nohyphens\noindent\authorSurname{}, \authorFirstInitial{}. \yearOfPublication{}.
%  \dissertationTitle{}. \dissertationSubtitle{}. Acta Universitatis
%  Upsaliensis. \emph{\series{}} \serialNumber{}. \numberOfPages{}~pp.
%  \placeOfPublication{}. ISBN~\ISBN{}}\end{flushleft}

% Type your abstracttext here. Remove the latin text. Try to not type more than 300 words.
\noindent 

  \vspace{11pt}

  \noindent
  \emph{Keywords: }\keywords

  \vspace{11pt}

  \noindent
  \emph{\authorFirstName{} \authorSurname{}, \department{}, M\"{a}lardalen University,
  \departmentaddress{}}

  \vspace{11pt}

  \noindent \copyright{} \authorFirstName{} \authorSurname{}
  \yearOfPublication{}

  \vspace{11pt}
  
  \noindent ISSN \ISSN{}

  \noindent ISBN \ISBN{}

  \noindent \urn{} (http://urn.kb.se/resolve?urn=\urn)
}
\end{abstract}}

% Title page dummy for Compehensive summaries
\newcommand{\titlepagedummy}%
{\newpage
\newpage\thispagestyle{empty}
{\noindent\LARGE Design of Assured and Efficient Safety-critical Embedded Systems}\vspace{84pt}

\noindent This is the title page dummy. This page should be substituted for a real title page. The real title page will be provided by Publishing and Graphic services after having submitted your posting details in DiVA. The DiVA registration form can be found at \\ \href{https://uu.diva-portal.org/dream/}{https://uu.diva-portal.org/dream/}.

\vspace{1em}

\noindent More information about the publishing routines can be found at \\
\href{http://beta.ub.uu.se/pgs}{http://beta.ub.uu.se/pgs}}

%Abstractdummy
\newcommand{\abstractdummy}%
{\newpage\thispagestyle{empty}
{\noindent\LARGE Abstract page}\vspace{84pt}

This is the abstract dummy. This page should be substituted for real abstract/imprint page. The real abstract page will be provided by Publishing and Graphic services after having submitted your posting details in DiVA. The DiVA registration form can be found at \\ \href{https://uu.diva-portal.org/dream/}{https://uu.diva-portal.org/dream/}.
\vspace{1em}

%\noindent More information about the publishing routines can be found at \\
%\href{http://beta.ub.uu.se/pgs}{http://beta.ub.uu.se/pgs}
}

% Front matter commands
\newcommand{\frontmatterMonograph}%
{\halftitlepage\thetitlepage\abstractdummy\dedication}

\newcommand{\frontmatterCS}%
{\nasabstract\samman\dedication\ack\listofpapers}