% Filling in this bibliographic information facilitates the
% processing of this document.
% Insert linebreaks if necessary
% The abstract page is not generated by default. The reason for this is that it
% will be generated by the unit for Publishing and Graphic sevices from the metadata 
% template ("Spikningsmallen"). If you want to include the abstract page generated from this template,
% please edit the \frontmatterMonograph or \frontmatterCS command at the en of this file
%. Because of this, some of the bibliografic data on this page will not be used.

% Abstract and titelpage
\newcommand{\authorSurname}{Mahmud} % Your surname
\newcommand{\authorFirstName}{Nesredin} % Your given name
\newcommand{\authorFirstInitial}{N} % Initial of given name
\newcommand{\authorEmail}{nesredin.mahmud@mdh.se} % Your e-mail address
\newcommand{\dissertationTitle}{Design of Assured and Efficient Safety-critical Systems using Formal Methods and Optimization Techniques} % The title of the dissertation
\newcommand{\dissertationSubtitle}{Subtitle of the dissertation}% The subtitle of the dissertation (if there is any).
\newcommand{\yearOfPublication}{2019} % Year of publication
\newcommand{\placeOfDisputation}{M\"{a}lardalen University} % Place of disputation
\newcommand{\dateOfDisputation}{Thursday, June 13, 2019} % Date of disputation (Day, Month, Year)
\newcommand{\timeOfDisputation}{13:15} % Time of disputation (Swedish time format)
\newcommand{\disputationLanguage}{English} % Language the examination will be conducted in.
\newcommand{\numberOfPages}{51} % The page number of the last page
\newcommand{\placeOfPublication}{Uppsala} % The place of publication
\newcommand{\ISBN}{978-91-554-5436-4} % The ISBN number of the dissertation.
\newcommand{\series}{Uppsala Dissertations from the Faculty of Science and Technology} % The title of the series
\newcommand{\serialNumber}{1214} % The number in the series.
\newcommand{\keywords}{ embedded system, real-time system, safety-critical system, requirements specification, formal method, Simulink, integer-linear programming, metaheuristics} % Key words separated by comma
\newcommand{\department}{Department of Electronic Publishing} % Name of your department
\newcommand{\departmentaddress}{Villavagen 14, SE-752 36 Uppsala,
Sweden} % Address of your department
\newcommand{\ISSN}{1651-6214} % (ISSN for Digital comprehensive summaries of Uppsala dissertations from the faculty of science and technology)
\newcommand{\urn}{urn:nbn:se:uu:diva-3344} % URN number

% List of papers
\newcommand{\listofpapers}
{\cleardoublepage
\pdfbookmark[0]{List of Papers}{LOP}
\chapter*{List of Papers Included in the Thesis}
\noindent This thesis is based on the following papers:
\vspace{1\baselineskip}

        % It is possible to refer to the published papers using labels in the text.
        % Suggested order
        % Author 1 surname, Author 2 first name initial., Author 1 surname, Author 2 first name
        % initial. etc. (Year of publication) Paper main title.
        % Paper subtitle. Name of journal in italics, volume(number):page rage
        % Example
        
    \begin{description}	
      \item[Paper A]  ReSA: An ontology-based requirement specification language tailored to automotive systems. Nesredin~Mahmud, Cristina~Seceleanu and Ljungkrantz~Oscar.\textit{In the 10th IEEE International Symposium on Industrial Embedded Systems (SIES)(pp. 1-10). IEEE}.\label{lbl_resa}
    
      \item[Paper B]  ReSA tool: Structured requirements specification and SAT-based consistency-checking. Nesredin~Mahmud, Cristina~Seceleanu and Ljungkrantz~Oscar. \textit{In the 2016 Federated Conference on Computer Science and Information Systems (FedCSIS)(pp. 1737-1746). IEEE}.\label{lbl_resatool}
      
      \item[Paper C]  Specification and semantic analysis of embedded systems requirements: From description logic to temporal logic. Nesredin~Mahmud, Cristina~Seceleanu and Ljungkrantz~Oscar. \textit{In the International Conference on Software Engineering and Formal Methods (pp. 332-348). Springer, Cham.} Acceptance rate: 26\%.\label{lbl_resadl}
            
     \item[Paper D] SIMPPAAL - A Framework For Statistical Model Checking of Industrial Simulink Models. Predrag~Filipovikj, Nesredin~Mahmud, Raluca~Marinescu, Cristina~Seceleanu, Oscar~Ljungkrantz and Henrik~L\"{o}nn. \textit{Submitted to ACM Transaction on Software Engineering and Methodology (TOSEM). ACM Journals.}\label{lbl_simulink_ilp}
   
     \item[Paper E] Power-aware Allocation of Fault-tolerant Multirate AUTOSAR Applications. 
     Nesredin~Mahmud, Guillermo~Rodriguez-Navas, Hamid~Faragardi, Saad~Mubeen and Cristina~Seceleanu. \textit{In the 25th Asia-Pacific Software Engineering Conference (APSEC'18). IEEE.} \label{lbl_resa}. \label{lbl_softwareallocation_ilp}
     
	\item[Paper F] Optimized Allocation of Fault-tolerant Embedded Software with End-to-end Timing Constraints\\[6pt]
	Nesredin~Mahmud, Guillermo~Rodriguez-Navas, Hamid~Faragardi, Saad~Mubeen and Cristina~Seceleanu. Optimized Allocation of Fault-tolerant EmbeddedSoftware with End-to-end Timing Constraints. \textit{M\"{a}lardalen Real-time Research Center Technical Report (MRTC)}. Submitted to Elsevier JSA Journal. \label{lbl_softwareallocation_pso}
    \end{description}
\section*{\Large  List of Papers Not Included in the Thesis}
\vspace{1\baselineskip}
\begin{itemize}	
	\item Evaluating industrial applicability of virtualization on a distributed multicore platform. Nesredin Mahmud, Kristian Sandstr{\"o}m, and Aneta Vulgarakis.  \textit{In Proceedings of the 2014 IEEE Emerging Technology and Factory Automation (ETFA), pp. 1-8. IEEE, 2014.}
	\item The multi-resource server for predictable execution on multi-core platforms. Rafia Inam, Nesredin Mahmud, Moris Behnam, Thomas Nolte, and Mikael Sj{\"o}din. \textit{In 2014 IEEE 19th Real-Time and Embedded Technology and Applications Symposium (RTAS), pp. 1-12. IEEE, 2014.}
	\item Simulink to UPPAAL statistical model checker: Analyzing automotive industrial systems. Predrag Filipovikj, Nesredin Mahmud, Raluca Marinescu, Cristina Seceleanu, Oscar Ljungkrantz, and Henrik L{\"o}nn. \textit{In International Symposium on Formal Methods, pp. 748-756. Springer, Cham, 2016.}
\end{itemize}

\vspace{1\baselineskip} \noindent {\small Reprints were made with permission from
the publishers.}}


% Dedication. Dedication is optional. You can enter whatever feels right and use the fonts and images you like as long as they are embedded in the document. The suggested format does not have to be used.
\newcommand{\dedication}%
{\cleardoublepage
\thispagestyle{empty}
\vspace*{\stretch{3}}
\begin{flushright}
		
{\fontfamily{pzc}\Large\selectfont\emph{To My Parents}}

\end{flushright}
\vspace*{\stretch{1}}} % 

% Frontmatter commands
% Halftitle page for monographs
\newcommand{\halftitlepage}{%
\thispagestyle{empty}
\begin{center}
	ACTA UNIVERSITATIS UPSALIENSIS

	\emph{\series{}}
	
	\serialNumber
\end{center}
\cleardoublepage}

% Title page for monographs
\newcommand{\thetitlepage}%
{\thispagestyle{empty}
\vspace*{40mm}
\begin{center}
\Large \authorFirstName{} \authorSurname{}

\vspace{6mm}

\Huge{ \dissertationTitle{}}

\vspace{6mm}

\fontsize{14}{16}\fontshape{it}\selectfont\dissertationSubtitle{}			
\end{center}
\begin{figure}[b]
\begin{center}
\ifpdf
\includegraphics{UU_logo_pc_sv_42}
\else
\includegraphics{UU_logo_pc_sv_42.eps}
\fi
\end{center}
\end{figure}}

% Abstract. This is optional. 
\newcommand{\thesisabstract}{%
\begin{abstract}{%
	\newpage
	\thispagestyle{empty}
    \fontsize{9}{11}\selectfont
    \noindent
    \begin{flushleft}{\nohyphens
    Dissertation at M\"{a}lardalen University to be publicly examined in \placeOfDisputation{}, \dateOfDisputation{} at \timeOfDisputation{}
    for the Degree of Doctor of Philosophy. The examination will be conducted in \disputationLanguage{}
}\end{flushleft}

    \noindent\textbf{Abstract}
%    \vspace{-9pt}
%    \begin{flushleft}{\nohyphens\noindent\authorSurname{}, \authorFirstInitial{}. \yearOfPublication{}.
%    \dissertationTitle{}. \dissertationSubtitle{}. Acta Universitatis
%    Upsaliensis. \emph{\series{}} \serialNumber{}. \numberOfPages{}~pp.
%    \placeOfPublication{}. ISBN~\ISBN{}}\end{flushleft}

% Type your abstracttext here. Remove the latin text. Try to not type more than 300 words.
\noindent 

    \vspace{11pt}

    \noindent
    \emph{Keywords: }\keywords

    \vspace{11pt}

    \noindent
    \emph{\authorFirstName{} \authorSurname{}, \department{}, M\"{a}lardalen University,
    \departmentaddress{}}

    \vspace{11pt}

    \noindent \copyright{} \authorFirstName{} \authorSurname{}
    \yearOfPublication{}

    \vspace{11pt}
    
    \noindent ISSN \ISSN{}

    \noindent ISBN \ISBN{}

    \noindent \urn{} (http://urn.kb.se/resolve?urn=\urn)
}
\end{abstract}}

% Title page dummy for Compehensive summaries
\newcommand{\titlepagedummy}%
{\newpage
\newpage\thispagestyle{empty}
{\noindent\LARGE  Design of Assured and Efficient Safety-critical Embedded Systems}\vspace{84pt}

\noindent This is the title page dummy. This page should be substituted for a real title page. The real title page will be provided by Publishing and Graphic services after having submitted your posting details in DiVA. The DiVA registration form can be found at \\ \href{https://uu.diva-portal.org/dream/}{https://uu.diva-portal.org/dream/}.

\vspace{1em}

\noindent More information about the publishing routines can be found at \\
\href{http://beta.ub.uu.se/pgs}{http://beta.ub.uu.se/pgs}}

%Abstractdummy
\newcommand{\abstractdummy}%
{\newpage\thispagestyle{empty}
{\noindent\LARGE Abstract page}\vspace{84pt}

\begin{abstract}
	\noindent\textbf{Abstract}
	\noindent Safety-critical systems should be analyzed rigorously to remove software/specifications errors, i.e., their requirements specifications should be unambiguous, comprehensible and consistent, and the software design should conform to the specifications, hence avoiding undesirable system failures. In particular, analyzing the timing behavior of safety-critical software that is refined by multi-rate periodic tasks, and has data age constraints, i.e., freshness of data, across the input-output path of end-to-end software functionality, is not trivial due to the undersampling and oversampling effects, which results as data propagates from higher to lower rates and vice versa, respectively. Furthermore, when the safety-critical software is deployed on a distributed architecture, e.g., electrical/electronic vehicular system, besides assuring the timeliness, the reliability of the distributed software should be maximized to counter the higher risk of failures in the distributed computing, hence improving the overall predictability of the safety-critical system. However, reliability design usually requires additional critical system resources such as power and energy. Thus, to accommodate the growing complexity of software functionality, the design of the safety-critical systems should consider the efficient use of critical system resources such as power source while meeting the timing and reliability requirements.
    
    In this thesis, we propose formal methods and optimization techniques to assure improved quality of requirements specifications and software design, and to efficiently map software functionality to hardware. Contributions of the thesis are: (i) a constrained requirements specifications language of embedded systems, \textit{ReSA}; (ii) a formal analysis of ReSA specifications via SMT and Ontology; (iii) a statistical model checking of a software design, modeled in Simulink, via transformation to a network of stochastic timed automata; (iv) a resource efficient allocation of fault-tolerant software with end-to-end timing and reliability constraints via integer integer-linear programming and hybrid particle-swarm optimization. Our proposed solutions are evaluated on automotive use cases such as the adjustable-speed limiter (ASL) and the brake-by-wire (BBW) systems from Volvo Group Trucks Technology (VGTT), and on an engine management system benchmark from Bosch.

\end{abstract}
%This is the abstract dummy. This page should be substituted for real abstract/imprint page. The real abstract page will be provided by Publishing and Graphic services after having submitted your posting details in DiVA. The DiVA registration form can be found at \\ \href{https://uu.diva-portal.org/dream/}{https://uu.diva-portal.org/dream/}.
\vspace{1em}

%\noindent More information about the publishing routines can be found at \\
%\href{http://beta.ub.uu.se/pgs}{http://beta.ub.uu.se/pgs}
}

% Front matter commands
\newcommand{\frontmatterMonograph}%
{\halftitlepage\thetitlepage\abstractdummy\dedication}

\newcommand{\frontmatterCS}%
{\titlepagedummy\abstractdummy\dedication\listofpapers}