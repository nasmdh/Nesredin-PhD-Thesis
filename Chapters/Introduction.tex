\chapter{Introduction}\label{chapter_introduction}
\lettrine{R}{eal-time} systems are usually characterized by timely computations, which are bounded by \textit{deadlines} besides correct results of the computations~\cite{Buttazzo2003}. They are applied in many \textit{safety-critical embedded} systems, which are specialized computer systems designed for safety-critical applications~\cite{WangJiacun2017RES}, e.g., brake-by-wire, flight control. For instance, the brake-by-wire system~\cite{bibid} applies torque proportional to the pressing of the brake pedal in order to slow down (or halt) the vehicle. Besides computing the correct torque, the system must act timely, i.e., not too soon or not too late so that a traffic accident is avoided. Therefore, safety-critical real-time systems should be analyzed for functional as well as timing correctness, and in fact, they should be analyzed rigorously according to functiona safety standards, e.g., ISO 26262 ``Road vehicles-Functional safety''~\cite{iso201126262} recommends the use of \textit{formal methods}, which are mathematical techniques and tools that enable unambiguous specification, modeling and rigorous analysis~\cite{o2017concise}, to develop safety-critical automotive systems. In this thesis, we apply various formal methods early, i.e., at the requirements specification and software design levels, to improve the quality assurance of developing safety-critical embedded systems.

Over the last decades, a lot of functionality that have been provided by mechanical and electical systems have moved to software functionality, e.g., the x-by-wire technology~\cite{bibid}, and several software functionality, which are driven by innovation and safety standards, are integrated into safety-critical systems to provide advanced, safe and reliable services, e.g., the advanced driver-assistance systems, vision system for autonomous and self-driving vehicles. In response, several computing architectures have moved from a federated architecture (i.e., dedicated hardware systems executing safety-critial software systems or applications) into consolidated (i.e., multiple safety-critial software applications co-hosted on the same hardware) and distributed architectures. In distributed computing~\cite{Kopetz2003Real-timeApplications}, the safety-critical software is mapped on multiple hardware systems to capitalize on the computational power provided by the distributed architecture, e.g., the brake-by-wire software executing on multiple electronic control units (ECU) that are connected through a network bus, e.g., CAN~\cite{}. The distributed software, unlike in the case of federated computing, it is exposed to a greater degree of permanent and transient faults, therefore, reliability of the software should be maximized to improve the overall dependability~\cite{} of the  system. However, such measures requires additional critical system resources besides computational resources, e.g., power and energy, which are constrained in battery-driven embedded systems. Thus, the distributed safety-critical software should be deployed efficiently on the distributed architecture while meeting the timing and reliability requirements of the software.


In this thesis, we propose a design approach which considers rigoruous analysis and efficiency of safety-critical embedded systems by applying formal methods and optimization techniques at the early stages of software development. Thus, the safety-critical  specifications should be unambiguous, comprehensible, etc~\cite{ieereqspecstandard}, and the software design should conform to the specifications. Since natural languag is inherently ambigous~\cite{ieereqspecstandard}, many functional safety standards, e.g., ISO 26262, recommends semi-formal and formal specification methods to specify safety-critical embedded systems requirements, e.g., template-based languages, controlled natural language, temporal logic. The template-based specification methods, e.g., requirements boilerplates~\cite{Hull2011RequirementsEngineering}, property-specification systems~\cite{Dwyer1999PatternsVerificationb}, etc., lack meta-model to effectively create templates, and is usually cumbersome to select the templates. The controlled natural languages, e.g., Attempto~\cite{attempto96}\cite{Fuchs2008AttemptoRepresentation}, etc., renders the syntax and semantics of the natural language and have formal semantics, however lacks support for embedded systems, hence are less effective. Rather, we propose a constrained natural language which is domain-specific and uses the notion of boilerplates to facilitate reuse. The specifications have semantics in Boolean logic and description logic to enable rigorous analysis via Boolean satisfiability~\cite{Malik2009BooleanSuccess} and ontology~\cite{f25ea3c6f4b743cd90c150926bbcf3db}, respectively.

The requirements specifications are used in subsequent system development including software design to verify the latter for correct functionality. The software design is usually modeled, simulated and analyzed before implementation. In this regard, Simulink is one of the most widely used development environment for multi-domain, multi-rate, discrete and continuous safety-critical systems in industry~\cite{JamesB.Dabney2003MasteringSimulink}. For this main reason, there is increasing interest in formal analysis of Simulink models~\cite{Manamcheri2011AModels}. Simulink Design Verifier\footnote{Simulink Design Verifier - https://se.mathworks.com/products/sldesignverifier.html}, which is based on the Prover\footnote{Prover - https://www.prover.com/software-solutions-rail-control/formal-verification/} model checker, is the de~facto tool in the Simulink environment to formally verify Simulink design models. However, it has limited functionality, e.g., it supports only discrete models, has issues with scalability due to state-space explosion, and lacks verification of timed properties~\cite{Leitner2008SimulinkStudy}.  In contrast, we propose a scalable, timed analysis via a statistical model checking~\cite{Legay2010StatisticalOverview}, which uses traces of executions and statistical analysis techniques, e.g., monte-carlo simulation.

The software design should be mapped to hardware effectively, that is satisfying the timing and reliability requirements of the distributed safety-critical software, but also efficiently to minimize the power consumption of the distributed system to facilitate accommodating complex software functionality. We assume the software is scheduled using a fixed-priority preemptive policy, which is quite common in industry, and posses end-to-end timing requirements, e.g., the time duration between the brake-pedal press and the slow-down (or halt) of the vehicle. Furthermore,  we consider fault tolerance as a means to maximize reliability of the distributed safety-critical software by mapping redundant software functionality on different computing units.  We propose  \textit{exact} and \textit{heuristic} optimization methods, which deliver optimal and near-optimal solutions, respectively, to efficiently map the distributed safety-critical software to a network of computing nodes. Specifically, we propose a formulation of integer-linear programming (ILP)~\cite{Mahmud5222}, which is solved using branch and bound. Furthermore, we propose a hybrid-particle optimization~\cite{Mirjalili2019ParticleOptimisation}, which is a meta-heuristic algorithm, to solve the shortcomings of the exact method for large-scale problems~\cite{Mahmud2019Power-awareOptimization} with trade-off over optimality.

The main contributions of the thesis are: 
\begin{enumerate*}[label=(\roman*)]
\item formal analysis of natural language requirements - we propose a fairly expressive, flexible yet structured and domain-specific constrained natural language called \textit{ReSA}~\cite{resatool}\cite{Mahmud2015ReSA:Systems}. The language has semantics in Boolean and description logic to support shallow and rigorous analysis, respectively. The Boolean specifications are checked for consistency using the Boolean satisfiability via the Z3 SMT solver.  The ReSA tool is integrated seamlessly into EATOP to complement the requirements modeling of EAST-ADL. The language and its tool support are validated on the adjustable speed limiter (ASL) use case, which is a vehicle speed control system provided by Volvo Group Trucks Technology (VGTT);
%Whereas the description logic is used to encode the specifications as ontology, subsequently is checked for consistency at the lexical level via a semantic reasoner (i.e., HermiT as used in Prot{\'e}g{\'e} tool), which is a software program that produces logical consequencies from asserted axioms and facts.

\item scalable and formal analysis of Simulink models - we propose a pattern-based, execution-order preserving automatic transformation of a Simulink model into a network stochastic timed automata that can be formally analyzed using the UPPAAL statistical model checker (SMC)~\cite{Filipovikj2016SimulinkSystems}. The statistical model checker analyzes a state-transition system by conducting statistical analysis on the collected traces of the system executions~ \cite{Bulychev2012UPPAAL-SMC:Automata}, effectively mitigating the state-space explosion of (exact) model checking \cite{Legay2010StatisticalOverview}. Our proposed technique is validated on the brake-by-wire (BBW) use case, which is an industrial prototype provide for academic uses from VGTT;

\item efficient allocation of distributed safety-critical software applications - we propose an integer-linear programming (ILP) model and hybrid particle swarm optimization algorithms to allocate AUTOSAR software applications on a network of heterogenuous computing nodes with respect to processor speed, failure rate and power consumption specifications. The  ILP problem is implemented using ILOG CPLEX, which is a toolset for modeling and solving optimization problems. The proposed allocation methods are validated on an automotive benchmark developed according to AUTOSAR.
\end{enumerate*}


%Our solutions are evaluated on industrial automotive use cases and on a realistic benchmark. The formal analysis of the natural language requirements specifications in ReSA and the formal analysis of Simulink models are evaluated on the adjustable speed-limiter (ASL) and brake-by-wire (BBW) systems provided by Volvo Group Trucks Technology (VGTT). ASL is a speed-limitation automotive function which controls the vehicle speed of Volvo trucks from speeding up, and is useful in roads where speed-limitation signs are in place. The ASL use case consists of around 300 requirements, which are specified in natural language, architectural models in EAST-ADL and Simulink models. The integrated software allocation is evaluated on the engine management system benchmark provided by Bosch~\cite{} provided for AUTOSAR applications. The benchmark consists of statistics of the schedulable objects, such as mean values, shares of timing specifications and activation mechanisms of the schedulable objects in the system.

\section{Thesis Outline Overview}
The thesis is divided into two parts. The first part is a summary of our research. It is organized as follows: in Chapter 2, we give the background on the logic-based reasoning using Boolean and description logic, Simulink, UPPAL statistical model checking, and the optimization techniques based on ILP and PSO. In Chapter 3, we explain the research problem and outline the research goals. The thesis contributions are discussed in Chapter 4, which states the peer-review papers mappings to the contributions. In Section 5, we provide the related work on requirements specification, formal analysis of Simulink models, and software allocation. In Chapter 6, we describe the research method applied to conduct the research. Finally, in Chapter 7, we conclude the thesis, and outline the limitation and discuss possible directions for future work.

