\chapter{Introduction}\label{chapter_introduction}
\lettrine{R}{eal-time} systems are usually characterized by timely computations, which are bounded by \textit{deadline}, besides correct results of the computations~\cite{Buttazzo2003}. They are applied in many \textit{safety-critical embedded} systems, which are specialized computer systems designed for safety-critical applications~\cite{WangJiacun2017RES}, e.g., the braking system inside vehicles applies proportional force on the wheel to the pressing of a brake pedal in order to slow down (or halt) the vehicle, and must act within sometime otherwise the system fails, consequently, accident can happen. Therefore, safety-critical real-time systems should be analyzed rigorously for functional and timing correctness, which is also specified in the functional safety standards, such as the ISO 26262 ``Road vehicles-Functional safety''~\cite{iso201126262}. The latter standard also suggests the use of \textit{formal methods}, which are mathematical techniques and tools that enable unambiguous specification, modeling and rigorous analysis~\cite{o2017concise}, to develop safety-critical automotive systems.

In distributed computing~\cite{Kopetz2003Real-timeApplications}, the safety-critical software is mapped on multiple hardware systems to capitalize on the computational power provided by the distributed architecture, e.g., the braking software can be executed on multiple electronic control units (ECU). Since the distributed software is  normally exposed to a greater degree of permanent and transient faults, reliability of the safety-critical software should be maximized to improve dependability of the  system which requires additional critical systems resource such as power and energy besides computational resources. However, the embedded hardware is usually resource constrained, therefore, the software should be efficiently mapped to the hardware to conserve critical system resources, thereby accommodate current and future growth of the software functionality.

In this thesis, we apply formal methods to improve the requirements specifications of safety-critical systems, and to analyze the functional and timing behavior of the safety-critical software against the specifications. The safety-critical  specifications should be unambiguous, comprehensible, etc~\cite{ieereqspecstandard}. According to the ISO 26262 standard, semi-formal or formal languages are recommended to specify safety-critical requirements. However, natural language is the de~facto method to specify embedded systems requirements in industry because it is intuitive and expressive, though inherently ambiguous~\cite{ieereqspecstandard}. In the context of natural language, template-based specification and controlled natural language can be considered semi-formal and formal specification methods. The template-based specification methods, e.g., requirements boilerplates~\cite{Hull2011RequirementsEngineering}, property-specification systems~\cite{Dwyer1999PatternsVerificationb}, etc., lack meta-model to effectively create templates, and is usually cumbersome to select the templates. The controlled natural languages, e.g., Attempto~\cite{attempto96}\cite{Fuchs2008AttemptoRepresentation}, etc., renders the syntax and semantics of the natural language and have formal semantics, however lacks support for embedded systems, hence are less effective. In this thesis, we propose a constrained natural language which is domain-specific and uses the notion of boilerplates to facilitate reuse. The specifications have semantics in Boolean logic and description logic to enable rigorous analysis via Boolean satisfiability~\cite{Malik2009BooleanSuccess} and ontology~\cite{f25ea3c6f4b743cd90c150926bbcf3db}, respectively.

The specifications are employed in subsequent system development including software design to verify the latter for correct functionality. The software design is usually modeled, simulated and analyzed before implementation. In this regard, Simulink is one of the most widely used development environment for multi-domain, multi-rate, discrete and continuous safety-critical systems in industry~\cite{JamesB.Dabney2003MasteringSimulink}. For this main reason, there is increasing interest in formal analysis of Simulink models~\cite{Manamcheri2011AModels}. Simulink Design Verifier\footnote{Simulink Design Verifier - https://se.mathworks.com/products/sldesignverifier.html}, which is based on the Prover\footnote{Prover - https://www.prover.com/software-solutions-rail-control/formal-verification/} model checker, is the de~facto tool in the Simulink environment to formally verify Simulink design models. However, it has limited functionality, e.g., it supports only discrete models, has issues with scalability due to state-space explosion, and lacks verification of timed properties~\cite{Leitner2008SimulinkStudy}.  In contrast, we propose a scalable, timed analysis via a statistical model checking~\cite{Legay2010StatisticalOverview}, which uses traces of executions and statistical analysis techniques, e.g., monte-carlo simulation, etc., first by transforming Simulink models into a network of stochastic timed automata using timed-automata patterns~\cite{Filipovikj2018SimppaalModels}.

The software design should be mapped to hardware effectively, that is satisfying the timing and reliability requirements of the distributed safety-critical software, but also efficiently to minimize the power consumption of the distributed system to facilitate extensibility of the software, and also accommodates the demand from the increasing software functionality. We consider the software is scheduled using a fixed-priority preemptive policy, which is quite common in industry, and posses end-to-end timing requirements, e.g., the time duration between the brake-pedal press and the slow-down (or halt) action. Furthermore,  we consider the fault tolerance as a means to maximize reliability of the distribute safety-critical software by mapping redundant software functionality on different computing units.  We propose  \textit{exact} and \textit{heuristic} optimization methods, which deliver optimal and near-optimal solutions, respectively, to efficiently map the distributed safety-critical software to a network of computing units. Specifically, we propose a formulation of integer-linear programming (ILP)~\cite{Mahmud5222}, which is solved using branch and bound. Furthermore, we propose a hybrid-particle optimization~\cite{Mirjalili2019ParticleOptimisation}, which is a meta-heuristic algorithm, to solve the shortcomings of the exact method for large-scale problems~\cite{Mahmud2019Power-awareOptimization} with trade-off over non-optimality.

%\section{Research Contributions Overview}
%In this subsection, we give overview of the thesis contributions, and later in Section x, the contributions are further discussed in detail.
\begin{itemize}
\item \textbf{Formal Analysis of natural language requirements:}  we propose a fairly expressive, flexible yet structured and domain-specific constrained natural language, called \textit{ReSA}~\cite{resatool}\cite{Mahmud2015ReSA:Systems}. The language has semantics in Boolean and description logic to support for shallow and rigorous analysis, respectively. The Boolean specifications are checked for consistency using the satisfiability-modulo theory via the Z3 SMT solver. Whereas, the description logic is used to encode the specification as ontology, where we check consistency of the specifications at the lexical level using Reasoner (Inference engine) such Hermit. The ReSA tool, which consists of an editor and implements consistency-checking functionality, is integrated seamlessly into EATOP, which is an open source EAST-ADL IDE, to complement the requirements modeling. 

\item \textbf{Scalable analysis of Simulink models:} we propose a pattern-based, execution-order preserving automatic transformation of atomic and composite Simulink blocks into stochastic timed automata that can be formally analyzed using UPPAAL Statistical Model Checker \cite{Bulychev2012UPPAAL-SMC:Automata}. Our method is scalable, and has been validated on industrial use cases \cite{Filipovikj2016SimulinkSystems}. The statistical model checker analyzes a state-transition system by conducting statistical analysis on the collected traces of the system executions, effectively mitigating the state-space explosion of (exact) model checking \cite{Legay2010StatisticalOverview}. 

\item \textbf{Efficient Power consumption ILP and metaheuristics:} we propose an integer-linear programming (ILP) model to the allocation of distributed software on the network of heterogeneous computing units, which have different processor speed, failure rate and power consumption specifications. The ILP implemented in JAVA using the ILOG CPLEX interface, and subsequently solved the CPLEX solver.
\item \textbf{Validation on industrial use cases: } 
Our contributions such as its the ReSA language as well as the proposed formal analysis of Simulink model is validated on industrial use cases, which are provided
\end{itemize}


Our solutions are evaluated on industrial automotive use cases and on a realistic benchmark. The formal analysis of the natural language requirements specifications in ReSA and the formal analysis of Simulink models are evaluated on the adjustable speed-limiter (ASL) and brake-by-wire (BBW) systems provided by Volvo Group Trucks Technology (VGTT). ASL is a speed-limitation automotive function which controls the vehicle speed of Volvo trucks from speeding up, and is useful in roads where speed-limitation signs are in place. The ASL use case consists of around 300 requirements, which are specified in natural language, architectural models in EAST-ADL and Simulink models. The integrated software allocation is evaluated on the engine management system benchmark provided by Bosch~\cite{} provided for AUTOSAR applications. The benchmark consists of statistics of the schedulable objects, such as mean values, shares of timing specifications and activation mechanisms of the schedulable objects in the system.

\section{Thesis Outline Overview}
The thesis is divided into two parts. The first part is a summary of our research. It is organized as follows: in Chapter 2, we give the background information on description logic, Boolean satisfiability problem, Simulink, stochastic timed automata, and meta-heuristic optimization. In Chapter 3, we explain the research problem and outline the research goals. The thesis contributions are discussed in Chapter 4, followed by the related work in Chapter 5. In Chapter 3, we describe the research method applied to conduct the research. Finally, in Chapter 7, we conclude the thesis and outline possible directions for future work.
\subsection*{Paper A}
ReSA: An ontology-based requirement specification language tailored to automotive systems. Nesredin~Mahmud, Cristina~Seceleanu and Oscar~Ljungkrantz.\textit{In the 10th IEEE International Symposium on Industrial Embedded Systems (SIES)(pp. 1-10). IEEE, 2015}.\label{lbl_resa}\\[6pt]
\textbf{Abstract:} \textit{Automotive systems are developed using multi-leveled architectural abstractions in an attempt to manage the increasing complexity and criticality of automotive functions. Consequently, well-structured and unambiguously specified requirements are needed on all levels of abstraction, in order to enable early detection of possible design errors. However, automotive industry often relies on requirements specified in ambiguous natural language, sometimes in large and incomprehensible documents. Semi-formal requirements specification approaches (e.g., requirement boilerplates, pattern-based specifications, etc.) aim to reduce requirements ambiguity, without altering their readability and expressiveness. Nevertheless, such approaches do not offer support for specifying requirements in terms of multi-leveled architectural concepts, nor do they provide means for early-stage rigorous analysis of the specified requirements. In this paper, we propose a language, called ReSA, which allows requirements specification at various levels of abstraction, modeled in the architectural language of EAST-ADL. ReSA uses an automotive systems' ontology that offers typing and syntactic axioms for the specification. Besides enforcing structure and more rigor in specifying requirements, our approach enables checking refinement as well as consistency of requirements, by proving ordinary boolean implications. To illustrate ReSA's applicability, we show how to specify some requirements of the Adjustable Speed Limiter, which is a complex, safety-critical Volvo Trucks user function.}\\[6pt]
\textbf{Personal Contributions: }I was the main driver of the paper. I developed the ReSA language including its syntax and semantics, and Cristina Seceleanu proposed a consistency analysis technique besides giving useful comments and ideas on the design of the language. Oscar Ljungkrantz provided useful materials from VGTT that were eventually analyzed for the language development, and gave feedback on the language design and implementation from an industrial viewpoint.\\
\textbf{Status:} Published

\subsection*{Paper B}
ReSA tool: Structured requirements specification and SAT-based consistency-checking. Nesredin~Mahmud, Cristina~Seceleanu and Oscar~Ljungkrantz. \textit{In the 2016 Federated Conference on Computer Science and Information Systems (FedCSIS)(pp. 1737-1746). IEEE, 2016.}\label{lbl_resatool}\\[6pt]
\textbf{Abstract:} \textit{Most industrial embedded systems requirements are
	specified in natural language, hence they can sometimes be
		ambiguous and error-prone. Moreover, employing an early-stage
		model-based incremental system development using multiple
		levels of abstraction, for instance via architectural languages
		such as EAST-ADL, calls for different granularity requirements
		specifications described with abstraction-specific concepts that
		reflect the respective abstraction level effectively.
		In this paper, we propose a toolchain for structured requirements
		specification in the ReSA language, which scales to multiple
		EAST-ADL levels of abstraction. Furthermore, we introduce
		a consistency function that is seamlessly integrated into the
		specification toolchain, for the automatic analysis of requirements
		logical consistency prior to their temporal logic formalization
		for full formal verification. The consistency check subsumes
		two parts: (i) transforming ReSA requirements specification into
		boolean expressions, and (ii) checking the consistency of the
		resulting boolean expressions by solving the satisfiability of their
		conjunction with the Z3 SMT solver. For validation, we apply
		the ReSA toolchain on an industrial vehicle speed control system,
		namely the Adjustable Speed Limiter.}\\[6pt]%abstract
	\textbf{Personal Contributions: }I was the main driver of the paper. I developed the ReSA toolchain that consists of the editor and the consistency checker including the integration with the Z3 SAT solver in the backend. Cristina Seceleanu formulated the consistency checking and together with Oscar Ljungkrantz, they contributed to the paper with useful comments and ideas.\\
		\textbf{Status: }Published

\subsection*{Paper C}
	Specification and semantic analysis of embedded systems requirements: From description logic to temporal logic. Nesredin~Mahmud, Cristina~Seceleanu and Oscar~Ljungkrantz. \textit{In the International Conference on Software Engineering and Formal Methods (SEFM)(pp. 332-348). Springer, Cham, 2017.}\label{lbl_resadl}\\[6pt]%authors
	\textbf{Abstract:} \textit{Due to the increasing complexity of embedded systems, early detection of software/hardware errors has become desirable. In this context, effective yet flexible specification methods that support rigorous analysis of embedded systems requirements are needed. Current specification methods such as pattern-based, boilerplates normally lack meta-models for extensibility and flexibility. In contrast, formal specification languages, like temporal logic, Z, etc., enable rigorous analysis, however, they usually are too mathematical and difficult to comprehend by average software engineers. In this paper, we propose a specification representation of requirements, which considers thematic roles and domain knowledge, enabling deep semantic analysis. The specification is complemented by our constrained natural language specification framework, ReSA, which acts as the interface to the representation. The representation that we propose is encoded in description logic, which is a decidable and computationally-tractable ontology language. By employing the ontology reasoner, Hermit, we check for consistency and completeness of requirements. Moreover, we propose an automatic transformation of the ontology-based specifications into Timed Computation Tree Logic formulas, to be used further in model checking embedded systems.}\\[6pt]%abstract
	\textbf{Personal Contributions:} I was the main driver of the language. I developed the ReSA language semantics using event-base approach, which is encoded in description logic. Cristina~Seceleanu and Ljungkrantz~Oscar provided with useful ideas and comments.\\
	\textbf{Status:} Published

\subsection*{Paper D}
SIMPPAAL - A Framework For Statistical Model Checking of Industrial Simulink Models. Predrag~Filipovikj, Nesredin~Mahmud, Raluca~Marinescu, Cristina~Seceleanu, Oscar~Ljungkrantz and Henrik~L\"{o}nn. \textit{Submitted to ACM Transactions on Software Engineering and Methodology (TOSEM).}\label{lbl_simulink_ilp}
\\[3pt]{\footnotesize This article is an extended version of the following conference paper:
     Simulink to UPPAAL Statistical Model Checker: Analyzing Automotive Industrial Systems.
Predrag Filipovikj, Nesredin Mahmud, Raluca Marinescu, Cristina Seceleanu, Oscar Ljungkrantz, Henrik L{\"o}nn. In Proceedings of the 21st International
Symposium on Formal Methods (FM2016), pages 748-756. Limassol, Cyprus. Springer, LNCS, November 2016. Revisions required. }\\[6pt]%authors
	\noindent \textbf{Abstract:} \textit{The evolution of automotive systems has been rapid. Nowadays, electronic brains control dozens of functions in vehicles, like
		braking, cruising, etc. Model-based design approaches, in environments such as MATLAB Simulink, seem to help in addressing
		the ever-increasing need to enhance quality, and manage complexity, by supporting functional design from a set of block
		libraries, which can be simulated and analyzed for hidden errors, but also used for code generation. For this reason, providing
		assurance that Simulink models fulfill given functional and timing requirements is desirable. In this paper, we propose a
		pattern-based, execution-order preserving automatic transformation of atomic and composite Simulink blocks into stochastic
		timed automata that can then be formally analyzed with Uppaal Statistical Model Checker (Uppaal SMC). To enable this, we
		first define the formal syntax and semantics of Simulink blocks and their composition, and show that the transformation is
		provably correct for a certain class of Simulink models. Our method is supported by the SIMPPAAL tool, which we introduce
		and apply on two industrial Simulink models, a prototype called the Brake-by-Wire and an operational Adjustable Speed
		Limiter system. This work enables the formal analysis of industrial Simulink models, by automatically generating stochastic
		timed automata counterparts.}\\[6pt]%abstract
	\textbf{Personal Contributions: } The three co-authors contributed equally to writing the paper. Technically, I equally contributed with proposing the pattern-based semantics of Simulink blocks, together with Predrag Filipovikj. I introduced a mechanism to enforce the execution order of the blocks using inter-arrival times. Predrag implemented the flattening algorithm and the tool for the automatic transformation of Simulink models into a network of timed automata with stochastic semantics. Raluca Marinescu contributed with analyzing the BBW system, Cristina Seceleanu contributed with defining the methodology, and with useful ideas and comments. Guillermo Rodriguez-Navas wrote the related work section. The industrial coauthors provided the use cases and commented on the final draft.\\
	\noindent\textbf{Status:} Revisions required. 

\subsection*{Paper E}
Power-aware Allocation of Fault-tolerant Multirate AUTOSAR Applications.
     Nesredin~Mahmud, Guillermo~Rodriguez-Navas, Hamid~Faragardi, Saad~Mubeen and Cristina~Seceleanu. \textit{In the 25th Asia-Pacific Software Engineering Conference (APSEC'18). IEEE.}  
     \label{lbl_softwareallocation_ilp}
\\[6pt]%authors
\textbf{Abstract:} \textit{The growing complexity of automotive functionality has attracted revolutionary computing architectures such as mixed-criticality design, which enables effective consolidation of software applications with different criticality on a shared execution platform. Mixed-critical design that is required to satisfy end-to-end timing and reliability specifications should consider power-efficient software design in order to accommodate more and more functionality. Due to the recursive and exhaustive nature of the real-time and reliability analysis, exact methods, e.g., branch and bound, dynamic programming, are prohibitively expensive. We propose hybrid particle-swarm optimization algorithms based on differential evolution and hill-climbing algorithms to minimize power consumption of the safety-critical software, which have end-to-end timing and reliability requirements, on a network of heterogeneous computing units. The optimization approach employs fault tolerance to maximize reliability of the software applications subsequently meet the reliability requirements. Our proposed integrated software-allocation approach is evaluated using a range of synthetic software applications based a real-world automotive benchmark. The evaluation makes comparative analysis of the differential evolution, particle-swarm optimization, integer-linear programming and hybrid particle-swarm optimization algorithms. The results show that the hybrid algorithms based on the hill-climbing algorithms outperform the rest of the meta-heuristic algorithms, in particular, the stochastic version of the hill-climbing algorithm scales well in large software allocation optimization problems while its overall optimality performance can be deemed acceptable.
}\\[6pt]%abstract
\textbf{My Contributions: } I was the main driver of the paper. I developed the system model with the guidance of the co-authors, and formulated the optimization problem with the guidance of Hamid~Faragardi, implemented the the problem in Java, and collected and analyzed the experimental results. The co-authors gave writing updates, useful ideas and comments on the paper, and specifically: Guillermo~Rodriguez-Navas on reliability analysis, Hamid~Faragardi on optimization, Saad~Mubeen on the timing analysis and Cristina~Seceleanu on the objective function and constraints.\\
\textbf{Status:} Published

\subsection*{Paper F}
Optimized Allocation of Fault-tolerant Embedded Software with End-to-end Timing Constraints.	\textit{M\"{a}lardalen Real-time Research Center Technical Report (MRTC), ISRN MDH-MRTC-325/2019-1-SE, 2019}. Submitted to Elsevier Journal of Systems Architecture (JSA).\\[6pt]%authors
\textbf{Abstract:} \textit{It is desirable to optimize power consumption of distributed safety-critical software that realize fault tolerance and maximize reliability as a result, to support the increasing complexity of software functionality in safety-critical embedded systems. Likewise, safety-critical applications that are required to meet end-to-end timing constraints may require additional computing resources. In this paper, we propose a scalable software-to-hardware allocation based on hybrid particle-swarm optimization with hill-climbing and differential algorithms to efficiently map software components to a network of heterogeneous computing nodes while meeting the timing and reliability constraints. The approach assumes fixed-priority preemptive scheduling, and delay analysis that value freshness of data, which is typical in control software applications.
Our proposed solution is evaluated on a range of software applications, which are synthesized from a real-world automotive AUTOSAR benchmark. The evaluation makes comparative analysis of the different algorithms, and a solution based on integer-linear programming, which is an exact method. The results show that the hybrid with the hill-climbing algorithms return very close solutions to the exact method and outperformed the hybrid with the differential algorithm, though consumes more time. The hybrid with the stochastic hill-climbing algorithm scales better and its optimality can be deemed acceptable.}\\[6pt]%abstract
\textbf{Personal Contributions: } I was the main driver of the paper. I developed the system model with the guidance of the co-authors, and formulated the optimization problem with the guidance of Hamid~Faragardi, implemented the the problem in Java, and collected and analyzed the experimental results. The co-authors gave writing updates, useful ideas and comments on the paper, and specifically: Guillermo~Rodriguez-Navas on reliability analysis, Hamid~Faragardi on optimization, Saad~Mubeen on the timing analysis and Cristina~Seceleanu on the objective function and constraints. \\%my contribution
\textbf{Status:} Submitted to Journal of System Architecture (JSA), Elsevier Journals.\\

