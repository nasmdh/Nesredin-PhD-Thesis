\chapter{Problem Formulation}\label{ch_problem_formulation}
Over the last decades, the complexity of the safety-critical software has been rising in most embedded systems. This is easily evident in modern cars, which implement many and complex automotive functions, as well as with the emergence of the electrical and autonomous vehicles. Thus, the thesis is motivated by the need for advanced (or rigorous) methods to specify, model and analyze complex safety-critical automotive software, and their seamless integration into industrial practice, at heavy-vehicle manufacturers such as VGTT and Scania. Furthermore, the thesis is motivated by the need for efficient mapping of safety-critical software to hardware in a distributed computing setting to facilitate software extensibility and support the increasing functionality of the automotive systems. 

The \textit{overall goal} of the thesis is as follows:
\begin{mybox}[attach title to upper={\ ---\ }]{Overall Goal}
	Provide assurance of functionality and quality of embedded software systems, at various levels of abstraction, via formal analysis and optimization of critical system resources.
\end{mybox}

The overall goal is refined via \textit{research goals}, which narrow the overall goal by splitting it into smaller and more concrete goals deemed essential to achieve the overall goal. 

\section{Research Goals}\label{research_challenges}
Many safety-critical automotive systems are developed according to the ISO 26262 standard, which recommends highly the use of semi-formal languages to specify safety-critical requirements to improve quality of the specifications, e.g., by reducing ambiguity and improving comprehensibility. In the context of textual representations, the semi-formal specification methods are constrained natural languages, such as templates (requirements boilerplates~\cite{Farfeleder2011DODT:Development}\cite{Mahmud2015ReSA:Systems}), controlled natural languages~\cite{Kuhn2014ALanguages}(e.g., Attempto~\cite{attempto96}).

The template-based methods inherently lack meta-model (or grammar), therefore, it is difficult to add new templates effectively, moreover, template selection is usually cumbersome. The existing controlled natural languages lack effective support of specifying embedded systems requirements.

Thus, the first research goal is defined as follows: 
\setcounter{rgcounter}{1}
\begin{researchgoal}
Reduce ambiguity and improve the comprehensibility of natural-language requirements using domain-specific knowledge of embedded systems.
\end{researchgoal}

One of the mechanisms to improve natural language specifications is by constraining the language, including its syntax, semantics and the lexicon \cite{Kuhn2014ALanguages}. The design of a constrained natural language for the specification of requirements is not trivial. By constraining the language, its expressiveness and intuitiveness can be impaired~\cite{ieereqspecstandard}\cite{Myachykov2013SyntacticRussian}, therefore, appropriate trade offs should be made during the design in order to have a robust and effective specification language.

Besides improving quality of individual requirements, the latter should be analyzed in ensemble in order to detect errors that span multiple specifications (like logical contradictions). However, natural language lacks formal (or precise and unambiguous) semantics, therefore is difficult to rigorously analyze natural-language requirements specifications. There are several methods to natural language semantics, for which the use of \textit{logic} is common~\cite{Clark2010TheProcessing}.

Therefore, we define the second research goal as follows: 
\begin{researchgoal}
Facilitate logic-based analysis of embedded systems requirements specifications.
\end{researchgoal}

Natural language specifications are constructed from syntactic units, such as words, phrases, clauses, statements, etc. Consequently, the rigorous analysis of specifications involve parsing and interpreting the syntactic units, which is a complex problem in computational linguistics~\cite{Clark2010TheProcessing}. The depth of the interpretation (or semantics) affects the applicability of the methods a great deal, for instance, the propositional logic representation of the specifications is simple and the analysis scales well, however, it is shallow as it abstracts away the details. On the other hand, the first-order-logic representations are more rigorous, thus enable thorough analysis, yet less tractable. Therefore, an appropriate interpretation of the natural language specifications is crucial.

The software designs and software-design units (or behavioral models) should conform to the requirements specifications. We consider the software-design units are modeled in Simulink, which is the most widely used model-based development environment in industry used for modeling and simulating the behavior of multi-domain, discrete, and continuous embedded systems. Simulink also enables the generation of code from discrete Simulink models which directly execute on specific platforms.

Consequently, it is crucial to conduct rigorous analysis of Simulink models to reduce the number of errors introduced in the generated code. The de~facto Simulink analysis techniques by type checking, simulation, and formal verification via Simulink Design Verifier (SDV\footnote{https://se.mathworks.com/products/sldesignverifier.html}) are not sufficient to address the full correctness of safety-critical real-time Simulink models. SDV lacks support for checking temporal correctness as specified in timed properties, e.g., in TCTL, and it also lacks support for verifying continuous models and suffers from scalability due to its reliance on the exhaustive model-checking~\cite{Leitner2008SimulinkStudy}. In contrast to exhaustive model checking, the statistical model-checking verifies properties over sufficiently many traces of system simulations collected via statistical methods. Such analysis approach makes the technique highly scalable, if compared to exhaustive model checking.  

Our third research goal is formulated as follows:
\begin{researchgoal}
Enable formal analysis of large-scale, multi-rate and hybrid Simulink models using statistical model-checking.
\end{researchgoal}

Simulink consists of connected and hierarchical Simulink blocks, which encode mathematical functions~\cite{JamesB.Dabney2003MasteringSimulink}. For industrial systems, the number of blocks in a Simulink model can be in the order of thousands, and the blocks can be triggered with different sampling frequencies for discrete blocks and without any sampling frequency 
for continuous blocks. Therefore, typical industrial Simulink models are usually complex and comprise mixed signals, multiple rates, discrete and continues Simulink blocks, making the model checking challenging.

In distributed architectures, the automotive software is allocated on multiple computing units (or ECU), it is exposed to permanent and transient faults, hence the need to maximize the reliability of the safety-critical software system. Fault tolerance using redundancy is the most popular approach to improve reliability by replicating software functionality on multiple ECU. However, designing for fault tolerance requires additional critical system resources such as power sources. In this regard, the software-to-hardware allocation plays a crucial role in minimizing the power consumption of fault-tolerant distributed safety-critical software, without breaching the timing and reliability constraints of the safety-critical software.

Given the above challenges, we formulate the fourth research goal as follows:
\begin{researchgoal}
 Minimize the power consumption of distributed safety-critical software while satisfying the timing and reliability constraints during software allocation.
\end{researchgoal}

The software allocation model is not trivial as we consider an exact method of schedulability analysis and reliability calculations, which makes the optimization complex. We consider the worst-case response time analysis~\cite{Baruah2011Response-timeSystems}\cite{Davis2007ControllerRevised} to check schedulability of the tasks, and assume age constraints on the cause-effect chains~\cite{mubeen2013support}, which is practical but computationally expensive. For the reliability analysis, we apply an exact method of calculation based on the state enumeration, in contrast to the series-parallel method, which is trivial and computationally less expensive. 

In order to show the validity of our proposed solutions, working prototypes should be developed, and evaluated on on industrial uses cases. The validation should consider scalability and engineer-friendliness of methods and tools, besides effectiveness. 

Thus, the last research goal is:
\begin{researchgoal}
 Provide automated and engineering-friendly support for the requirements specification, and quality checking, formal analysis of Simulink models, and software allocation of embedded applications. 
\end{researchgoal}

The  integration of our proposed methods and tools into the existing development process requires close cooperation between the domain experts and the practitioners. The role of the domain experts should be to simplify usage of the tools, for instance by rendering their interface to existing ones, and the practitioners should cooperate with materials that assist the validation of the proposed solutions. The cooperation is not trivial considering the challenge of formal methods, and companies culture of being reserved in applying them. In this thesis, we also target the integration of some thesis results into in-house tools at VGTT.
