\chapter{Problem Formulation}
The thesis is motivated by the need for advanced (or rigorous) requirements specification, modeling and analysis of safety-critical automotive systems, essentially to improve the existing methods and tools of automotive systems development at VGTT and Scania.

It is also inspired by the increasing complexity of automotive functionality implemented by an electrical/electronic system inside a modern truck, which is resource constrained, hence the need for efficient use of critical system resources such as power and energy besides computation and communication resources. 
Thus, the \textit{overall goal} of the thesis is to:
\begin{center}
	\ogl{provide assurance and extensibility of safety-critical system design, at the various levels of abstraction, via formal analysis and optimization techniques}
\end{center}

The overall goal is refined via \textit{research goals}, which state the needs or concerns that the thesis should address and are formulated as follows:

\section{Research Goals}\label{research_challenges}
Safety-critical automotive systems are developed according to the ISO 26262 standard, including the development process, methods and tools, etc. The standard requires the use of semi-formal specification languages to specify requirements less ambigous and comprehensible specifications, which are usually constrained natural languages, such as templates, e.g., requirements boilerplates, etc., and controlled natural languages, e.g., Attempto, PING, etc.

The template-based methods inherently lack meta-model (or grammar), therefore is difficult to add new templates effectively, moreover, template selection is usually cumbersome. The existing controlled natural languages lack effective support of specifying embedded systems requirements.

Thus, the first research goal is to:
\setcounter{rgcounter}{1}
\begin{researchgoal}
\rgl{reduce ambiguity and improve the comprehensibility of natural-language requirements using domain-specific knowledge of embedded systems. }
\end{researchgoal}

One of the mechanisms to improve natural language specifications is by constraining the language, including its syntax, semantics and the lexicon \cite{Kuhn2014ALanguages}. The design of a constrained natural language for the specification of requirements is not trivial mainly because: i) by constraining the language, its expressiveness and intuitiveness can be impaired~\cite{ieereqspecstandard}\cite{Myachykov2013SyntacticRussian}, therefore, appropriate trade-offs should be made during the design in order to have a robust and effective specification language; ii) domain knowledge/expertise is highly needed.

Requirements should be analyzed in ensemble in order to detect errors that span multiple specifications, e.g., logical contradictions. However, natural language lacks formal (or precise and unambiguous) semantic, therefore is difficult to rigoruously analyze (or reason) natural-language requirements specifications. There are several methods to natural language semantics, of which logic is the most applied method. Thus, the second research goal is to:
\begin{researchgoal}
 \rgl{facilitate formal analysis of the requirements specifications through transformation to Boolean and description logics}
\end{researchgoal}

Natural language specifications are constructed from syntactic units, such as words, phrases, clauses, statements, etc. Consequently, rigorous analysis of specifications involve parsing and interpreting the syntactic units, which is a complex problem in computational linguistics~\cite{Clark2010TheProcessing}. The depth of the interpretation greatly affects the applicability of the methods, e.g., the propostional logic representation is simple and the analysis scales well, however, it is shallow as it abstracts the details. On the other hand, first-orde-logic representations are more rigor, thus enable thorough analysis, but are less tractable. Therefore, proper use of the methods is crucial to benefit from the semantics.

The software designs and software-design units (or behavioral models) should conform to the requirements specifications. In this thesis, we consider the software-design units are modeled in Simulink, which is the most widely used model-based development environment in industry to model and simulate the dynamics of multi-domain, discrete, continous embedded systems. Simulink also supports the generation of code from discrete Simulink models that directly execute on specific platforms, thus is crucial to conduct rigoruous analysis of Simulink models to reduce errors introduced at generated code. 

The automotive Simulink models that we encounter at VGTT and Scania are in the scale of thousands blocks and are realize complex safety-critical functionality. %Therefore, they should be analyzed for funcional and timing specification errors agains the requirements specifications besides the basic analysis using Simulation. 
The de~facto Simulink analysis techniques, e.g., by type checking, simulation, and formal verification via the Simulink Design Verifier (SDV\footnote{https://se.mathworks.com/products/sldesignverifier.html}) are not sufficient to address the full correctness of safety-critical real-time Simulink models. SDV lacks support for checking temporal correctness as specified in timed properties, e.g., in TCTL, and also lacks support for verifying continuous models and suffers from scalability due to its reliance on the exact model-checking \cite{Leitner2008SimulinkStudy}. In contrast to exact model checking, statistical model-checking collects sufficient traces of system simulations, and consequently applies statistical methods to verify properties. It scales better, and the accuracy of the analysis can be improved by taking many traces of simulations. Thus, the fourth research goal of the thesis is to:
\begin{researchgoal}
\rgl{enable scalable formal analysis of multi-rate and hybrid Simulink models using statistical model-checking}
\end{researchgoal}

Simulink consists of connected and hierarchical Simulink blocks, which encode mathematical functions~\cite{JamesB.Dabney2003MasteringSimulink}. For industrial systems, the number of blocks in a Simulink model can be in the order of thousands, and the blocks can be triggered with different sampling frequencies for discrete blocks and without any sampling frequency 
for continuous blocks. Therefore, typical industrial Simulink models are usually complex and comprise mixed signals, multiple rates, discrete and continues Simulink blocks, making formal analysis challenging.

In the distributed computing, the automotive software is distributed on multiple computing units (or ECU). In this case, the greater failure risks of the distributed automotive software necessitate maximizing system reliability such by implementing fault tolerance, e.g., using redundant software software functionality on multiple ECU, which requires additional computation, I/O and more power. In this regard, the software-to-hardware allocation process plays a crucial role, that is effective and efficient allocation should satisfy the safety-critical software system requirements such as timing and reliability, but also should mimiize the resource consumptions.

Thus, the third research goal is to
\begin{researchgoal}
 \rgl{Minimize power consumption of distributed safety-critical software while satifying timing and reliability requirements, in the allocation process of software to hardware.}
\end{researchgoal}

Software allocation is NP hard and is difficult to find a solution in the general case. However, for less complex problems, exact methods, e.g., using integer-linear programming, etc., works, however, for large and complex probelms, the exact methods are limited, in stead, heauristics is usually applied. In this thesis, we assume, fixed-preemptive scheduling policy and timing analysis based on response time, furthermore, we consider end-to-end timing analysis simultaneously, as a result, which the software allocation is trivial. 

In order to show the validity of our proposed solutions, a working prototype should be developed and should also evaluated on industrial uses cases. The validation should consider scalability and engineer-friendliness of methods and tools besides effectiveness. Thus, the last research goal is to:
\begin{researchgoal}
 \rgl{Provide automated and engineering-friendly support for the requirements specification, software allocation of embedded  and formal analysis of Simulink models.}
\end{researchgoal}

Seamless integration of our proposed methods and tools into the existing development process require close cooperations between the domain experts and the practitioners. The role of the domain experts should be to simplify usage of the tools, e.g., by rendering their interface to exisiting once, etc., and the practitioners should cooperate from providing to materials to the validation of the tools, which is not trivial considering the challenge of forma methods, and companies culture for being restrctive.
