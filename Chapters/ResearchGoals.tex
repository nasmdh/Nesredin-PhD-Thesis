\chapter{Problem Formulation}
The research is motivated by problems encountered in the development of automotive systems from Volvo Group Trucks Technology (VGTT) and Scania, specifically in the area of requirements specification and verification of automotive software. The problem is articulated under the VeriSpec project\footnote{VeriSpec project URL: https://bit.ly/2J9Vyyp} with particular focus on effectiveness, scalability and engineer-friendliness of methods and tools, and seamless integration of the latter into existing automotive development process. Moreover, it is inspired by the increasing complexity of automotive functionality implemented by the electrical/electronic system, and hence, the need for critical resource optimization such as power and energy usage.

At VGTT, similar to other domains, e.g., automotive systems are developed in stages using Based on preliminary study of automotive software development at VGTT through informal discussions, interviews, use cases analysis and first-hand experience on existing automotive systems development tools, we identify three important issues that can potentially impact the dependability of safety-critical automotive systems and the performance of the development process. The issues are  summarized as follows: i) proprietary and open-source IDEs are used to develop automotive systems such as SystemWeaver variant, EATOP variant. The IDEs use natural language for the specification of requirements, which are linked to the system models. Neverthless, the specifications are sometimes incomprehensible and inconsistent due to the inherent ambiguity of natural language as well as errors introduced due to human factors,  ii) Simulink is frequently used to model, simulate autmotive systems before code generation at VGTT. However, there is lack of early verifiation especially related to timing and functional verification at the model-level, that is before code-generation, iii) finally, we observe increasing complexity of the volvo truck eleccal/electronic system due to advanced functionaly intoduced to assist the driver and improve safety.

The main research goal of the thesis is to :
\begin{center}
	\ogl{provide assurance and extensibility of safety-critical software functionality, at the various levels of abstraction, via formal analysis and optimization techniques}
\end{center}

\section{Research Goals}\label{research_challenges}
The main research goal is refined into five research goals. The first two research goals ensure quality of requirements specifications, the third research goal assures quality of software design (modeled in Simulink), the fourth research goal assures optimal use of power while meeting timing and reliability requirements, and the last research goal address the need for the validation of our proposed methods and tools.

The first research goal deals with the problem of requirements specification using natural language, which is intuitive, expressive and flexible, hence easy to use. However, natural language is inherently ambiguous and therefore can sometimes lead to undesirable (or bad quality specifications), that is incomprehensible and ambiguous. The template-based specification methods, e.g., requirement boilerplates, specification pattern systems reduce ambiguity and also improve readability, however, they are normally too restrictive, and selecting appropriate templates is usually cumbersome. The existing controlled natural languages, e.g., Attempto, PING, etc., closely renders the good qualities of natural languages, however are not domain-specific to embedded systems, hence are not effective, e.g., Attempto, PING, etc. Therefore, the first research goal is to:
\setcounter{rgcounter}{1}
\begin{researchgoal}
\rgl{improve the comprehensibility and analyzability of natural language requirements using domain-specific knowledge and formal semantics.}
\end{researchgoal}

One of the mechanisms to improve natural language specifications is by constraining the language, including its syntax, semantics and the lexicon \cite{Kuhn2014ALanguages}. The design of a constrained natural language for the specification of requirements is not trivial mainly because: i) by constraining the language, its expressiveness and intuitiveness can be impaired~\cite{ieereqspecstandard}\cite{Myachykov2013SyntacticRussian}, therefore, appropriate trade-offs should be made during the design in order to have a robust and effective specification language; ii) domain knowledge/expertise is highly needed; iii) in the absence of concrete system model, the language should support effective specification.

Besides assuring the quality of requirements specifications individually, the latter should be analyzed in ensemble in order to detect errors that span multiple specifications, e.g., logical inconsistencies within specifications. In essence, the specifications should possess rich semantics that relate individual specifications to each other. In this regard, the second research goal is to:
\begin{researchgoal}
 \rgl{facilitate formal analysis of the requirements specifications through transformation to Boolean and description logics}
\end{researchgoal}

Natural language requirements specifications are constructed from syntactic units, such as words, phrases, clauses, statements, etc. Consequently, rigorous analysis of specifications involve parsing and interpreting the syntactic units, which is a complex problem in computational linguistics, mainly due to the multitude of interpretation (or semantics) paradigms \cite{Clark2010TheProcessing}. Model-theoretic semantics is a widely-applied paradigm in computational semantics. It computes truth values of sentences by inductively applying interpretation functions on the syntactic units (or structures) of the sentences in relation to mathematical systems, e.g., propositional, first-order systems. The depth of the interpretation greatly affects the use and applicability of the interpretation approach. For instance, with the assumption that a proposition equates to a clause in a sentence, propositional logic elegantly represents sentences and scales well to find truth values. However, it provides a shallow interpretation of the sentences as it abstracts the various units of clauses by propositional variables. In contrast, first-order logic representations provide richer semantics and thus enable rigorous analysis, but yet lesser tractable. Therefore, the type of analysis and level of rigor needed drives the appropriate semantics definition.

The existing Simulink analysis techniques, e.g., by type checking, simulation, and formal verification via Simulink Design Verifier (SDV\footnote{https://se.mathworks.com/products/sldesignverifier.html}) are not sufficient to address the full correctness of safety-critical real-time Simulink models. SDV lacks support for checking temporal correctness as specified in timed properties, e.g., in TCTL, and also lacks support for verifying continuous models and suffers from scalability due to its reliance on the exact model-checking \cite{Leitner2008SimulinkStudy}. In contrast to exact model checking, statistical model-checking collects sufficient traces of system simulations, and consequently applies statistical methods to verify properties. It scales better, and the accuracy of the analysis can be improved by taking many traces of simulations. Thus, the fourth research goal of the thesis is to:
\begin{researchgoal}
\rgl{Enable scalable formal analysis of multirate and hybrid Simulink models using statistical model-checking}
\end{researchgoal}

Simulink consists of connected and hierarchical Simulink blocks, which encode mathematical functions~\cite{JamesB.Dabney2003MasteringSimulink}. For industrial systems, the number of blocks in a Simulink model can be in the order of thousands, and the blocks can be triggered with different sampling frequencies for discrete blocks and without any sampling frequency 
for continuous blocks. Therefore, typical industrial Simulink models are usually complex and comprise mixed signals, multiple rates, discrete and continues Simulink blocks, making formal analysis challenging.

Afterwards, the specifications are used to create the high-level system design (or architecture) that realizes the required functionality. The system architecture consists of software and hardware logical components, as well as mappings from the software components to the hardware components (or computation nodes). The latter activity of the system design, also known as \textit{software allocation}, should be handled effectively in order to preserve the functionality of the software architecture, including functional correctness and extra-functional attributes such as timing and reliability. Furthermore, the allocation should be efficient in order to conserve critical system resources while satisfying requirements as well as design and hardware constraints.

Effective and efficient allocation of multirate software applications is highly needed, as it is crucial to safety-critical systems such as automotive systems, aircrafts, etc. Many multirate applications are deployed on several computation nodes, which are possibly heterogeneous, over shared communication networks. Therefore, the third research goal is to
\begin{researchgoal}
 \rgl{Minimize power consumption of distributed safety-critical applications during allocation on a  network of heterogeneous computing units}
\end{researchgoal}

Allocation of multirate software applications on a network of computation nodes is challenging mainly due to the complex timing analysis that needs to be considered during the allocation process \cite{Mahmud5222, mubeen2013support}. The timed paths of multirate applications increasing the search space, and therefore finding an efficient allocation becomes exponential. Furthermore, the heterogeneity of the computation nodes forces the search method to consider every computation node for better result, thus increasing the optimization time, as opposed to considering homogeneous nodes. Since the software allocation could be intractable especially for large systems, methods based on heuristics and approximation should be provided instead of exact ones.

In order to show the validity of our proposed solutions, a working prototype should be developed and should be applied on industrial uses cases. Our proposed specification and analysis methods and tools should be engineering friendly in order to facilitate their adoption in industry, for instance they should embody intuitiveness and seamless integration into industrial practices. Therefore, the last research goal is as follows:
\begin{researchgoal}
 \rgl{Provide automated and engineering-friendly support for the requirements specification, software allocation of embedded  and formal analysis of Simulink models.}
\end{researchgoal}

Seamless integration of new development methods and tools require appropriate interfaces to plug into existing development methods and tools in order to facilitate adoption of formal techniques, by lowering the required effort and cost. In particular, formal methods should be accompanied by engineering-friendly interfaces, as the syntax of the formal notations is not familiar to most engineers, likewise, understanding the underlying semantics requires a substantial shift from traditional software development paradigms. In order to tackle these challenges, very close cooperation with engineers and know-how of domain-specific industrial tools and practices are paramount.

