%%%%%%%%%%%%%%%%%%%%%%%%%%%%%%%%%%%%%%%%%%%
% Template for Doctoral Theses at Uppsala %
% University. The template is based on    %
% the layout and typography used for      %
% dissertations in the Acta Universitatis %
% Upsaliensis series                      %
% Ver 4.0.1 - 2008-11-03                  %
%                                         %
% Publishing & Graphic Services           %
% espik@ub.uu.se                          %
%                                         %
%%%%%%%%%%%%%%%%%%%%%%%%%%%%%%%%%%%%%%%%%%%


\documentclass[11pt,a4paper,twoside, openright]{book} % Default font 11pt, A4, all pages are printed the same, new chapter always begins on a right-hand page.

% Package import
% Language, diacritics and hyphenation
\usepackage[swedish,english]{babel} % Use English and Swedish languages. English default. English hyphenation.
\usepackage[latin1]{inputenc} % Unix users should include this package instead of ansinew or applemac for correct handling of diacritics.
%\usepackage[applemac]{inputenc} % Macintosh users should include this package instead of ansinew or latin1 for correct handling of diacritics.
%\usepackage[ansinew]{inputenc} % % Windows users should include this package instead of applemac or latin1 for correct handling of diacritics.
\usepackage[T1]{fontenc}% Enables the use of 8-bit characters and �, �, � can be typed as �, �, � instead of the corresponding commands.

% Page elements
\usepackage{mathptmx} % Default font for dissertations is Times.
%\usepackage{fourier} % If mathematics don't display well using Times, then use Fourier.
\usepackage{helvet} % Default sans-serif font.
\usepackage{courier}% Default typewriter (monotype) font.
\usepackage{Thesis} % This package is specific for theses written at Uppsala University. Modifications to the classfile and the document can be found here.
\ifpdf
   \usepackage[pdftex]{graphicx}
  \else
    \usepackage[dvips]{graphicx}
\fi % Used for figures
\usepackage{booktabs} % Produces good three line tables which is the standard tabel format for disserations.
\usepackage[colorlinks=true, urlcolor=black, pagecolor=black, linkcolor=black, citecolor=black, filecolor=black, menucolor=black, pdfpagelabels,bookmarksnumbered=true]{hyperref} % Use this package to produce links in the electronic version of the document. All hyperlinks are black. Page view is set to Fit Width and page layout is set to display the document spread.  Make page anchors using the formatted form of the page number.  Set PDF page labels; i.e., write the value of \thepage to the PDF file so that Acrobat Reader can display the page number as (say) 'ii (4 of 40)' rather than simply '4 of 40'.

\usepackage[usenames,dvipsnames,svgnames,table]{xcolor}
% Nas styles
\usepackage{nasutilsp}

% Make the first letter of a paragraph bigger, spans over multiple lines
\usepackage{lettrine}

% enumerate inline
\usepackage[inline]{enumitem}

% Bibliographic information
% Filling in this bibliographic information facilitates the
% processing of this document.
% Insert linebreaks if necessary
% The abstract page is not generated by default. The reason for this is that it
% will be generated by the unit for Publishing and Graphic sevices from the metadata 
% template ("Spikningsmallen"). If you want to include the abstract page generated from this template,
% please edit the \frontmatterMonograph or \frontmatterCS command at the en of this file
%. Because of this, some of the bibliografic data on this page will not be used.

% Abstract and titelpage
\newcommand{\authorSurname}{Mahmud} % Your surname
\newcommand{\authorFirstName}{Nesredin} % Your given name
\newcommand{\authorFirstInitial}{N} % Initial of given name
\newcommand{\authorEmail}{nesredin.mahmud@mdh.se} % Your e-mail address
\newcommand{\dissertationTitle}{Design of Assured and Efficient Safety-critical Systems using Formal Methods and Optimization Techniques} % The title of the dissertation
\newcommand{\dissertationSubtitle}{Subtitle of the dissertation}% The subtitle of the dissertation (if there is any).
\newcommand{\yearOfPublication}{2019} % Year of publication
\newcommand{\placeOfDisputation}{M\"{a}lardalen University} % Place of disputation
\newcommand{\dateOfDisputation}{Thursday, June 13, 2019} % Date of disputation (Day, Month, Year)
\newcommand{\timeOfDisputation}{13:15} % Time of disputation (Swedish time format)
\newcommand{\disputationLanguage}{English} % Language the examination will be conducted in.
\newcommand{\numberOfPages}{51} % The page number of the last page
\newcommand{\placeOfPublication}{Uppsala} % The place of publication
\newcommand{\ISBN}{978-91-554-5436-4} % The ISBN number of the dissertation.
\newcommand{\series}{Uppsala Dissertations from the Faculty of Science and Technology} % The title of the series
\newcommand{\serialNumber}{1214} % The number in the series.
\newcommand{\keywords}{ embedded system, real-time system, safety-critical system, requirements specification, formal method, Simulink, integer-linear programming, metaheuristics} % Key words separated by comma
\newcommand{\department}{Department of Electronic Publishing} % Name of your department
\newcommand{\departmentaddress}{Villavagen 14, SE-752 36 Uppsala,
Sweden} % Address of your department
\newcommand{\ISSN}{1651-6214} % (ISSN for Digital comprehensive summaries of Uppsala dissertations from the faculty of science and technology)
\newcommand{\urn}{urn:nbn:se:uu:diva-3344} % URN number

% List of papers
\newcommand{\listofpapers}
{\cleardoublepage
\pdfbookmark[0]{List of Papers}{LOP}
\chapter*{List of Papers Included in the Thesis}
\noindent This thesis is based on the following papers:
\vspace{1\baselineskip}

        % It is possible to refer to the published papers using labels in the text.
        % Suggested order
        % Author 1 surname, Author 2 first name initial., Author 1 surname, Author 2 first name
        % initial. etc. (Year of publication) Paper main title.
        % Paper subtitle. Name of journal in italics, volume(number):page rage
        % Example
        
    \begin{description}	
      \item[Paper A]  ReSA: An ontology-based requirement specification language tailored to automotive systems. Nesredin~Mahmud, Cristina~Seceleanu and Oscar~Ljungkrantz.\textit{In the 10th IEEE International Symposium on Industrial Embedded Systems (SIES)(pp. 1-10). IEEE, 2015}.\label{lbl_resa}
    
      \item[Paper B]  ReSA tool: Structured requirements specification and SAT-based consistency-checking. Nesredin~Mahmud, Cristina~Seceleanu and Oscar~Ljungkrantz. \textit{In the 2016 Federated Conference on Computer Science and Information Systems (FedCSIS)(pp. 1737-1746). IEEE, 2016.}\label{lbl_resatool}
      
      \item[Paper C]  Specification and semantic analysis of embedded systems requirements: From description logic to temporal logic. Nesredin~Mahmud, Cristina~Seceleanu and Oscar~Ljungkrantz. \textit{In the International Conference on Software Engineering and Formal Methods (SEFM)(pp. 332-348). Springer, Cham, 2017.}\label{lbl_resadl}
            
     \item[Paper D] SIMPPAAL - A Framework For Statistical Model Checking of Industrial Simulink Models. Predrag~Filipovikj, Nesredin~Mahmud, Raluca~Marinescu, Cristina~Seceleanu, Oscar~Ljungkrantz and Henrik~L\"{o}nn. \textit{Submitted to ACM Transactions on Software Engineering and Methodology (TOSEM). Initially submitted in November 2018, revisions required.} 
     \\[3pt]{\footnotesize This article is an extended version of the following conference paper:
     Simulink to UPPAAL Statistical Model Checker: Analyzing Automotive Industrial Systems.
Predrag Filipovikj, Nesredin Mahmud, Raluca Marinescu, Cristina Seceleanu, Oscar Ljungkrantz, Henrik L{\"o}nn. In Proceedings of the 21st International
Symposium on Formal Methods (FM2016), pages 748-756. Limassol, Cyprus. Springer, LNCS, November 2016. }\label{lbl_simulink_ilp}

     \item[Paper E] Power-aware Allocation of Fault-tolerant Multirate AUTOSAR Applications.
     Nesredin~Mahmud, Guillermo~Rodriguez-Navas, Hamid~Faragardi, Saad~Mubeen and Cristina~Seceleanu. \textit{In the 25th Asia-Pacific Software Engineering Conference (APSEC'18). IEEE.} \label{lbl_resa} \label{lbl_softwareallocation_ilp}
        \vspace{1cm}
	\item[Paper F] Optimized Allocation of Fault-tolerant Embedded Software with End-to-end Timing Constraints.	Nesredin~Mahmud, Guillermo~Rodriguez-Navas, Hamid~Faragardi, Saad~Mubeen and Cristina~Seceleanu. \textit{M\"{a}lardalen Real-time Research Center Technical Report (MRTC), ISRN MDH-MRTC-325/2019-1-SE, 2019.}\\
	{\footnotesize Note: The research is submitted as a journal paper to Elsevier Journal of Systems Architecture (JSA). Under Review}. \label{lbl_softwareallocation_pso}
    \end{description}
\section*{\Large  List of Papers Not Included in the Thesis}
\vspace{1\baselineskip}
\begin{itemize}	
	\item Evaluating industrial applicability of virtualization on a distributed multicore platform. Nesredin Mahmud, Kristian Sandstr{\"o}m, and Aneta Vulgarakis.  \textit{In Proceedings of the 2014 IEEE Emerging Technology and Factory Automation (ETFA), pp. 1-8. IEEE, 2014.}
	\item The multi-resource server for predictable execution on multi-core platforms. Rafia Inam, Nesredin Mahmud, Moris Behnam, Thomas Nolte, and Mikael Sj{\"o}din. \textit{In 2014 IEEE 19th Real-Time and Embedded Technology and Applications Symposium (RTAS), pp. 1-12. IEEE, 2014.}
	\item Simulink to UPPAAL statistical model checker: Analyzing automotive industrial systems. Predrag Filipovikj, Nesredin Mahmud, Raluca Marinescu, Cristina Seceleanu, Oscar Ljungkrantz, and Henrik L{\"o}nn. \textit{In International Symposium on Formal Methods, pp. 748-756. Springer, Cham, 2016.}
\end{itemize}

\vspace{1\baselineskip} \noindent {\small Reprints were made with permission from
the publishers.}}


% Dedication. Dedication is optional. You can enter whatever feels right and use the fonts and images you like as long as they are embedded in the document. The suggested format does not have to be used.
\newcommand{\dedication}%
{\cleardoublepage
\thispagestyle{empty}
\vspace*{\stretch{3}}
\begin{flushright}
		
{\fontfamily{pzc}\Large\selectfont\emph{To My Parents}}

\end{flushright}
\vspace*{\stretch{1}}} % 
% Ack
\newcommand{\ack}%
{\cleardoublepage
\pdfbookmark[0]{List of Papers}{LOP}
%\thispagestyle{empty}
%\vspace*{\stretch{3}}

\chapter*{Acknowledgements}
%\vspace{1\baselineskip}

I would like to express my deepest gratitude to my main supervisor Associate Professor Cristina Seceleanu. Your passion for research is inspiring, and your guidance, encouragements, determination, and patience have been the driving force of my PhD journey. It has been a pleasure to be around with you and get the best mentor one could ever get. I would like to thank my co-supervisor Oscar Ljungkrantz for the lively reception at Volvo Group Trucks Technology (VGTT), Gothenburg, and the wonderful researchers he introduced to me. It is my pleasure also to thank my co-supervisor Guillermo Rodriguez-Navas, for the great team work, valuable feedback, and encouragements. I am grateful to have you all as my supervisors. Your effort has made this work worthwhile.

Moreover, I would like to thank my examining committee Professor Peter Csaba {\"O}lveczky, University of Oslo, Professor Wolfgang Ahrendt, Chalmers University of Technology, Associate Professor Raffaela Mirandola, Politecnico di Milano, for accepting our call and reviewing my thesis, and Associate Professor Antonio Cicchetti for reviewing the thesis proposal. Special gratitude goes to the faculty examiner Professor Joost-Pieter Katoen, RWTH Aachen University, for accepting the opponentship. It is an honour to have you all at my defense. Also, I would like to extend my appreciation to Damir Isovic, Thomas Nolte, Hans Hansson, Moris Behnam and the IDT administration staff, who facilitated my PhD defense and influenced my academic career.

I wish to acknowledge with much appreciation the role of Henrik L{\"o}nn, Oscar Ljungkrantz, as well as Saad Mubeen and Hamid Reza Faragardi, who supported my research technically at different levels, including co-authoring joint papers. Your willingness to share and support is highly admired. For the list is too long, I briefly express my respect and love to my colleagues at MDH, and outside such as Tewodros, Abu Hussaine, Bj{\"o}rn. Your friendly support, jokes, and cultural exchanges will always be memorable.

I am thankful to Allah for granting me the well-being and strength to do my studies. I thank my parents for giving me unconditional love, and my brothers, sisters and lifelong friends Sal, Jem, Nas, Kb, Ja, Melow, Zigi, Asf, Bini, Mu and others who supported me throughout my life - thank you Big Brother~Abdulwhab for being the example. In times of emotional stress, the support of my dear wife, Elham Salih, has been unconditional and unwavering. You are strong, confident and intelligent person and love of my life, and I thank the OOG for that! Our little boy, Mohammed, you should know that we love you so much. So learn the world with open mind and make us proud!

I would like to thank VINNOVA for sponsoring this research via the VeriSpec project, in which the financial support has made this work possible.

\vspace*{\stretch{1}}} % 

% Frontmatter commands
% Halftitle page for monographs
\newcommand{\halftitlepage}{%
\thispagestyle{empty}
\begin{center}
	ACTA UNIVERSITATIS UPSALIENSIS

	\emph{\series{}}
	
	\serialNumber
\end{center}
\cleardoublepage}

% Title page for monographs
\newcommand{\thetitlepage}%
{\thispagestyle{empty}
\vspace*{40mm}
\begin{center}
\Large \authorFirstName{} \authorSurname{}

\vspace{6mm}

\Huge{ \dissertationTitle{}}

\vspace{6mm}

\fontsize{14}{16}\fontshape{it}\selectfont\dissertationSubtitle{}			
\end{center}
\begin{figure}[b]
\begin{center}
\ifpdf
\includegraphics{UU_logo_pc_sv_42}
\else
\includegraphics{UU_logo_pc_sv_42.eps}
\fi
\end{center}
\end{figure}}

% Abstract. This is optional. 
\newcommand{\thesisabstract}{%
\begin{abstract}{%
	\newpage
	\thispagestyle{empty}
    \fontsize{9}{11}\selectfont
    \noindent
    \begin{flushleft}{\nohyphens
    Dissertation at M\"{a}lardalen University to be publicly examined in \placeOfDisputation{}, \dateOfDisputation{} at \timeOfDisputation{}
    for the Degree of Doctor of Philosophy. The examination will be conducted in \disputationLanguage{}
}\end{flushleft}

    \noindent\textbf{Abstract}
%    \vspace{-9pt}
%    \begin{flushleft}{\nohyphens\noindent\authorSurname{}, \authorFirstInitial{}. \yearOfPublication{}.
%    \dissertationTitle{}. \dissertationSubtitle{}. Acta Universitatis
%    Upsaliensis. \emph{\series{}} \serialNumber{}. \numberOfPages{}~pp.
%    \placeOfPublication{}. ISBN~\ISBN{}}\end{flushleft}

% Type your abstracttext here. Remove the latin text. Try to not type more than 300 words.
\noindent 

    \vspace{11pt}

    \noindent
    \emph{Keywords: }\keywords

    \vspace{11pt}

    \noindent
    \emph{\authorFirstName{} \authorSurname{}, \department{}, M\"{a}lardalen University,
    \departmentaddress{}}

    \vspace{11pt}

    \noindent \copyright{} \authorFirstName{} \authorSurname{}
    \yearOfPublication{}

    \vspace{11pt}
    
    \noindent ISSN \ISSN{}

    \noindent ISBN \ISBN{}

    \noindent \urn{} (http://urn.kb.se/resolve?urn=\urn)
}
\end{abstract}}

% Title page dummy for Compehensive summaries
\newcommand{\titlepagedummy}%
{\newpage
\newpage\thispagestyle{empty}
{\noindent\LARGE  Design of Assured and Efficient Safety-critical Embedded Systems}\vspace{84pt}

\noindent This is the title page dummy. This page should be substituted for a real title page. The real title page will be provided by Publishing and Graphic services after having submitted your posting details in DiVA. The DiVA registration form can be found at \\ \href{https://uu.diva-portal.org/dream/}{https://uu.diva-portal.org/dream/}.

\vspace{1em}

\noindent More information about the publishing routines can be found at \\
\href{http://beta.ub.uu.se/pgs}{http://beta.ub.uu.se/pgs}}

%Abstractdummy
\newcommand{\abstractdummy}%
{\newpage\thispagestyle{empty}
{\noindent\LARGE Abstract page}\vspace{84pt}

This is the abstract dummy. This page should be substituted for real abstract/imprint page. The real abstract page will be provided by Publishing and Graphic services after having submitted your posting details in DiVA. The DiVA registration form can be found at \\ \href{https://uu.diva-portal.org/dream/}{https://uu.diva-portal.org/dream/}.
\vspace{1em}

%\noindent More information about the publishing routines can be found at \\
%\href{http://beta.ub.uu.se/pgs}{http://beta.ub.uu.se/pgs}
}

% Front matter commands
\newcommand{\frontmatterMonograph}%
{\halftitlepage\thetitlepage\abstractdummy\dedication}

\newcommand{\frontmatterCS}%
{\titlepagedummy\abstractdummy\dedication\ack\listofpapers} % This file should be edited by the author.

\begin{document}
%\frontmatter
    %\frontmatterMonograph % Monograph authors use this front matter 
    %\frontmatterCS % Authors of Comprehensive Summaries use this front matter 
    % Creates the front matter (title page(s), abstract, list of papers)
    % for either a Comprehensive Summary or a Monograph.
    
    \tableofcontents
    
    % Optional tables
    %\listoftables
    %\listoffigures

\mainmatter
% Safety-critical embedded systems should be analyzed rigorously in order to assure their functional correctness, moreover, they should be analyzed at the system design level, that is before implementation and testing for improved effectiveness and efficiency. In model-based development, the various activities of system design such the requirements specification, software and hardware design, and the mapping of software to hardware are conceptualized using models and abstraction levels, e.g., in SysML, AUTOSAR, EASTADL architectural languages. The requirements specifications should be unambiguous, comprehensible, consistent, etc., and the software should satisfy the specifications such as the timing constraints besides functional behavior. Moreover, as the safety-critical functionality is getting complex, the software is usually distributed on multiple computing processors (or units), in which case, the software functionality should be mapped to hardware efficiently to conserve critical system resources such as power consumption.

% In distributed computing, the risk of failure normal increases, therefore, maximizing the reliability of the safety-critical software is crucial to meet the reliability goals that are set to mitigate the risk. Fault-tolerance is the most common approach to meet reliability demands, which is in odds against the constrained nature of embedded hardware systems such as the power and energy supply of battery-driven systems. So, it is crucial to optimize the total power consumption of the distributed safety-critical software in order to accommodate current and future growth of software functionality.

% In this thesis, we propose formal techniques and optimization methods respectively to assure improved quality of requirements specifications and software design, and the efficient use the total power consumption in the distributed system. The formal language we propose is ReSA, which is a domain-specific, constrained natural language, and has formal semantics in Boolean and description logic. We also propose statistical model-checking of software that is modeled in Simulink, a de facto development environment of safety-critical embedded systems in industry. Finally, we propose an integrated mapping of software to hardware that optimizes the total power consumption and preserves the timing and reliability constraints of the safety-critical software, which is modeled in AUTOSAR, via the integer-linear programming and meta-heuristic methods.

% Our proposed domain-specific language and the statistical model checking of Simulink models is evaluated on automotive use cases from Volvo Group Trucks Technology (VGTT), which consist of the adjustable-speed limiter (ASL) and the brake-by-wire (BBW) systems. The proposed power optimization methods are evaluated on automotive benchmark from Bosch.
    % This includes the "Instruction", "Problem and Solutions" and "Example" files. After reading it, remove it from Thesis.tex. 
     %\input{Chapters/Example.tex}
     \chapter{Introduction}\label{chapter_introduction}
\lettrine{R}{eal-time} systems are usually characterized by timely computations, which are bounded by \textit{deadlines} besides correct results of the computations~\cite{Buttazzo2003}. They are applied in many \textit{safety-critical embedded} systems, which are specialized computer systems designed for safety-critical applications~\cite{WangJiacun2017RES}, e.g., brake-by-wire, flight control. For instance, the brake-by-wire system~\cite{bibid} applies torque proportional to the pressing of the brake pedal in order to slow down (or halt) the vehicle. Besides computing the correct torque, the system must act timely, i.e., not too soon or not too late so that a traffic accident is avoided. Therefore, safety-critical real-time systems should be analyzed for functional as well as timing correctness, and in fact, they should be analyzed rigorously according to functiona safety standards, e.g., ISO 26262 ``Road vehicles-Functional safety''~\cite{iso201126262} recommends the use of \textit{formal methods}, which are mathematical techniques and tools that enable unambiguous specification, modeling and rigorous analysis~\cite{o2017concise}, to develop safety-critical automotive systems. In this thesis, we apply various formal methods early, i.e., at the requirements specification and software design levels, to improve the quality assurance of developing safety-critical embedded systems.

Over the last decades, a lot of functionality that have been provided by mechanical and electical systems have moved to software functionality, e.g., the x-by-wire technology~\cite{bibid}, and several software functionality, which are driven by innovation and safety standards, are integrated into safety-critical systems to provide advanced, safe and reliable services, e.g., the advanced driver-assistance systems, vision system for autonomous and self-driving vehicles. In response, several computing architectures have moved from a federated architecture (i.e., dedicated hardware systems executing safety-critial software systems or applications) into consolidated (i.e., multiple safety-critial software applications co-hosted on the same hardware) and distributed architectures. In distributed computing~\cite{Kopetz2003Real-timeApplications}, the safety-critical software is mapped on multiple hardware systems to capitalize on the computational power provided by the distributed architecture, e.g., the brake-by-wire software executing on multiple electronic control units (ECU) that are connected through a network bus, e.g., CAN~\cite{}. The distributed software, unlike in the case of federated computing, it is exposed to a greater degree of permanent and transient faults, therefore, reliability of the software should be maximized to improve the overall dependability~\cite{} of the  system. However, such measures requires additional critical system resources besides computational resources, e.g., power and energy, which are constrained in battery-driven embedded systems. Thus, the distributed safety-critical software should be deployed efficiently on the distributed architecture while meeting the timing and reliability requirements of the software.


In this thesis, we propose a design approach which considers rigoruous analysis and efficiency of safety-critical embedded systems by applying formal methods and optimization techniques at the early stages of software development. Thus, the safety-critical  specifications should be unambiguous, comprehensible, etc~\cite{ieereqspecstandard}, and the software design should conform to the specifications. Since natural languag is inherently ambigous~\cite{ieereqspecstandard}, many functional safety standards, e.g., ISO 26262, recommends semi-formal and formal specification methods to specify safety-critical embedded systems requirements, e.g., template-based languages, controlled natural language, temporal logic. The template-based specification methods, e.g., requirements boilerplates~\cite{Hull2011RequirementsEngineering}, property-specification systems~\cite{Dwyer1999PatternsVerificationb}, etc., lack meta-model to effectively create templates, and is usually cumbersome to select the templates. The controlled natural languages, e.g., Attempto~\cite{attempto96}\cite{Fuchs2008AttemptoRepresentation}, etc., renders the syntax and semantics of the natural language and have formal semantics, however lacks support for embedded systems, hence are less effective. Rather, we propose a constrained natural language which is domain-specific and uses the notion of boilerplates to facilitate reuse. The specifications have semantics in Boolean logic and description logic to enable rigorous analysis via Boolean satisfiability~\cite{Malik2009BooleanSuccess} and ontology~\cite{f25ea3c6f4b743cd90c150926bbcf3db}, respectively.

The requirements specifications are used in subsequent system development including software design to verify the latter for correct functionality. The software design is usually modeled, simulated and analyzed before implementation. In this regard, Simulink is one of the most widely used development environment for multi-domain, multi-rate, discrete and continuous safety-critical systems in industry~\cite{JamesB.Dabney2003MasteringSimulink}. For this main reason, there is increasing interest in formal analysis of Simulink models~\cite{Manamcheri2011AModels}. Simulink Design Verifier\footnote{Simulink Design Verifier - https://se.mathworks.com/products/sldesignverifier.html}, which is based on the Prover\footnote{Prover - https://www.prover.com/software-solutions-rail-control/formal-verification/} model checker, is the de~facto tool in the Simulink environment to formally verify Simulink design models. However, it has limited functionality, e.g., it supports only discrete models, has issues with scalability due to state-space explosion, and lacks verification of timed properties~\cite{Leitner2008SimulinkStudy}.  In contrast, we propose a scalable, timed analysis via a statistical model checking~\cite{Legay2010StatisticalOverview}, which uses traces of executions and statistical analysis techniques, e.g., monte-carlo simulation.

The software design should be mapped to hardware effectively, that is satisfying the timing and reliability requirements of the distributed safety-critical software, but also efficiently to minimize the power consumption of the distributed system to facilitate accommodating complex software functionality. We assume the software is scheduled using a fixed-priority preemptive policy, which is quite common in industry, and posses end-to-end timing requirements, e.g., the time duration between the brake-pedal press and the slow-down (or halt) of the vehicle. Furthermore,  we consider fault tolerance as a means to maximize reliability of the distributed safety-critical software by mapping redundant software functionality on different computing units.  We propose  \textit{exact} and \textit{heuristic} optimization methods, which deliver optimal and near-optimal solutions, respectively, to efficiently map the distributed safety-critical software to a network of computing nodes. Specifically, we propose a formulation of integer-linear programming (ILP)~\cite{Mahmud5222}, which is solved using branch and bound. Furthermore, we propose a hybrid-particle optimization~\cite{Mirjalili2019ParticleOptimisation}, which is a meta-heuristic algorithm, to solve the shortcomings of the exact method for large-scale problems~\cite{Mahmud2019Power-awareOptimization} with trade-off over optimality.

The main contributions of the thesis are: 
\begin{enumerate*}[label=(\roman*)]
\item formal analysis of natural language requirements - we propose a fairly expressive, flexible yet structured and domain-specific constrained natural language called \textit{ReSA}~\cite{resatool}\cite{Mahmud2015ReSA:Systems}. The language has semantics in Boolean and description logic to support shallow and rigorous analysis, respectively. The Boolean specifications are checked for consistency using the Boolean satisfiability via the Z3 SMT solver.  The ReSA tool is integrated seamlessly into EATOP to complement the requirements modeling of EAST-ADL. The language and its tool support are validated on the adjustable speed limiter (ASL) use case, which is a vehicle speed control system provided by Volvo Group Trucks Technology (VGTT);
%Whereas the description logic is used to encode the specifications as ontology, subsequently is checked for consistency at the lexical level via a semantic reasoner (i.e., HermiT as used in Prot{\'e}g{\'e} tool), which is a software program that produces logical consequencies from asserted axioms and facts.

\item scalable and formal analysis of Simulink models - we propose a pattern-based, execution-order preserving automatic transformation of a Simulink model into a network stochastic timed automata that can be formally analyzed using the UPPAAL statistical model checker (SMC)~\cite{Filipovikj2016SimulinkSystems}. The statistical model checker analyzes a state-transition system by conducting statistical analysis on the collected traces of the system executions~ \cite{Bulychev2012UPPAAL-SMC:Automata}, effectively mitigating the state-space explosion of (exact) model checking \cite{Legay2010StatisticalOverview}. Our proposed technique is validated on the brake-by-wire (BBW) use case, which is an industrial prototype provide for academic uses from VGTT;

\item efficient allocation of distributed safety-critical software applications - we propose an integer-linear programming (ILP) model and hybrid particle swarm optimization algorithms to allocate AUTOSAR software applications on a network of heterogenuous computing nodes with respect to processor speed, failure rate and power consumption specifications. The  ILP problem is implemented using ILOG CPLEX, which is a toolset for modeling and solving optimization problems. The proposed allocation methods are validated on an automotive benchmark developed according to AUTOSAR.
\end{enumerate*}


%Our solutions are evaluated on industrial automotive use cases and on a realistic benchmark. The formal analysis of the natural language requirements specifications in ReSA and the formal analysis of Simulink models are evaluated on the adjustable speed-limiter (ASL) and brake-by-wire (BBW) systems provided by Volvo Group Trucks Technology (VGTT). ASL is a speed-limitation automotive function which controls the vehicle speed of Volvo trucks from speeding up, and is useful in roads where speed-limitation signs are in place. The ASL use case consists of around 300 requirements, which are specified in natural language, architectural models in EAST-ADL and Simulink models. The integrated software allocation is evaluated on the engine management system benchmark provided by Bosch~\cite{} provided for AUTOSAR applications. The benchmark consists of statistics of the schedulable objects, such as mean values, shares of timing specifications and activation mechanisms of the schedulable objects in the system.

\section{Thesis Outline Overview}
The thesis is divided into two parts. The first part is a summary of our research. It is organized as follows: in Chapter 2, we give the background on the logic-based reasoning using Boolean and description logic, Simulink, UPPAL statistical model checking, and the optimization techniques based on ILP and PSO. In Chapter 3, we explain the research problem and outline the research goals. The thesis contributions are discussed in Chapter 4, which states the peer-review papers mappings to the contributions. In Section 5, we provide the related work on requirements specification, formal analysis of Simulink models, and software allocation. In Chapter 6, we describe the research method applied to conduct the research. Finally, in Chapter 7, we conclude the thesis, and outline the limitation and discuss possible directions for future work.


    \chapter{Background}
In this section, we give background on the different concepts and technologies used in this thesis, which include logic-based reasoning Simulink, stochastic timed automata,  statistical model checking, integer-linear programming and population-based metaheuristics.
\section{Logic-based Reasoning}
In this thesis, we express requirements specifications in  Boolean logic and apply \textit{Boolean satisfiability} (SAT) to detect inconsistencies within the specifications. Furthermore, we use \textit{ontology} to represent the specifications, and conduct rigorous analysis via a linguistic approach.
\subsection*{Boolean Satisfiability}
The study of a Boolean formula is generally concerned with the set of truth assignments (assignments of 0 or 1 to each of the variables) that make the formula true. Finding such assignments by answering the simpler question: ``does there exist a truth assignment that satisfies the Boolean formula?'' is called the Boolean satisfiability problem (SAT), which is proven to be NP complete~\cite{devlin2008satisfiability}\cite{Biere2009HandbookSatisfiability}. However, efficient heuristics and approximation algorithms do exist to solve such problems.

SAT problems are usually checked using decision procedures known as SAT solvers. Z3 integrates several background theories (or Satisfiability Module Theories, SMT) and decision procedures based on SAT solvers~\cite{DeMoura2008Z3:Solver}. It is a well known SMT solver and a theorem prover developed by Microsoft Research which has been used in the formal verification and reasoning of software and hardware systems, mainly due to its strong support for integration to verification tools via its well known APIs written, e.g., in Java, Python, C.

Z3's input language is an extension of the SMT-LIB2 standard, which are commands used to assert logical statements and instruct the solver to do some action, e.g., to check for satisfiability, the command \textit{(check-sat) }is invoked. Z3 returns \textit{sat} if the formulas are satisfiable. If the formulas are not satisfiable, Z3 returns \textit{unsat}, or if it cannot decide the result, it returns \textit{unknown}. In case of unsat, if the unsat-core option is enabled, Z3 can return the subset of unsatisfiable assertions by invoking the \textit{(get-unsat-core)} command.

\subsection*{Ontology and Description Logic}
Ontology is a knowledge representation technique frequently used in the artificial intelligence and other domains to facilitate automated decision making. It is defined as ``a formal specification of a shared conceptualization'' according to Borst \cite{Borst1997ConstructionOntologies}. It employs different modeling entities, e.g., concepts, instances, axioms, which are compared to classes, objects, inheritance in Object-Oriented Programming. The ontology is consistent if every axioms and relations (or assertions) hold~\cite{Mankovskii2009OWL:Language}.

Description logic (DL)~\cite{Baader2010TheApplications} is a knowledge presentation language, and is the most widely language used to construct ontologies. It is a fragment of first-order logic with many of the core reasoning problems decidable. For this reason, it has several application, e.g., in web semantics (e.g., OWL)~\cite{conf/owled/ShearerMH08}, artificial intelligence~\cite{10.1007/978-94-017-9297-4_7}, bioinformatics~\cite{Rector2006}, software engineering, natural language processing. Thus, it is usually backed by solid reasoning services that terminate and deliver results in reasonable time. 

\section{Simulink}
Simulink~\cite{JamesB.Dabney2003MasteringSimulink} is a graphical development environment for the modeling, simulation and analysis of embedded systems which is widely used in industry to model and inspect the dynamics of systems before implementation. It is robust as it supports multi-domain, continuous, discrete, hybrid systems, also discrete systems that execute with different sampling times (or multi-rate). Figure~\ref{fig_sm_multi-rate} shows a multi-rate subsystem of the brake-by-wire Simulink model that models the brake pedal (i.e., continuous behavior) and global brake functionality (i.e., discrete behavior), which executes every 10ms and 20ms.  
\begin{figure}[h]
	\centering
	\includegraphics[width=0.9\linewidth]{images/sm}
	\caption{Brake pedal and global brake controller of the brake-by-wire model.}
	\label{fig_sm_multi-rate}
\end{figure}

Simulink is frequently used in the development of safety-critical systems, e.g., automotive, avionics, furthermore, it is used to generate code automatically from the models. Therefore, it is crucial that such models are analyzed rigorously. The Simulink Design Verifier (SDV) provides a formal verification technique that is based on the exact model checking to exhaustively verify functional properties~\cite{MathWokrksSimulinkVerifier}, however, its functionality is limited as it lacks support for timed analysis, which is vital to ensure predictability of safety-critical embedded systems. 

\subsection*{Simulink Blocks}
A Simulink model is constructed from communicating function blocks (or Simulink blocks). The blocks implement simple to complex functionality and consist of Input and Output ports, which enable communication via connectors. The latter modeling elements support basic and user-defined datatypes, e.g., integer, floating-point; and simple and complex data structures, e.g., scalar, vector, matrix.  Simulink blocks are classified into \textit{virtual} (non-computational, e.g., Mux, Demux blocks) and \textit{non-virtual} (computational, e.g., Gain, Integrator) blocks based on their computability. The non-virtual blocks improve the visualization of the model, but unlike the non-virtual blocks, they are not executed, hence do not affect the execution semantics of the model.  The blocks can be composed into groups, e.g., using Subsystem, Model blocks, to enable hierarchical modeling, which is used to improve the visualization, and to enforce execution order on a group of blocks. The fundamental constructs of the composite blocks are \textit{atomic} blocks, e.g., Gain, Sum blocks.

The non-virtual (computational) atomic blocks can be categorized into \textit{continuous} and \textit{discrete} blocks based on the execution semantics of the blocks.  A discrete block executes periodically with sample time $t_s$, whereas a continuous block executes over infinitesimal sample times. Since the Simulink blocks libraries are not usually sufficient to model practical embedded system systems, Simulink supports mechanisms to extend functionality that engineers can exploit to develop complex systems. The mechanisms include S-function, Custom Block and Masking. S-function is a computer language of Simulink blocks which allows advanced implementations of block routines, written in MATLAB, C, C++, or Fortran.


\subsection*{Execution of Simulink Blocks}
During the initial phase of the simulation in the Simulink environment, the model is compiled, thus the order in which the blocks are executed is established in the \textit{sorted order} list. Figure~\ref{fig_sm_exec_order} shows the execution order via the labels in red color, annotated as $s:b$, where $s$ denotes the system/subsystem index and $b$ the block index\footnote{https://se.mathworks.com/help/simulink/ug/controlling-and-displaying-the-sorted-order.html}. Basically, the list is determined according to the data dependency of blocks' outputs on the blocks' input ports, i.e., if the output depends on the current value of the input, the input port is identified as \textit{direct-feedthrough} port. Thus, to preserve the data dependency in the model, the sort-order rules require that the blocks that derive other blocks with direct-feedthrough ports must come first in the list, e.g., blocks that derive Gain block. However, the blocks with non-direct-feedthrough ports, e.g., Delay block, can execute in any order, considering the previous rule applies. The execution order is also affected by the user defined priorities, nevertheless, the priorities do not violate the rules. The sorted order list can be fetched by simulating the model in the debugging mode.
\begin{figure}
	\centering
	\includegraphics[width=1\linewidth]{images/sm_exec_order}
	\caption{A Subsystem block from the brake-by-wire model, computes vehicle speed. The labes in red color are indexed identifiers and denote of the execution order.}
	\label{fig_sm_exec_order}
\end{figure}

\section{Stochastic Timed Automata}
The theory of stochastic timed automata builds upon the timed automata theory via probability to support stochastic behavior in modeling and analysis of real-time systems. It can be used to model continuous-time behaviors, e.g., pressing the brake pedal, which occurs at continuous time points, and others, e.g., waiting time in network communication, uncertainly in input values, soft real-time deadlines. 

A timed automaton $\mathcal{A}$ is a finite-state automaton proposed by Alur et al.~\cite{Alur1999TimedAutomata} to model and analyze real-time systems, i.e., by extending the automaton with real-valued clock variables $X$. It is widely used in the model checking to verify functional and timing requirements, e.g.,liveness, safety properties. The stochastic version automaton $\langle\mathcal{A},\mu,\gamma\rangle$ basically extends the timed automaton with probability, i.e., on the delays between states via the delay density functions $\mu$ and discrete transitions between locations via the output probability functions $\gamma$, where  $\mathcal{A}$ is represented as a tuple $\langle L,l_0,X,\Sigma,E,I\rangle$, and its elements are described as follows: $L$ is a set of locations of which $l_0$ is the initial location, $E$ is the set of edges between locations,  $\Sigma$ is a set of action labels, and $I$ is a set of invariants, which are Boolean constraints over $X$ assigned to locations, i.e., the system (or automaton) can only stay in the location as long as its invariants hold. An edge $(l,g,a,r,l')\in E$ denotes a transition relation from the location $l$ to $l'$, with conjunctions of guards $g$, action $a$ and clock resets $r$. The guard is also a Boolean constraint over $X$, and when it holds, the transition on the edge is enabled.
\begin{figure}
	\centering
	\includegraphics[width=0.6\linewidth]{images/sta_dummy_example}
	\caption{STA example.}
	\label{fig_staexample}
\end{figure}

\begin{example}[STA Example] Figure~\ref{fig_staexample} is a dummy STA example with three locations A,B and C, where A is the initial location. The guards of the automaton are labeled in Green color, the clock resets in Blue and the constraints in Red.
\end{example}

Assume that, $d(l,v)$ is the infimum delay which satisfies the disjunction of guards on the edge that start from $l$, and assume that, $D(l,v)$ is the superimum delay which satisfies the invariant $I(l)$. If the delay is bounded, i.e., there exists $D(l,v)$, in location $l$, each delay density function in $l$, $\mu_s$, is assumed to be uniformly distributed on the interval $[d(l,v),D(l,v)]$. However, if the location is time unbounded, the delays are assumed to be exponentially distributed with the rate $R(l)$. To illustrate this point, consider the same STA example, the delays at the location A are exponentially distributed, i.e., with rate 100, since the the location is time unbounded, and delays at B and C are uniformly distributed on the interval $[4,8]$ and $[10,10]$, respectively. The output probability of locations A and C is 1 and of location B is 1/2. 
%Note: logical path, and probability of taking the path is purely stochastic

\subsection*{Network of Stochastic Timed Automata}
Under the assumption of input-enabledness, disjointness of clock sets and output action sets, a network of STA (denoted by NSTA) is parallel composition of $STA_i$, i.e., $(STA||STA2||,...,||STAn)$, where n is the number of STA in the netowrk.  The state of a networked TA is a tuple $\langle s_1,s_2,...,s_n\rangle$, where $s_i=(l,v)\in L_i\times X_i$ . In NSTA, the automata communicate through broadcast channels and globally shared variables. Moreover, the semantics of the NSTA relies on the independent computation of the individual STA (or components), i.e.,  each component automaton, based on the delay density functions and the output probability functions decides repeatedly on which output to generate and at which point in time. In this race, the output will be determined by the component automaton that has chosen to produce the output after the minimum delay~\cite{David2011StatisticalAutomata}.


\section{Statistical Model Checking}
Statistical model checking~\cite{Agha2018AChecking} uses a finite set of randomly selected execution traces (or runs) of the state-based system $S$ to check if the traces provide probability evidence for accepting or rejecting the specification (or property) $\varphi$, i.e., $T\in Traces(S)\models \varphi$. It applies statistical techniques such as hypothesis testing and Monte Carlo simulation to compute the probability evidence. Since most interesting problems in timed automata, e.g., the reachability problem, become undecidable in NPTA, statistical model checking is a viable alternative to approximate the problems via probability estimation. Moreover, in contrast to the exact model checking technique, statistical model checking as any other Simulation-based techniques consumes much less memory and CPU, and returns results in much less computation time..

\uppaalsmc{} is an extension of the model checker \uppaal with support for statistical model checking. It admits a system modeled as NPTA, and the properties to checked are specified in the probabilistic extension of the WMTL (or PWMTL) logic~\cite{Bulychev2012UPPAAL-SMC:Automata}, which are expressed inductively as follows:
\begin{align}
\varphi::=P(\Diamond_{c\ C}\phi)\bowtie p | P(\Box c\leq t)\bowtie p
\end{align}
where $c$ is a clock, $C\in \mathbb{N}$ is a bound, $\phi$ is a state-predicate, $\bowtie \in\{\leq,\geq\}$, and $p \in [0, 1]$.

\uppaalsmc~\cite{David2011TimeSystems}\cite{Bulychev2012UPPAAL-SMC:Automata} can be used to check quantitative and qualitative properties via \textit{hypothesis testing} and \textit{probability comparison}, and \textit{probability evaluation}, respectively. The
hypothesis testing is a qualitative statistical method which checks if the probability of satisfying a PWMTL property is less than or equal to a certain probability threshold $p$ over a network of STA. The probability of making hypothesis testing errors is bounded by the strength parameters $\alpha$ and $\beta$, where $\alpha$ is the probability of accepting the null hypothesis while the opposite of it holds (false negative), and $\beta$ is the probability of accepting the opposite of the null hypothesis while the null hypothesis holds (false positive). Another qualitative property that can be checked using \uppaalsmc{} is the comparison of probabilities satisfying two PWMTL properties.

The probability evaluation is a quantitative method which determines the probability of satisfying a PWMTL property, $p$, over a network of NPTA. The evaluation is an approximation interval $[p-\mu,p+\mu]$ with a confidence $1-\alpha$, where $\mu$ and $\alpha$ are user configurable parameters. The number of traces are determined based on the parameters automatically by the model checker.
\vspace{2cm}
\section{Integer linear Programming}
An integer linear programming (or integer programming, \ilp) is a type of optimization problem which consists of integer decision variables, and the objective function and the constraints are linear. The binary (or 0-1) \ilp{} problem is a special case of ILP whose decision variables are binary, i.e., takes 0 or 1. It is represented as:
\begin{align}
    \label{eqn_ilp_obj}
	&Maximize\ \sum_{j=1}^{n}{c_jx_j}\\\nonumber
	\mbox{Subjec to:}&\\\nonumber
	&\sum_{j=1}^n{a_{ij}x_j}\leq b_i&\mbox{ for all } i=1,...,m\\
	&x_j\in\{0,1\} &\mbox{ for all } j=1,...,n
\end{align}
where (\ref{eqn_ilp_obj}) is the objective function with $n$ cost coefficients $c_j:j=1,...,n$, (\ref{eqn_ilp_obj}) are $m$ linear constraints with coefficients $a_j:1,...,m$ and requirements $b_j:1,...,m$, and $x_j:1,...,n$ are binary decision variables.

ILP is used in many control and engineering applications, e.g., fleet and flight routing in transportation, airfoil design in avionics engineering, controller design in industrial automation, automotive, etc., resource allocation. The binary (0-1) variant is applied in several resource allocation problems, e.g., assignment of operating system tasks to processors. The software allocation (i.e., in component-based software development) can be formulated as a binary integer programming problem, where $x_j=1$ denotes a component is mapped to the $j^{th}$ processor and $x_j=0$ denotes a component is not mapped to the $j^{th}$ processor.

ILP problems are frequently solved via exact algorithms, e.g., branch and bound, but can also be dealt via heuristics, e.g., using simulated annealing, hill climbing. In this work, we use the CPLEX solver form IBM to the software allocation optimization.
\section{Population-based Metaheuristics}
The ILP approach and for that matter exact methods do not scale well on complex optimization problems, or else their application is usually prohibitively expensive when applied on large-scale problems, i.e., with many decision variables. In contrast, metaheuristics~\cite{2006HandbookMetaheuristics}\cite{Gonzalez2007HandbookMetaheuristics}, which is a type of heuristics with search strategy, is more efficient computation-wise, albeit less optimal. The population-based metaheuristics is a class of meta-heuristic algorithms which uses a set of individuals (or population) to guide the search strategy and determine the global optima of the problem. The most common population-based algorithms consist of evolutionary algorithms, differential evolution~\cite{Storn1997DifferentialSpaces}\cite{Das2016RecentSurvey}, particle swarm optimization~\cite{Poli2008AnApplications}\cite{Mirjalili2019ParticleOptimisation}.

In this thesis work, i.e., for the efficient software allocation, we apply the differential evolution, particle-swarm optimization and the latter's hybrid with local search algorithms such as hill climbing and stochastic hill climbing to find the best (or near optimal) solution. In the case of differential evolution, special evolutionary operators, i.e., mutation, crossover and selection, are employed to update the population, thus traverse across the search space to reach at the global near-optima. The technique is briefly describe as follows: a mutant agent (or individual), i.e., the offspring $\textbf{v}$, is generated for every agent from three other agents in each iteration according to (\ref{eqn_de_mutation}), subsequently combined using the crossover operator (\ref{eqn_de_crossover}). If the offspring $\textbf{u}$ is more fit than the agent $\textbf{x}$, i.e., closer to the best solution, it replaces the agent otherwise is discarded as shown in the selection strategy (\ref{eqn_de_selection}).
\begin{align}
\label{eqn_de_mutation}
\textbf{v} & \leftarrow   \textbf{a} + F\circ(\textbf{b}-\textbf{c})\\
\label{eqn_de_crossover}
\textbf{u}&\leftarrow crossOver(\textbf{v},\textbf{x},CF,F)\\
\label{eqn_de_selection}
\textbf{x} &\leftarrow 
\begin{cases}
	\textbf{u} & \mbox{if } f(\textbf{u}) < f(\textbf{x})\mbox{ functions}\\
	\textbf{x} & \mbox{otherwise }
\end{cases}
\end{align}
where $F\in[0,2]$ is known as differential weight and controls the diversity of the mutant, and $CF\in[0,1]$ is known as the crossover probability and controls the degree of crossover between the agent and the mutant.

Similarly, the particle-swarm optimization employs individuals (or particles in this case). The particles unlike in the differential evolution are memory-based, i.e., they record their best position so far $\textbf{p}_{bst}$, and the algorithm remembers the best position of the population $\textbf{z}$. The motion of each particle is guided by its velocity (\ref{eqn_pso_velocity}) and attractions towards its best position $\textbf{p}_{bst}-\textbf{p}$ and towards the best position of the population $\textbf{z}-\textbf{p}$ (\ref{eqn_pso_position}), where $\textbf{p}$ is an n-dimensional matrix which represents the current position of a particle.
\begin{align}
\label{eqn_pso_velocity}
\textbf{v} &\leftarrow  \omega\textbf{v} + c_1Rand()\circ(\textbf{p}_{bst}-\textbf{p}) + c_2Rand()\circ(\textbf{z}-\textbf{p})\\
\label{eqn_pso_position}
\textbf{p} &\leftarrow \textbf{p} + \textbf{v},
\end{align}
where $\omega$ is the weight of the velocity, also known as \textit{inertia coefficient} and controls the convergence of the algorithm. The $c_1, c_2$ constants are acceleration coefficients and control the weight of attraction towards the cognitive and social components, respectively. $Rand()\in U(0,1)$ is a random function along the acceleration coefficients, which is element-wise multiplied with the components to improve diversity of the search by introducing stochastic behavior
    \chapter{Problem Formulation}
Over the last decades, the complexity of the safety-critical software has been on the rise which is evident on the modern cars, which implement many and complex automotive functions, and also on the emergence of the electrical and autonomous vehicles. Thus, the thesis is motivated by the need for advanced (or rigorous) methods to the requirements specification, modeling and analysis of the complex safety-critical automotive software, and their seamless integration into the existing methods and tools of the automotive systems development at VGTT and Scania. Furthermore, the thesis is motivated by the need for efficient mapping of safety-critical software to hardware in the distributed computing to facilitate software extensibiliy and support the increasing functionality of the automotive systems. 

Thus, the \textit{overall goal} of the thesis is to:
\begin{mybox}[attach title to upper={\ ---\ }]{Overall Goal}
	Provide assurance of safety-critical functionality, at the various levels of abstraction, via formal analysis, and optimization of critical system resources.
\end{mybox}

The overall goal is refined via \textit{research goals}, which state the needs or concerns that the thesis should address and are formulated as follows:

\section{Research Goals}\label{research_challenges}
Many safety-critical automotive systems are developed according to the ISO 26262 standard, which recommends highly the use of semi-formal languages to specify safety-critical requirements to improve quality of the specifications, e.g., by reducing ambiguity and improving comprehensibility. In the context of textual representations, the semi-formal specification methods are constrained natural languages, such as templates (e.g., requirements boilerplates~\cite{Farfeleder2011DODT:Development}\cite{Mahmud2015ReSA:Systems}), controlled natural languages~\cite{Kuhn2014ALanguages}(e.g., Attempto~\cite{attempto96}).

The template-based methods inherently lack meta-model (or grammar), therefore is difficult to add new templates effectively, moreover, template selection is usually cumbersome. The existing controlled natural languages lack effective support of specifying embedded systems requirements.

Thus, the first research goal is to:
\setcounter{rgcounter}{1}
\begin{researchgoal}
Reduce ambiguity and improve the comprehensibility of natural-language requirements using domain-specific knowledge of embedded systems.
\end{researchgoal}

One of the mechanisms to improve natural language specifications is by constraining the language, including its syntax, semantics and the lexicon \cite{Kuhn2014ALanguages}. The design of a constrained natural language for the specification of requirements is not trivial. By constraining the language, its expressiveness and intuitiveness can be impaired~\cite{ieereqspecstandard}\cite{Myachykov2013SyntacticRussian}, therefore, appropriate trade-offs should be made during the design in order to have a robust and effective specification language.

Besides improving quality of individual requirements, the latter should be analyzed in ensemble in order to detect errors that span multiple specifications, e.g., logical contradictions. However, natural language lacks formal (or precise and unambiguous) semantics, therefore is difficult to rigorously analyze (or reason) natural-language requirements specifications. There are several methods to natural language semantics, of which the use of \textit{logic} is common~\cite{Clark2010TheProcessing}.

Thus, the second research goal is to:
\begin{researchgoal}
Facilitate formal analysis of the requirements specifications through transformation to Boolean and description logics.
\end{researchgoal}

Natural language specifications are constructed from syntactic units, such as words, phrases, clauses, statements, etc. Consequently, rigorous analysis of the specifications involve parsing and interpreting the syntactic units, which is a complex problem in computational linguistics~\cite{Clark2010TheProcessing}. The depth of the interpretation (or semantics) greatly affects the applicability of the methods, e.g., the propositional logic representation of the specifications is simple and the analysis scales well, however, it is shallow as it abstracts away the details. On the other hand, the first-order-logic representations are more rigor, thus enable thorough analysis but are less tractable. Therefore, appropriate interpretation of the natural language specifications is crucial.

The software designs and software-design units (or behavioral models) should conform to the requirements specifications. We consider the software-design units are modeled in Simulink, which is the most widely used model-based development environment in industry to model and simulate the behavior of multi-domain, discrete, continuous embedded systems. Simulink also enables the generation of code from discrete Simulink models which directly execute on specific platforms, thus is crucial to conduct rigorous analysis of the Simulink models to reduce errors introduced in the generated code.

The de~facto Simulink analysis techniques, e.g., by type checking, simulation, and formal verification via the Simulink Design Verifier (SDV\footnote{https://se.mathworks.com/products/sldesignverifier.html}) are not sufficient to address the full correctness of safety-critical real-time Simulink models. SDV lacks support for checking temporal correctness as specified in timed properties, e.g., in TCTL, and also lacks support for verifying continuous models and suffers from scalability due to its reliance on the exact model-checking~\cite{Leitner2008SimulinkStudy}. In contrast to the exact model checking, the statistical model-checking verifies properties over sufficiently collected traces of system simulations via statistical methods. It scales better over the trade-off for exhaustiveness. 

Thus, the third research goal of the thesis is to:
\begin{researchgoal}
Enable formal analysis of large-scale, multi-rate and hybrid Simulink models using statistical model-checking.
\end{researchgoal}

Simulink consists of connected and hierarchical Simulink blocks, which encode mathematical functions~\cite{JamesB.Dabney2003MasteringSimulink}. For industrial systems, the number of blocks in a Simulink model can be in the order of thousands, and the blocks can be triggered with different sampling frequencies for discrete blocks and without any sampling frequency 
for continuous blocks. Therefore, typical industrial Simulink models are usually complex and comprise mixed signals, multiple rates, discrete and continues Simulink blocks, making the model checking challenging.

In the distributed computing, the automotive software is allocated on multiple computing units (or ECU), consequently is exposed to higher permanent and transient faults, hence necessitates to maximize the reliability of the safety-critical software system. Fault tolerance using redundancy is the most widely approach to improve reliability such as by replicating software functionality on multiple ECU, however, it requires additional critical system resources such as power sources. In this regard, the software-to-hardware allocation plays a crucial role to minimize the power consumption of fault-tolerant distributed safety-critical software while satisfying the timing and reliability constraints of the safety-critical software.

Thus, the fourth research goal is to
\begin{researchgoal}
 Minimize the power consumption of distributed safety-critical software while satisfying the timing and reliability constraints during the software allocation.
\end{researchgoal}

The software allocation model is not trivial as we consider exact method of schedulability analysis and reliability calculations, which makes the optimization complex. We assume a sufficient and necessary scheduling test based on the worst-case response time analysis~\cite{Baruah2011Response-timeSystems}\cite{Davis2007ControllerRevised}, moreover end-to-end delay analysis based on the age delay semantics~\cite{mubeen2013support}, which is practical but computationally expensive. For the reliability analysis, we apply an exact method of calculation based on the state enumeration, in contrast to the series-parallel method, which is trivial and computationally less expensive. As a result, we consider an exact optimization method using ILP for relatively small and metaheuristics for relatively large software allocation problems.

In order to show the validity of our proposed solutions, a working prototype should be developed and should also be evaluated on industrial uses cases. The validation should consider scalability and engineer-friendliness of methods and tools besides effectiveness. 

Thus, the last research goal is to:
\begin{researchgoal}
 Provide automated and engineering-friendly support for the requirements specification, software allocation of embedded  and formal analysis of Simulink models.
\end{researchgoal}

Seamless integration of our proposed methods and tools into the existing development process require close cooperation between the domain experts and the practitioners. The role of the domain experts should be to simplify usage of the tools, e.g., by rendering their interface to existing once, etc., and the practitioners should cooperate with materials that assist the validation of the proposed solutions. The cooperation is not trivial considering the challenge of formal methods, and companies culture for being restrictive.

    \chapter{Thesis Contributions}
The general contribution of the thesis is a safety-critical design approach that employs formal methods at various stages of software development such as requirements specification and software design, and a power-efficient allocation of software to hardware while satisfying timing and reliability of the software in a distributed environment. It satisfies the research goals explained in Subsection~\ref{research_challenges} via the following contributions: a requirements specification language tailored to embedded systems~\cite{Mahmud2015ReSA:Systems}\cite{resatool}, formal analysis of the specifications~\cite{resatool}\cite{Mahmud2017SpecificationLogic} via formal and knowledge-based methods, formal analysis of large-scale Simulink models~\cite{Filipovikj2018SimppaalModels} and efficient mapping of software to hardware in the context of distributed architecture~\cite{Mahmud5222}\cite{Mahmud2019Power-awareOptimization} via integer-linear programming and metaheuristics. 

\section{Design Workflow: Contributions Overview}
Figure~\ref{fig_workflow} illustrates the workflow of an embedded software development that make of use our contributions. The workflow is contextualized in the automotive software development where Simulink, EAST-ADL and AUTOSAR architectural language are used. 
\begin{figure}[h]
	\centering
	\ifpdf
	\includegraphics[width=\linewidth]{images/workflow}
	\else
	\includegraphics[width=.8\linewidth]{images/workflow.eps}
	\fi
	\caption{Thesis contributions workflow.} 
	\label{fig_workflow}
\end{figure}

Initially the textual requirements are specified in our domain-specific language ReSA, which is a constrained natural language tailored to embedded systems. The ReSA editor supports content completion as it is connected to a system model, which contains instantiation of words/phrases. If the ReSA editor is required to be domain-specific, e.g., to the automotive domain that employs EAST-ADL, the system model is specialized to the later model, thus enabling requirements specifications by accessing ESAST-ADL model elements from the editor. To check the consistency, the ReSA specifications are translated into Boolean and description logic, subsequently, analyzed via SAT solver and inference engine (or reasoner), respectively.

After the quality of the specifications are improved, i.e., become unambiguous,  comprehensible, consistent, they can be used in the verification of behavioral software designs. In this thesis, we consider the software design is made via Simulink, which is one of the most widely integerated environment for modeling, simulation and analysis of embedded system models. In order to rigorously analyze a Simulink model, we transform it automatically into a formal model, i.e., a network of stochastic timed automata. Subsequently, the latter is analyzed via statistical model checking against properties initially expressed in ReSA, but manually translated into the query (properties) language of the model checker, i.e., probabilistic weighed-metric temporal logic (PWMTL).

The Simulink model, at the architectural level, is represented by an EAST-ADL software architecture at the implementation level, i.e., via an AUTOSAR software architecture, which consists of software components that communicate through virtual function bus (VBS).  Note that, the refinement from the AUTOSAR model to the Simulink model is manual, which is the normal practice, however, the Simulink model composition (or structure) is automatically generated. The software architecture complements the Simiulink model with computational resource specifications, i.e., via runnables, which are schedulable piece of code (or objects) by the AUTOSAR OS. Moreover, it complements with other resource specifications, e.g., power specification, interface specification of the underlaying hardwares. In this regard, we propose (near) optimal software to hardware optimization techniques that incorporate the timing and reliability of software applications as constraints, and optimizes the power consumption of the software.

The thesis contributions are summarized as follows:
\section{Contr.~1: The ReSA Language}\label{rc_resa}
We propose a domain-specific and constrained natural language, both at the syntax and semantics level, which is designed to improve comprehensibility and reduce ambiguity of embedded systems requirement specifications. The language employs concepts from the embedded systems domain, e.g., \textit{System}, \textit{Parameter}, \textit{Device}, \textit{State}, \textit{Mode}, \textit{Entity}, to typeset domain-specific words/phrases. Besides grammatical rules (or syntax), semantic relations between the concepts limit on the possible construction of the specifications, e.g., grammatically the ``The ASL shall limit the driver.'' specification is correct, but semantically, it makes no sense. To solve this problem, the ReSA type system imposes on the a type constraint on the relation between instances ``The ASL''\textit{:system} and ``the driver''\textit{:user}.

\subsection*{Syntax}
The language design is modular, i.e., it uses sentence catagories, e.g., \textit{Simple}, \textit{Compound}, to structure the specifications, thus improve comprehensibility and reduce ambiguity. It also allow complex specifications by inductively applying the grammer rules of the language. The following context-free grammar is a snippet of the ReSA language. 
\begin{bnf*}
	\bnfprod{specification}
	{\bnfpn{simple} \bnfor \bnfpn{complex}\bnfor\bnfpn{nested-complex} ...}\\
	\bnfprod{simple}
	{\bnfpn{mainClause} .} \\%\bnftd
		\bnfprod{complex}
	{\bnfpn{subClause}, \bnfpn{mainClause}.} \\%\bnftd
	\bnfprod{mainClause}
	{\bnfpn{primitiveClause}\bnfor ...} \\%\bnftd
		\bnfprod{primitiveClause}
	{\bnfpn{sub} \bnfpn{verb} \bnfpn{obj}\bnfor...} \\%\bnftd
	\vdots
\end{bnf*}
{\footnotesize Note: please checkout the ReSA documentation for the complete grammar and its tool support from its webpage located at the URL {https://bitbucket.org/nasmdh/resa/src/master/}. }
\begin{example}[Adjustable Speed Limiter (ASL) Requirements]\label{ex_resa_example}
	ASL is a speed control system found in Volvo trucks. It controls the vehicle speed to not exceed a certain speed, which is set by the driver or legal authorities. It consists around 300 fuctional and extra-functional requirements. Just to demo the usage of the language, we specify only four of ASL equirements.
\begin{elaboration}{0.95}
	\small
		\begin{tabular}{lp{.9\textwidth}}
			R1 & ASL:system shall send "the driver"\textit{:user} notification:status every 200ms. \\
			R2 &\begin{tabular}[t]{@{}l@{}}if ``driver:user'' selects ``ASL speed control''\textit{:mode} and\\
				\hspace{0.5cm}(vehicle is in ``pre-running'' mode or
				vehicle is in ``running'' mode)\\ then\\
				\hspace{0.5cm}ASL:system shall be enabled within 200ms and\\
				\hspace{0.5cm}``ASL enabled''\textit{:status} shall be presented to ``the driver''\textit{:user} \\ endif\end{tabular} \\
			R3 & After ASL:system is enabled, if IncButton\textit{:inDevice} is pressed, ASL:system shall be activated.\\
			R4 &if driver selects ``ASL speed control''\textit{:mode}
			then 
			ASL\textit{:system} shall be disabled.
		\end{tabular}
\end{elaboration}
\end{example}

\subsection*{Tool Support and Validation on ASL}
The language is implemented in the Xtext framework\footnote{\url{Xtext: https://www.eclipse.org/Xtext/}}, which is an Eclibse-based integrated development environment for the development of domain-specific and general-purpose languages. The ReSA editor, which is shown in Figure~\ref{fig_resagui}, supports content completion, errors/warring messages and boilerplate management. The editor is integrated to EATOP, i.e., the editor can be trigged from within the IDE, moreover, the content-completion feature dispalys contents of the EAST-ADL model elements, hence enabling consistent use of vocabularies during specification. The ReSA tool as a standalone, and as a plugin for EATOP, can be downloaded from the URL {\small \url{https://bitbucket.org/dashboard/overview}}. 
\begin{figure}[h]
	\centering
	\includegraphics[width=0.7\linewidth]{images/resagui}
	\caption{The ReSA graphical user interface (GUI).}
	\label{fig_resagui}
\end{figure}

The language as well as its tool support is validated on the 300 ASL requirements, which are initially expressed in natural lanugage. The requirements distribution is as shown in Figure~\ref{fig_ASLreqs}, i.e., in terms of the types of requirments. The requirements are rewritten in ReSA, and Figure~\ref{fig_ASLreqsresult}shows the sentence catagories (or boilerplates) employed to specify the requirments. In general, the language is expressive,  though is constrained., and over a period of time, more and more constructs are added to improve the expressiveness.
 \begin{figure}[h] 
 	\centering
 	\subfloat[Types of ASL Requirements.\label{fig_ASLreqs}]{% 
 		\centering
 		\includegraphics[width=0.45\textwidth]{images/aslreq}
 	} \hfill
 	\subfloat[Distribution of ASL boilerplate types.\label{fig_ASLreqsresult}]{% 
 		\centering
 		\includegraphics[width=0.45\textwidth]{images/aslbp} 
 	} 
 	\caption{Requirements specification ontology (screenshot from Prot\'eg\'e tool).} \label{fig_ontology}
 \end{figure}
\section{Contr.~2: {Formal Analysis of ReSA Specifications}}\label{rc_resaanalysis}
The editor shows syntax and typing warning and error messages. However, such static analysis are rudimentary since the analysis per specification. In this contribution, we propose two methods of consistency analysis: \textit{SAT-based} anaysis and \textit{ontology-based} analysis. 

\subsection*{SAT-based Analysis}
Boolean satisfiability (also known as SAT) can informally defined as the problem of finding truth-values (or assignments of boolean variables) such that a Boolean formula holds. It is NP-complete, however, there exist efficient heuristic algorithm which solve SAT problems of large-size, i.e., that consist of thousands of Boolean variables. Thus, first we transform the ReSA specifications into Boolean expressions~\cite{resatool}, then we formulate the problem of finding inconsistency the Boolean expressions in terms of SAT as follows:
\begin{definition}[Inconsistency of Specifications]
	Let $\Phi = \{\Phi_1, \Phi_2,\dots,\Phi_n\}$ is the ReSA specifications after translating ReSA specifications into Boolean expressions, where  $\Phi\in N$ is a Boolean formula and denotes a requirement specification. Thus, the set Boolean expressions $\Phi$ is inconsistent if the expression $\Phi_1 \land \Phi_2 \land\dots\land \Phi_n \Rightarrow false$ holds, that is, there exists at least one expression that cannot be satisfied, i.e., $\Phi_i\models false$.
\end{definition}
\begin{example}
Assume that we want to check the consistency of the requirements that are specified in Example~\ref{ex_resa_example}. Figure~\ref{fig_z3} shows the translation of the specifications into Boolean expressions (i.e. in Z3  SMT-LIB format). Each expression is asserted before the Z3 solver is called using the command $(check-sat)$. Some clauses are resolved for negation and opposite words, e.g., the fact that ``enabled'' is opposite of ``disabled'' implies $p_5=p_7\neq p_{11}$. The solver returned ``unsat (R2 R4 R1\_Assertion)'',  which means, the expressions are inconsistent and indicates the specifications that are responsible for the inconsistency (or unsat-core).
\end{example}
\begin{figure}[h]
	\centering
	\includegraphics[width=0.7\linewidth]{images/z3}
	\caption{The ReSA specifications R1- R4 encoded as Z3 format, and the unsat-core feedback from the Z3 solver, to localize source of the inconsistency.}
	\label{fig_z3}
\end{figure}

The SAT-based analysis, although, easy and scalable, it does not allow rigorously analysis. This is due to the fact that the Boolean (or propositional) variables abstract away details of the clauses in the specifications. The ontology-based approach solves this problem with trade off over scalability.

\subsection*{Ontology-based Analysis}
Ontology as a knowledge representation technique can be used to capture the intricate relations of words/phrases of specifications, e.g., synonyms, antonyms, generalization. Morover, it can be used to capture the syntax of the specifications as well, essentially modeling the the ReSA specification as a concrete requirements specification ontology. 

Besides the challenge of constructing the ontology, accurate representation of the specifiactions is crucial to a certain degree for the ontology to be useful, i.e., the specifications should be interpreted linguistically. In this research, we propose the \textit{event-based} semantics approach~\cite{Mahmud2017SpecificationLogic} to construct the meaning of each specification compositionally fromts its constituents (or grammar units). The event-based semantics, which is based on the first-order logic, uses an existentially quantified events to relate words/phrases (or arguments of predicate) , clauses, adjuncts of a sentence~\cite{Mahmud2017SpecificationLogic}, e.g., the clause ``ASL:system shall limit ``vehicle speed'':parameter'' is represented as $\exists e$[limiting($e$) \& Agent(ASL) \& Recipient(vehicle speed)], where Agent and Recipient are known as \textit{thematic} roles, which define the semantic roles of the arguments in the clause.  Since the requirement specifications entail technical words/phrase, the thematic roles of these arguments are not obvious, so we extended the thematic roles into the ReSA concepts, e.g., any instance of \textit{System} or \textit{User} is \textit{Agent}, similarly instances of \textit{Parameter} is \textit{Recipient}. In this way, the interpretation of the clauses become domain-specific.

Following the event-based semantics, the ReSA specifications are translated into description logic as ontology. Once the ontology is constructed considering the event-based semantic, thematic roles, lexical relation, we check the consistency of it via an inference engine (or reasoner), which is a software program that applies logical rules on a logical system, such as the ontology, to obtain new information.
\begin{definition}[Inconsistency of Specifications]
	The requirements specification ontology is inconsistent if there does not exists an interpretation (or a model) $M$ that satisfies the terminological assertions $T$ and the concrete assertions $A$, that is, $M \not\models ax$, where: $ax \in T \cup A$, and $T$ is a set of terminological assertions and $A$  a set of facts (or concrete assertions).
\end{definition}
%\begin{figure}
%	\centering
%	\subfloat[Concepts.\label{fig_continuous}]{% 
%		\includegraphics[width=0.3\textwidth]{images/ro_concepts}
%	} \hfill
%	\subfloat[Object properties.\label{fig_discrete}]{% 
%		\includegraphics[width=0.3\textwidth]{images/ro_objects} 
%	} 
%\hfill
%\subfloat[Individuals.\label{fig_discrete}]{% 
%	\includegraphics[width=0.3\textwidth]{images/ro_individuals} 
%} 
%	\caption{Requirements specification ontology (screenshot from Prot\'eg\'e tool).} \label{fig_ontology}
%\end{figure}

\section{Contr.~3: {Scalable Formal Analysis of Simulink Models}}\label{rc_sim}
Several research exist on formal anlaysis of Simulink models, e.g., using theorem proving, exact model checking, statistical model checking. Many of the proposed solutions are  limited either due to frequent user involvement or scalability issues. In this thesis, we propse a scalable formal analysis of larege-scale Simulink models via statistical model checking. The Simulink models can be discrete, continuous or hybrid, moreover, they can consist of atomic and composite Simulink blocks that are scattered on multiple files, which is a practical scenario. Essentially, we consider typical industrial Simulink models, e.g., the brake-by-wire and adjustable speed limiter Simulink models from VGTT~\cite{Filipovikj2018SimppaalModels}.

Our approach has the following important characterstics: (i) the transformation employs continous and discrete transformation patterns, which are generic and reusable, thus can be applied to any Simulink blocks; (ii) the transformation preserves the execution orders of the Simulink blocks; (iii) it is robust, i.e., it handles various models such as continuos, discrete, hybrid, and models which contain blocks that execute with different sampling rates, also known as multi-rate.

\subsection*{Transformation Patterns} 
The continuous-time and discrete-time stochastic timed automata (STA), which are shown in Figure~\ref{fig_patterns}, are used to transform the continuous-time and discrete-time Simulink blocks into their respective STAs. The automata in both cases makes transition from the ``Start'' location to the ``Operate'' location at the global time $sn*IAT$, which is determined according to each block's order of execution $sn$, i.e., the lower $sn$, the sooner the automaton makes  transition to the ``Operate'' location, and $IAT$ is an infinitesimal inter-arrival time between the release of the automata. In the case of the continuous-time STA pattern, the automaton makes a loop transition at the ``Operate'' location every infinitesimal sample time, which is distributed exponentially according to the rate  $\lambda=1000$. Whereas in the case of discrete-time STA pattern, it makes a loop transition every $ts$ sample time with probability of 1, as there is only one edge that goes out of the location and the delay transition is distributed uniformly in the interval $[ts,ts]$.
\begin{figure}[h] 
	\centering
	\subfloat[Continuous-time.\label{fig_continuous}]{% 
		\includegraphics[width=0.45\textwidth]{images/continuous-sta}
	} ~
	\subfloat[Discrete-time.\label{fig_discrete}]{% 
		\includegraphics[width=0.45\textwidth]{images/discrete-sta}
	} 
	\caption{STA transformation patterns.} 
	\label{fig_patterns}
\end{figure}

\subsection*{Simulink to NSTA Transformation Process} 
Besides proposing the transformation patterns, we provide an automatic transformation of a Simulink model to its equivalent network of STA by applying the following tasks:
\begin{enumerate*}[label=(\roman*)]
	\item the Simulink model is simulated, subsequently, the sorted-order list, which contains the execution order of the Simulink blocks, is extracted.
	\item in case the model is hierarchical, the execution order is flattened first, thus generating a flattened sorted-order list;
	\item the Simulink model is parsed, as a result, each block is translated into a stochastic timed automata via the corresponding transformation pattern. The sorted-order number $sn$ is instantiated from the sorted-order list on the fly.
	\item any non-computational blocks is accounted in the transformation but are not trasformed into automata, e.g., the Mux, SubSystem blocks;
	\item the connections between the blocks are translated into global variables, which acts as communication between the automata.
\end{enumerate*}
Please checkout Section~3 of~\cite{Filipovikj2018SimppaalModels} for the detailed discussion on the transformation as well as its soundness proof.

%\begin{figure}
%	\centering
%	\includegraphics[width=0.6\linewidth]{images/simulinkapproach}
%	\caption{Simulink to NSTA Transformation}
%	\label{fig_approachworkflow}
%\end{figure}

\subsection*{Validation on the Brake-by-wire Use Case} 
Our approach is applied on an industrial brake-by-wire Simulink model, which is a prototype developed for academic use by VGTT. The model consists of 320 blocks, of which 19 are discrete blocks, 26 constant blocks and the rest are continuous-blocks. The model design consists of of modules that collects sensory data, e.g., vehicle speed, pedal position, and computes the proportional torque force that is applied on the wheels to brake the vehicle. Besides, it has the  anti-braking functionality (i.e., the ABS) to avoid (or minimize) skidding, which occurs when the slip rate of a wheel is greater than its friction coefficient. 

The BBW requirements are initially specified in ReSA, subsequently are translated into PWMTL properties manually by using monitors (or observers), which are stopwatch stochastic priced timed automata. The monitors are modeled for particular requirements, and basically observe the progressionf the execution of the automata.
\begin{example}[Statistical model checking of BBW]\label{ex_resa}
	The BBW model is automatically transformed into a network of STAs, and we modeled the monitor (Appendix A) to represent 5 functional and timing requirements, of which 3 requirements are shown in the box below for illustration. \vspace{0.2cm}
\begin{elaboration}{0.9}
	\small
	{}
	\begin{tabular}{lp{.85\textwidth}}
	R1 & The brake request shall be computed within 10 ms. \\
	R3 & The brake request shall be be propagated to two different wheel actuators within 4 ms.\\
	R4 & If the brake request is 0, then the ABS shall set the torque to 0. \\
\end{tabular}
\end{elaboration}
\end{example}

Figure~\ref{fig_smcresult} shows the PWMTL propeties and their model checking results. R1 and R3 are safety requirements which are expressed using the $\Box$ operator as indicated by R1$_{BBW}$ and R3$_{BBW}$ PWMTL properties, respectively. R4 is a conditional requirement which is expressed using the $\Diamond$ operator as indicated by R4$_{BBW}$ PWMTL property. Each result shows the probability interval of satisfying the property (if qualitative) or the probability of satisfying a property (if quantitatiy), confidence level, the number of runs and the time needed by the model checker to return the result. For all properties, the probability interval (or confidence interval) approximates $[0.99,1]$ with confidence level $\approx0.999$, which can be interpreted as ``the properties are satisfied`` if the probability requirements lower-bound are in the degree of 0.99. However, in many safety-critical applications, the requirement on the probability is normally high, e.g., 0.999999 (essentially approximates to ``no failure should happen''). In this case, the interval should be more precise, which can be achieved by lowering $\alpha$ and $\beta$ of the hypothesis testing, which are the probabiliy of accepting false positive (i.e., accepting H$_0$ when H$_1$ holds) and false negative (i.e., accepting $H_1$ when $H_0$ holds), respectively~\cite{David2011StatisticalAutomata}.
\begin{figure}[h]
	\centering
	\includegraphics[width=0.8\linewidth]{images/smc_result}
	\caption{BBW SMC result snippet from UPPAAL SMC.}
	\label{fig_smcresult}
\end{figure}

\section{Contr.~4: ILP-based Software Allocation}\label{rc_ilp}
Safety-critical systems are usually resource constrained, that is, the execution hardware platform has limited computational resources and power/energy sources. Therefore, the system should be designed efficiently especially critical system resources such as power in order to accommodate complex safety-critical software functionality. In this section, we propose exact method of mapping software to hardware in order to optimize the total power consumption of a distributed safety-critical software via ILP, which is needed when highly-efficient design is needed especially small-scale sofware applications, e.g., software components not exceeding 10.

\subsection*{System Model}
We consider an AUTOSAR distributed safety-critical software that can be mapped to multiple computing nodes (or units), which are heterogeneous computing systems with respect to power specification, failure rate and processing speed. The software has end-to-end timing and reliability requirements, therefore, the mapping must satisfy the requirements. In order to meet the reliability requirement, the mapping applies fault tolerance by replicating software components subsequently mapping the replicas on different computing units. Moreover, the task and messages that realize the safety-critical software functionality are scheduled using a fixed-priority preemptive scheduling policy and fixed-priority non-preemptive scheduling policy (over a  CAN bus), respectively.

The AUTOSAR software functionality is implemented by Runnables that communicate over the virtual function bus (VFB), which abstracts the details of the run-time environment. We assume the runnables are triggered periodically and have multiple worst-case execution times as they can executed on processors that have different processing speed.
\begin{figure}[h]
	\centering
	\includegraphics[width=0.8\linewidth]{images/system_model.pdf}
	\caption{Computation time of the various algorithms for solving different instances of the software allocation problem.}
	\label{fig_allocationtime_ilp_metaheuristic}\vspace{-0.4cm}
\end{figure}

\subsection*{ILP Model}
Consider that the mapping solution is represented by a vector of binary matrices $\textbf{x}=\{\textbf{x}_1:1,...,K\}$, where \ttxkij represents the mapping of the software component $\sss{q}[k][ij]$ to the computing node $\ssb{n}[j]\in \mathcal{N}$. 
\begin{equation}
\label{eqn_solution}
\bspx{k}=
\begin{bmatrix} 
\ssx{k}{11} & \ssx{k}{12} & \dots & \ssx{k}{1K}\\
\ssx{k}{21} & \ssx{k}{22} & \dots & \ssx{k}{2K}\\
\vdots & \vdots & \ddots & \vdots\\
\ssx{k}{N1} & \ssx{k}{N2} & \cdots & \ssx{k}{NK}
\end{bmatrix}
\end{equation}
The power consumption of a software application $\{c_i:i=1,...,n_C\}$ that is mapped to computing units $N'\in N$ is calculated as the sum of the power consumption of each node $n_h\in N'$, thus $\mathcal{P}_{total}(x)=\sum_{n\in N'}{\mathcal{P}(u_n(x))}$, where $\mathcal{P}(u)$ is the power consumption of a computing node which linearly proportional to its utilization~\cite{Mahmud5222}. The utilization of a computing node is the sum of the utilization of its constituent tasks that realize corresponding software components.
\begin{align}
	(u_1,...,u_K)(\x)=\sum_{k=1}^{K}\sum_{\tau\in T_{c_i}}{\xkij*\frac{WCET_\tau}{P_\tau}},
\end{align}
where $T_{c_i}$ is the set of tasks realizing the software component $c_i$, $WCET_\tau$ and $P_\tau$ are the worst-case execution time and  period of the task $\tau$.

\subsubsection{Tasks Deadline Constraint}
The mapping solution must satisfy the timing and reliability requirements imposed on the safety-critiacal application. We assume the taskset in each computing node is scheduled using a fixed-priority preemptive scheduler~\cite{Sha2004RealPerspective} and we check the schedulability via the classical worst-case response time analysis~\cite{Baruah2011Response-timeSystems}. Since the time analysis model involves recursion which is difficult to model as a linear model. Therefore, we construct linear logical constraints $\xi$ that corresponds to the valid tasksets that can potentially partitioned to computing nodes, and these tasksets are schedulable.

The logical system is constructed as follows: first we identify the possible partitions of the software components, $P=2^C$, then we check the schedulability of each partition and catagorize them per node, i.e., $Y_j=\{p\in P| isSched(p,n_j)\}$, where $sched$ is a boolean function that returns true if $p$ is schedulable on $n_j$.  A partition is identified by a sequence of 0s and 1s based on its costituent components, e.g., the id of the partition $\{c_1,c_4,c_5\}\subseteq \{c_1,c_2,c_3,c_4,c_5\}$ is $10011=19$. So, given the mapping $\x$, the partition id that is mapped to the computing node $n_j$ can be computed as
\begin{align}
	g_j(\x)=\sum_{i=1}^{N}{10^{N-i}\max_{1\leq k< K}\xkij}
\end{align}
where the $\max_{1\leq k< K}$ function returns 1 if at least one component replica is mapped $n_j$, otherwise returns 0, indicating no replica is mapped to that node. Thus, each node must have a valid partition that is schedulable as indicated in the logical constraint,
\begin{align}
\bigvee\limits_{p\in Y_j}g_j (\x)= id(p) ,
\end{align}
where $id(p)$ is a predicate that returns the id of the partition $p$.

\subsubsection{End-to-end Timing Constraint}
The age delay of a chain is computed according to the semantics discussed in~\cite{Feiertag2009ASemantics}\cite{mubeen2013support}. The response-time logical assertions ensure that every task in the distributed system are schedulable for a mapping \ttx. Therefore, we can safely check if the age delay of each chain satisfy its end-to-end timing requirement without the need to check the chain's constituent tasks are schedulable or not. Similar to the previous approach, we construct logical constraints $\eta$ that correspond to the valid set of mappings of chains on the network of computing nodes.  Given $\Gamma =\{\Gamma_i:i=1,...,N\}$ of chains, the set of possible mappings of each chain $\Gamma_i$ on a set of $\mathcal{N}^\Gamma$. Thus the set of valid mappings of each chain satisfy the end-to-end timing requirement is $Z_i=\{\gamma\in \mathcal{N}^\Gamma| ageDelay(\gamma) < EE_\gamma$.

\subsubsection{Software Application Reliability Constraint}
The reliability constraint refers to the probability that the application functions by the time $t$, i.e $[0,t]$~\cite{Goel1985SoftwareApplicability}. We assume that the reliability requirement of a software application is 0.99999999 over a 2-year operation time, which is to say  that the safety-critical system almost should not fail within the stated period. Furthermore, we assume that the mean-time to failure of the computing nodes is given as $10^6$ hours (0.0000001 per hour), which is the usual practice in many functional safety design.

Unless the the software application is replicated on one or more computing nodes, the reliability requirement is usually unattainable. So, the allocation strategy is basically to replicate sufficient software components to meet the requirement at the same time it should not exceed the limit so that the application consumes less power. However, the reliability calculation is not trivial as the replication results in functional inter-dependency of the computing nodes, in this case,  the series-parallel method does not apply. For this reason, we propose an exact method based on state enumeration~\cite{ExactMethodstoComputeNetworkRe- liability}, which basically enumerates all the possible configurations of the nodes  $PS$. Subsequently, the reliability of the application becomes the total probability of the configurations that enable the functioning of the application~\cite{Mahmud5222}.
\begin{definition}[Software Application Failure]
	A software application fails in the configuration $s\in\xi$ if there exists a component type $c_i$ where all of its replicas $Q_i$ \textit{fail}, otherwise, it functions, as shown in Equation (\ref{eqn_appreliability_milp_components}).  The component replica $q_i,j\in Q_i$ of type $c_i$ fails if $n_h$ fails, that is $s_h=0$.
\end{definition}
\begin{align}
f_s(x) = \floor[\Bigg]{\frac{\sum_if_{c_i}(x,s)}{N}}=
\begin{cases}
1 & \mbox{if } application \mbox{ functions}\\
0 & \mbox{if } application \mbox{ fails}
\end{cases}\label{eqn_appreliability_milp_components}
\end{align}

Thus, the reliability of the software allocation is
\begin{align}
\label{eqn_appreliability_milp}
Reliability(\x)=\sum_{s\in PS}f_s(\x)*p_s,
\end{align}
where $p_s$ is the probability that the computing nodes attain a configuration $s$ which is computed as $p_s=\prod_{m\in M}\big[(z_m*\lambda_m) + (1-z_m)*(1-\lambda_m)\big]$, where $z_m\in\{0,1\}$ indicates if the node fails (i.e., 0) or functions (i.e., 1), $\lambda_m$ is the failure rate of the node $m$ and $1-\lambda_m$ is the opposite of failure rate, i.e., that is the rate that it does not fail.
\subsubsection{ILP Problem}
The goal of the ILP optimization is to minimize the total power consumption, which is the objective function, and the constraints Equation (\ref{lbl_deadline_constraint}), (\ref{lbl_e2e_constraint}) and (\ref{lbl_reliability_constraint})ensure that the optimization satisfies the tasks deadline, end-to-end timing and reliability constraints, respectively.
\begin{align}
\label{eqn_const_func}
\min_{\x\in X} \mathcal{P}_{total}(\x) && \text{ Subjected to:}\\
\label{lbl_deadline_constraint} 
\bigvee\limits_{p\in Y_j} g_j (\x) &= id(p)& \mbox{ for all } j=1,..,|\mathcal{N}|\\ 
\label{lbl_e2e_constraint}
\bigvee\limits_{\gamma\in Z_j} h_j (\x) &= id(\gamma) & \mbox{ for all } j=1,..,|\Gamma|\\ 
\label{lbl_reliability_constraint}
\sum_{s\in PS}f_s(\x)*p_s &\leq \RL&,
\end{align}

\subsection*{Tool Support and Validation on Automotive Benchmark}
Our proposed ILP approach is validated on different size of software applications, i.e., applications with variable number of software components, cause-effect chains and degree of replication. The applications are synthesized according to the automotive benchmark proposed by Kramer et al.~\cite{Kramer2015RealFree}, which discusses a typical complexity of an AUTOSAR engine management system in terms of, e.g., the number and execution time of runnables, and the number and activation patterns of cause-effect chains. 

We prepared 6 software applications and an execution platform that consists 8 nodes. The smallest application consist of 4 components and the largest 10 components. Furthermore, the cause-effect chains varies from 10 to 60 and activation patters ranges from 2 to 4. Figure~\ref{fig_ilp_results} shows the result of the optimization from the CPLEX solver, that is, the power consumption and the computation time, respectively. In general, the optimization maps components to nodes with lower power specification and higher processor speed, however, this is not always the case since the end-to-end timing and reliability constraints can dictate the selection of different nodes, which consume more power as a result. Figure~\ref{fig_computationtime_ilp} shows a slow increase of computation time from until the application with 8 number of components, which took  6.06 sec, and rapidly increased to 30.3 sec and 129.4 sec allocating application with 9 and 10 components, respectively. The optimization took extremely large amount of time for applications that consist of more than 10 components, and more than 15 component the waiting time was intractable, thus the optimization was interrupted manually. 
\begin{figure}[h] 
	\centering
	\subfloat[Power consumption of nodes.\label{fig_powerconsumption_ilp}]{% 
		\includegraphics[width=0.45\textwidth]{images/power}
	} ~
	\subfloat[Allocation times software applications.\label{fig_computationtime_ilp}]{% 
		\includegraphics[width=0.55\textwidth]{images/increasing_components} 
	} 
	\caption{ILP optimization of different sizes of software applications based on the number of software components, cause-effect chains, computing nodes.} \label{fig_ilp_results}
\end{figure}

\section{Contr.~5: Metaheuristics-based Software Allocation}\label{rc_pso}
In the previous contribution, we observe that the ILP approach does not scale well to the large software allocation problems. To address this problem, in this section, we propose approximation algorithms based on metahuristics, which is ``high-level problem-independent algorithmic framework that provides a set of guidelines or strategies to develop heuristic optimization algorithms''. Meta-heuristic algorithms does not guarantee optimality of solutions, nevertheless, the solutions can be deemed satisfactory. In the case of minimizing power consumption, what is more appealing to system designers is usually the benefit attained as a result of the power consumption reduction, e.g., accommodating more and more software applications, improving battery-life, rather than optimality. In this spirit, metahuristics can be useful to solve complex and large optimization problems, yet in reasonable time, as compared to exact methods, e.g., branch-and-bound, dynamic programming.

\subsection*{Hybrid Particle-swarm Optimization}
The canonical \pso{} technique uses the constriction factors to balance exploitation and exploration of the search space to get closer to the global optima, hence improving solution quality. Nevertheless, it still suffers from premature convergence or local minima especially when applied on complex and large problems~\cite{Rini2011ParticleChallenges}. Its hybridization is proven to perform better in many cases~\cite{Sengupta2018ParticlePerspectives}. In particular, it is shown to perform better in the tasks assignment problem, that is when hybridized with, e.g., the genetic algorithm~\cite{Sailer2013OptimizingAUTOSAR}, the hill-climbing~\cite{yin2007task}, simulated annealing~\cite{Zhao2007ASystem}, differential evolution~\cite{Storn1997DifferentialSpaces}. As compared to the hybridization with genetic, the hybridization with hill-climbing \hcpso{} is shown to perform better by Yin et al.~\cite{yin2007task} for the tasks allocation problem to maximize reliability of distributed systems. 

In this work, we apply \hcpso{} to the problem at hand, and to tackle its stagnation when applied to large problems. Moreover, we hybridize \pso{} with the differential evolution technique, \depso{}, to improve diversification by applying the mutation and cross-over operators of the differential evolution. Algorithm~\ref{alg_depso} show the pseudocode of the hybrid \pso{}. Line 3 and 4 compute the personal best and the swarm best solutions, respectively. For each particle in the swarm, the velocity and position is computed in Lines~5-8. Lines~9-13 apply the hybridization based on the choice of the algorithm, i.e., \de, \hcpso{} and \shpso{} intermittently, i.e., whenever the interval criterion condition is met.
\IncMargin{1em}
\begin{algorithm}[H]
	\SetAlgoLined
	\SetKwData{P}{P}\SetKwData{S}{sBest}
	\SetKwData{Generation}{Generation}
	\SetKwData{Interval}{Interval}
	\SetKwData{Particles}{Particles}
	\SetKwInOut{Input}{input}\SetKwInOut{Output}{output}
	\SetKwFunction{OptimizeUsingDE}{optimizeUsingDE}
	\SetKwFunction{OptimizeUsingHC}{optimizeUsingHC}
	\SetKwFunction{OptimizeUsingSHC}{optimizeUsingSHC}
	\SetKwFunction{ComputeParticleVelocity}{computeParticleVelocity}
	\SetKwFunction{ComputeParticlePosition}{computeParticlePosition}
	\SetKwFunction{InitPSO}{initPSO}
	\SetKwFunction{ComputePersonalBest}{ComputePersonalBest}
	\SetKwFunction{ComputeSwarmBest}{ComputeSwarmBest}
	
	\BlankLine
	\Input{\pso parameters, \de parameters}
	\Output{Software allocation solution \S .\textbf{x}}
	\BlankLine
	\Particles $P$ $\leftarrow$ \InitPSO{}\;
	\BlankLine
	\While{termination criteria}{
		$\textbf{p}_{bst}$ $\leftarrow$\ComputePersonalBest{$P$}\;
		$\textbf{z}\leftarrow$\ComputeSwarmBest{$P$}\;
		\BlankLine
		\ForEach{$p\in P$}{
			\ComputeParticleVelocity{$p$} according to Equation (\ref{eqn_pso_velocity})\;
			\ComputeParticlePosition{$p$} according to Equation (\ref{eqn_pso_position})\;
		}
		\If{interval criteria}{
			$P$ $\leftarrow$ \OptimizeUsingDE{$P$}\;
			\tcp{$P$ $\leftarrow$ \OptimizeUsingHC{$P$}}\label{hc}
			\tcp{$P$ $\leftarrow$ \OptimizeUsingSHC{$P$}}\label{sh}
		}
	}
	\caption{Hybrid \pso{} Pseudocode.}
	\label{alg_depso}
\end{algorithm}\DecMargin{1em}

\subsection*{Validation on the Bosch EMS Benchmark} 
In this section, we evaluate our proposed hybrid \pso{} algorithms for the allocation of software applications to heterogeneous computing units. The algorithms are evaluated against different specifications of automotive software applications and execution platforms with regard to effectiveness, stability and scalability. The software-application specifications consist of the number of software components $c$, runnables $r$, tasks $t$ and cause-effect chains $g$.  The specifications are synthesized from the automotive benchmark proposed by Kramel et al.~\cite{Kramer2015RealFree}. The benchmark indicates a strong correlation between runnables and cause-effect chains in terms of timing and activation patterns. It shows the timing specifications of runnables and their shares in an engine management system. Moreover, it shows the activation patterns of cause-effect chains, the runnables per activation and their shares in the system. The engine management system is one of the most complex automotive systems in the vehicular electrical/electronic execution platform. 

\noindent\\ \textbf{Result:} We conducted two experiments: i) the first experiment compares the algorithms in terms of quality of solutions, computational time, stability; ii) the second experiment evaluates the effect of applying the approximation algorithm on the overhead of the end-to-end delays calculations due to the replication overhead.

In the first experiment, as illustrated in Figure~\ref{fig_powerconsumption_ilp_metaheuristic}, we show that \ilp returned optimal solutions in the first three problems, likewise all except \pso{} and \depso{}. However, as the problem size increases, the hybrid PSO with the local search algorithms delivered better solutions as compared to \depso{}. 
\begin{figure}
	\centering
	\includegraphics[width=0.8\linewidth]{images/power_consumption.pdf}
	\caption{(Near) Optimal Power Consumption of the Different Software Allocation Problems.}
	\label{fig_powerconsumption_ilp_metaheuristic}\vspace{-0.4cm}
\end{figure}
Futher, we showed that \shpso{} scalled better than \hcpso{} and its results computation time became exponential as shown in Figure~\ref{fig_allocationtime_ilp_metaheuristic} and did not return solutions. In contrast, the hill-climbing-\pso{} hybrid performed the best in all last three problems except \hcpso{} failed to return near optimal solutions to the largest problem \pb{80}{20}{60}. The result also shows that the stochastic version of the hill-climbing-PSO hybrid \shpso{} scales better the over quality of the solution can deemed acceptable as compared to the \ilp{} in the first three problems and \hcpso{} in the next two problems. 
\begin{figure}[h]
	\centering
	\includegraphics[width=0.8\linewidth]{images/time_summary.pdf}
	\caption{Computation time of the various algorithms for solving different instances of the software allocation problem.}
	\label{fig_allocationtime_ilp_metaheuristic}\vspace{-0.4cm}
\end{figure}

Due to the replication of components to meet the reliability goals, the overhead of computing the delays is exponential. Figure~\ref{fig_chainsreplicationimprovements} compares the effect of with and without applying the approximation algorithm.  It shows that $61\%-81\%$ computational time improvement over the exact method while facing quality degradation only for samples $g_{30}d_{3}$ and  $g_{60}d_{2}$. The improvements are in seconds, which means for a single usage (or run) of the meta-heuristic optimization algorithms, it is not significant. However, considering practical engineering process, which requires several iterations, the commutative effect of appliying the approaximation algorithm can beneficial in terms of responsiveness to engineers. 
\begin{figure}[h]
	\centering
	\includegraphics[width=0.7\linewidth]{images/chains_replication_improvements}
	\caption{Effect of approximate algorithm over delay calculations with replication.}
	\label{fig_chainsreplicationimprovements}
\end{figure}

\section{Paper Contributions}
Tables~\ref{paper_contribution} shows the peer-reviewed papers and their contribution to the thesis.
% Please add the following required packages to your document preamble:
% \usepackage{booktabs}
\begin{table}[h]
	\begin{tabular}{@{}llllll@{}}
		\toprule
		Paper & Sec\ref{rc_resa} & Sec\ref{rc_resaanalysis}& Sec\ref{rc_sim} & Sec\ref{rc_ilp} & Sec\ref{rc_pso}\\ \midrule
		A & $\times$ &  &  &  &\\
		B & $\times$ & $\times$ &  & & \\
		C &  & $\times$ &  & & \\
		D &  &  & $\times$ &  &\\
		E &  &  &  & $\times$ &\\
		F &  &  &  & $\times$ &$\times$ \\ \bottomrule
	\end{tabular}
\caption{List of the peer-reviewed papers and their contribution to the thesis.}\label{paper_contribution}
\end{table}
%\chapter{List of Publications}\label{papersincl}
The papers to be included in the PhD thesis are listed as follows.

\subsection*{Paper A}
ReSA: An ontology-based requirement specification language tailored to automotive systems. Nesredin~Mahmud, Cristina~Seceleanu and Ljungkrantz~Oscar.\textit{In the 10th IEEE International Symposium on Industrial Embedded Systems (SIES)(pp. 1-10). IEEE}.\label{lbl_resa}\\[6pt]
\textbf{Abstract:} \textit{Automotive systems are developed using multi-leveled architectural abstractions in an attempt to manage the increasing complexity and criticality of automotive functions. Consequently, well-structured and unambiguously specified requirements are needed on all levels of abstraction, in order to enable early detection of possible design errors. However, automotive industry often relies on requirements specified in ambiguous natural language, sometimes in large and incomprehensible documents. Semi-formal requirements specification approaches (e.g., requirement boilerplates, pattern-based specifications, etc.) aim to reduce requirements ambiguity, without altering their readability and expressiveness. Nevertheless, such approaches do not offer support for specifying requirements in terms of multi-leveled architectural concepts, nor do they provide means for early-stage rigorous analysis of the specified requirements. In this paper, we propose a language, called ReSA, which allows requirements specification at various levels of abstraction, modeled in the architectural language of EAST-ADL. ReSA uses an automotive systems' ontology that offers typing and syntactic axioms for the specification. Besides enforcing structure and more rigor in specifying requirements, our approach enables checking refinement as well as consistency of requirements, by proving ordinary boolean implications. To illustrate ReSA's applicability, we show how to specify some requirements of the Adjustable Speed Limiter, which is a complex, safety-critical Volvo Trucks user function.}\\[6pt]
\textbf{Personal Contributions: }I was the main driver of the paper. I developed the ReSA language including its syntax and semantics, and Cristina Seceleanu proposed a consistency analysis technique besides giving useful comments and ideas on the design of the language. Oscar Ljungkrantz provided useful materials from VGTT that were eventually analyzed for the language development, and gave feedback on the language design and implementation from an industrial viewpoint.\\
\textbf{Status:} Published

\subsection*{Paper B}
ReSA tool: Structured requirements specification and SAT-based consistency-checking. Nesredin~Mahmud, Cristina~Seceleanu and Ljungkrantz~Oscar. \textit{In the 2016 Federated Conference on Computer Science and Information Systems (FedCSIS)(pp. 1737-1746). IEEE}.\label{lbl_resatool}\\[6pt]
\textbf{Abstract:} \textit{Most industrial embedded systems requirements are
	specified in natural language, hence they can sometimes be
		ambiguous and error-prone. Moreover, employing an early-stage
		model-based incremental system development using multiple
		levels of abstraction, for instance via architectural languages
		such as EAST-ADL, calls for different granularity requirements
		specifications described with abstraction-specific concepts that
		reflect the respective abstraction level effectively.
		In this paper, we propose a toolchain for structured requirements
		specification in the ReSA language, which scales to multiple
		EAST-ADL levels of abstraction. Furthermore, we introduce
		a consistency function that is seamlessly integrated into the
		specification toolchain, for the automatic analysis of requirements
		logical consistency prior to their temporal logic formalization
		for full formal verification. The consistency check subsumes
		two parts: (i) transforming ReSA requirements specification into
		boolean expressions, and (ii) checking the consistency of the
		resulting boolean expressions by solving the satisfiability of their
		conjunction with the Z3 SMT solver. For validation, we apply
		the ReSA toolchain on an industrial vehicle speed control system,
		namely the Adjustable Speed Limiter.}\\[6pt]%abstract
	\textbf{Personal Contributions: }I was the main driver of the paper. I developed the ReSA toolchain that consists of the editor and the consistency checker including the integration with the Z3 SAT solver in the backend. Cristina Seceleanu formulated the consistency checking and together with Oscar Ljungkrantz, they contributed to the paper with useful comments and ideas.\\
		\textbf{Status: }Published

\subsection*{Paper C}
	Specification and semantic analysis of embedded systems requirements: From description logic to temporal logic. Nesredin~Mahmud, Cristina~Seceleanu and Ljungkrantz~Oscar. \textit{In the International Conference on Software Engineering and Formal Methods (pp. 332-348). Springer, Cham.} Acceptance rate: 26\%.\label{lbl_resadl}\\[6pt]%authors
	\textbf{Abstract:} \textit{Due to the increasing complexity of embedded systems, early detection of software/hardware errors has become desirable. In this context, effective yet flexible specification methods that support rigorous analysis of embedded systems requirements are needed. Current specification methods such as pattern-based, boilerplates normally lack meta-models for extensibility and flexibility. In contrast, formal specification languages, like temporal logic, Z, etc., enable rigorous analysis, however, they usually are too mathematical and difficult to comprehend by average software engineers. In this paper, we propose a specification representation of requirements, which considers thematic roles and domain knowledge, enabling deep semantic analysis. The specification is complemented by our constrained natural language specification framework, ReSA, which acts as the interface to the representation. The representation that we propose is encoded in description logic, which is a decidable and computationally-tractable ontology language. By employing the ontology reasoner, Hermit, we check for consistency and completeness of requirements. Moreover, we propose an automatic transformation of the ontology-based specifications into Timed Computation Tree Logic formulas, to be used further in model checking embedded systems.}\\[6pt]%abstract
	\textbf{Personal Contributions:} I was the main driver of the language. I developed the ReSA language semantics using event-base approach, which is encoded in description logic. Cristina~Seceleanu and Ljungkrantz~Oscar provided with useful ideas and comments.\\
	\textbf{Status:} Published

\subsection*{Paper D}
SIMPPAAL - A Framework For Statistical Model Checking of Industrial Simulink Models. Predrag~Filipovikj, Nesredin~Mahmud, Raluca~Marinescu, Cristina~Seceleanu, Oscar~Ljungkrantz and Henrik~L\"{o}nn. \textit{Submitted to ACM Transaction on Software Engineering and Methodology (TOSEM). ACM Journals.}\label{lbl_simulink_ilp}\\[6pt]%authors
	\noindent \textbf{Abstract:} \textit{The evolution of automotive systems has been rapid. Nowadays, electronic brains control dozens of functions in vehicles, like
		braking, cruising, etc. Model-based design approaches, in environments such as MATLAB Simulink, seem to help in addressing
		the ever-increasing need to enhance quality, and manage complexity, by supporting functional design from a set of block
		libraries, which can be simulated and analyzed for hidden errors, but also used for code generation. For this reason, providing
		assurance that Simulink models fulfill given functional and timing requirements is desirable. In this paper, we propose a
		pattern-based, execution-order preserving automatic transformation of atomic and composite Simulink blocks into stochastic
		timed automata that can then be formally analyzed with Uppaal Statistical Model Checker (Uppaal SMC). To enable this, we
		first define the formal syntax and semantics of Simulink blocks and their composition, and show that the transformation is
		provably correct for a certain class of Simulink models. Our method is supported by the SIMPPAAL tool, which we introduce
		and apply on two industrial Simulink models, a prototype called the Brake-by-Wire and an operational Adjustable Speed
		Limiter system. This work enables the formal analysis of industrial Simulink models, by automatically generating stochastic
		timed automata counterparts.}\\[6pt]%abstract
	\textbf{Personal Contributions: } The three co-authors contributed equally to writing the paper. Technically, I equally contributed with proposing the pattern-based semantics of Simulink blocks, together with Predrag Filipovikj. I introduced a mechanism to enforce the execution order of the blocks using inter-arrival times. Predrag implemented the flattening algorithm and the tool for the automatic transformation of Simulink models into a network of timed automata with stochastic semantics. Raluca Marinescu contributed with analyzing the BBW system, Cristina Seceleanu contributed with defining the methodology, and with useful ideas and comments. Guillermo Rodriguez-Navas wrote the related work section. The industrial coauthors provided the use cases and commented on the final draft.\\
	\noindent\textbf{Status:} Under Review

\subsection*{Paper E}
Power-aware Allocation of Fault-tolerant Multirate AUTOSAR Applications. 
Nesredin~Mahmud, Guillermo~Rodriguez-Navas, Hamid~Faragardi,Saad~Mubeen and Cristina~Seceleanu. \textit{In the 25th Asia-Pacific Software Engineering Conference (APSEC'18). IEEE.} \label{lbl_resa}. \label{lbl_softwareallocation_ilp}
\\[6pt]%authors
\textbf{Abstract:} \textit{The growing complexity of automotive functionality has attracted revolutionary computing architectures such as mixed-criticality design, which enables effective consolidation of software applications with different criticality on a shared execution platform. Mixed-critical design that is required to satisfy end-to-end timing and reliability specifications should consider power-efficient software design in order to accommodate more and more functionality. Due to the recursive and exhaustive nature of the real-time and reliability analysis, exact methods, e.g., branch and bound, dynamic programming, are prohibitively expensive. We propose hybrid particle-swarm optimization algorithms based on differential evolution and hill-climbing algorithms to minimize power consumption of the safety-critical software, which have end-to-end timing and reliability requirements, on a network of heterogeneous computing units. The optimization approach employs fault tolerance to maximize reliability of the software applications subsequently meet the reliability requirements. Our proposed integrated software-allocation approach is evaluated using a range of synthetic software applications based a real-world automotive benchmark. The evaluation makes comparative analysis of the differential evolution, particle-swarm optimization, integer-linear programming and hybrid particle-swarm optimization algorithms. The results show that the hybrid algorithms based on the hill-climbing algorithms outperform the rest of the meta-heuristic algorithms, in particular, the stochastic version of the hill-climbing algorithm scales well in large software allocation optimization problems while its overall optimality performance can be deemed acceptable.
}\\[6pt]%abstract
\textbf{My Contributions: } I am the main driver of the paper. I developed the system model (including the power consumption, timing, reliability models) and further refined by the co-authors. Hamid~Faragardi and I developed the ILP model, and I implemented the ILP problem (including the system model) and collected experimental results. The co-authors gave useful ideas and comments on respected parts of the paper: Guillermo~Rodriguez-Navas on reliability modeling, Hamid~Faragardi on optimization and related work, Saad~Mubeen on the timing analysis, and Cristina~Seceleanu gave comments and ideas on the main contributions of the paper, including on the optimization objective and constraints.\\%my contribution
\textbf{Status:} Published

\subsection*{Paper F}
Optimized Allocation of Fault-tolerant Embedded Software with End-to-end Timing Constraints\\[6pt]
Nesredin~Mahmud, Guillermo~Rodriguez-Navas, Hamid~Faragardi,Saad~Mubeen and Cristina~Seceleanu. Optimized Allocation of Fault-tolerant EmbeddedSoftware with End-to-end Timing Constraints. \textit{M\"{a}lardalen Real-time Research Center Technical Report (MRTC)}. Submitted to Elsevier JSA Journal. \label{lbl_softwareallocation_pso}\\[6pt]%authors
\textbf{Abstract:} \textit{The growing complexity of automotive functionality has attracted revolutionary computing architectures such as mixed-criticality design, which enables effective consolidation of software applications with different criticality on a shared execution platform. Mixed-critical design that is required to satisfy end-to-end timing and reliability specifications should consider power-efficient software design in order to accommodate more and more functionality. Due to the recursive and exhaustive nature of the real-time and reliability analysis, exact methods, e.g., branch and bound, dynamic programming, are prohibitively expensive. We propose hybrid particle-swarm optimization algorithms based on differential evolution and hill-climbing algorithms to minimize power consumption of the safety-critical software, which have end-to-end timing and reliability requirements, on a network of heterogeneous computing units. The optimization approach employs fault tolerance to maximize reliability of the software applications subsequently meet the reliability requirements. Our proposed integrated software-allocation approach is evaluated using a range of synthetic software applications based a real-world automotive benchmark. The evaluation makes comparative analysis of the differential evolution, particle-swarm optimization, integer-linear programming and hybrid particle-swarm optimization algorithms. The results show that the hybrid algorithms based on the hill-climbing algorithms outperform the rest of the meta-heuristic algorithms, in particular, the stochastic version of the hill-climbing algorithm scales well in large software allocation optimization problems while its overall optimality performance can be deemed acceptable.
}\\[6pt]%abstract
\textbf{Personal Contributions: } I am the main driver of the paper. I developed the system model, the metaheuristic formulation, implemented the problem in Java, and collected and analyzed the experimental results. The co-authors gave writing update, useful ideas and comments on respected parts of the paper: Guillermo~Rodriguez-Navas on reliability modeling, Hamid~Faragardi on optimization, Saad~Mubeen on the timing analysis, and Cristina~Seceleanu gave idea on the main objective and constraints. \\%my contribution
\textbf{Status:} Submitted to Journal of System Architecture (JSA), Elsevier Journals.\\
%\subsection*{Paper F}
%\begin{tabular}{p{\textwidth}}
%	\noindent \textbf{SIMPPAAL meets ReSA: From Automated Requirements Specifications to Automated Formal Analysis of Simulink Models.}\\%titile
%	\noindent [authors part]\\[6pt]%authors
%	\noindent \textbf{Abstract:} \textit{The main objective of this work is to extend the validation of our contributions, that is the SIMPPAAL method and tool \cite{Filipovikj4714} for the verification of Simulink models based on SMC, for which the properties will be automatically generated by employing the structured requirements specification framework called ReSA \cite{resatool}. The verification will be applied on a set of Simulink models that are part of the Adjustable Speed Limiter system, which is an operational system installed in all Volvo Trucks.}\\[6pt]%abstract
%	\textbf{Status: }Candidate venue for submission: \textit{``NFM 2019: 11\textsuperscript{th} Annual NASA Formal Methods Symposium"} Submission deadline: 2018-12-14. Notification Date: 2019-02-22.\\%status
%	\textbf{In Case of Rejection: }SEKE 2019: International Conference on Software Engineering and Knowledge Engineering. Submission Deadline: 2019-03-01. Notification date: 2019-04-20.\\%my contribution
%\end{tabular}

    \input{Chapters/ResearchMethod.tex}
    \chapter{Related Work}
In this section are discussed the related work on specification and analysis of requirements, formal analysis Simulink models and allocation of software applications,

\section{Specification and Analysis of Embedded Systems Requirements}
Embedded system requirements are captured in different representations, e.g., textual, tabular, graphical, etc. The textual representation, which is the scope of the analysis, can be conveniently classified into two classes: i) controlled natural language (CNL), ii) template-based methods.  The syntax and semantics of the CNLs are similar to natural language except that the lexicon and the syntax are restricted for different reasons, of which improving comprehensibility of the text and formal representations to support rigorous analysis of the text are prominent. The \resa{} language is designed to improve comprehensibility of requirements specifications as well as to be computer-processable. There are many computer-processable CNL in literature \cite{Kuhn2014ALanguages}, e.g., Attempto Controlled English (ACE) \cite{attempto96}, Processable English (PENG) \cite{Schwitter2002EnglishLanguage}, etc. Similar to most computer processable languages, \resa{} has limited syntactic constructions, and allows knowledge representations, in contrast though, our language caters for embedded systems and therefore it uses concepts as well as semantic rules that are domain-specific to embedded systems. Similar to PENG, its implementation supports look-ahead in order to enable predictive and guided specification.

The template-based methods, in particular requirements boilerplate uses templates (or boilerplates), which are reusable, recurrent patterns to specify requirements, e.g., CESAR boilerplates \cite{Farfeleder2011DODT:Development}, RAT, etc. The main drawback of existing requirements boilerplates are: i) the templates are usually too limited therefore not expressive enough, ii) it is not easy to find the appropriate boilerplate during specification. In this regard, \resa{} extends boilerplates with a meta-model that guides a plausible instantiation of boilerplates.


\section{Formal Analysis of Simulink Models}
Several research endeavors have tackled the problem of formally analyzing Simulink models in order to gain better insight into the design of Simulink models, and they mainly differ in their coverage of the Simulink language, their robustness to formally analyze complex Simulink models, which is crucial for applicability of the proposed methods in industry. In this related work, we focus on techniques that employ model checking due to extra benefit of automation, such as over theorem proving, SAT solving.% on formal analysis of Simulink models, existing formalisms and their applicability in industry are discussed, and also compared to our solutions.

Simulink already supports formal analysis of Simulink models via its Simulink Design Verifier (SDV) product\footnote{https://se.mathworks.com/products/sldesignverifier.html}. Although there are limited resources that investigate the pros and cons of the product, the study by Nellen et al. \cite{Nellen2018FormalRecommendations} and Florian et al.~\cite{Leitner2008SimulinkStudy} indicates some limitation of the product, e.g., its inability to specify timed properties (e.g., cause-response LTL properties), degraded performance and less scalability against the SPIN model checker, inability to detect problems caused by race conditions in concurrent processes. 

The PlasmaLab proposed by Axel Legay et al.~\cite{Legay2016StatisticalLab} transforms Simulink sample traces into statistical models, which are eventually analyzed by their statistical mode checker. Although the checker is assisted by an algorithm that determines sufficiency of the sample traces, it is unclear how it works. Unlike many approaches, PlasmaLab can analyze any Simulink models as long as the simulation results sample trances, which is its main advantage. %Ferrante et al. \cite{Hocking2016ProvingModels} use contract-based theory in order to lift the block specification, and rely on a combination of SAT solvers and the NuSMV model checker for analysis. Hocking et al. \cite{Ferrante2012ParallelSystems} use the PVS specification language for writing the specification, and rely on the PVS theorem prover for analysis. A limitation of this strategy is that both steps still require much user interaction, so it is error-prone and requires certain understanding of the formal analysis engines, which is not common among embedded systems engineers.

%The model-to-model transformation approach basically refers to the transformation of the Simulink model into a formal model that can be checked via model checking, e.g., for reachability properties.
Other related work relay on exact model checking as opposed to statistical, e.g., Meenakshi et al. \cite{Meenakshi2006ToolChecker} propose a transformation into the input language of NuSMV model checker and supports only discrete Simulink blocks. Similarly, Bernat et al.~\cite{Barnat2012ToolDesigns} propose a transformation into an intermediate language, followed by compilation of the latter into Common Explicit-State Model Interface (CESMI) specification, which is an input language to the \devine{} model checker. These related work are limited in scalability due the use of exact model checking, thus cannot be use on complex and large-scale Simulink models. Some research endeavors propose transformations of just StateFlow/Simulink into Baysian statistical model~\cite{Zuliani2010BayesianVerification}, hybrid automata~\cite{ManamcheriSukumar2011TranslationAutomata}, the input language of \uppaal{}~\cite{Jiang2016FromDesign}. 

In contrast to the discussed related work, our approach is distinct mainly in three ways: i) we transform the Simulink model into networks of stochastic timed automata, which is checked via the statistical model checking, as opposed to exact model checking, using \uppaalsmc; ii) we verify the routines of the Simulink blocks, which are implemented in C, using Dafny (a program verifier) before the transformation; iii) our approach supports transformations of any type of Simulink blocks, and discrete and continuous Simulink models.


\section{Software Allocation Optimization}
Different allocation schemes deliver different system performance and therefore efficient software allocation is crucial. Ernest Wozniak et al. \cite{Wozniak2013AnArchitectures} proposed a synthesis mechanism for an AUTOSAR software application that can execute over multiple nodes, with the objective of fulfilling timing requirements. In contrast, we consider power consumption and reliability requirements besides timing. Similarly, Salah Saidi et al. \cite{Saidi2015AnArchitectures} proposed an ILP based approach for allocation of an AUTOSAR application on a multi-core framework in order to reduce the overhead of inter-process communication while we consider a network of computing nodes. Ivan Svogor et al. \cite{vsvogor2014extended} proposed a generic approach of identifying resource constraints and a way of handling different measurement units with Analytic Hierarchy Process (AHP) in order to allocate a component-based software application on a heterogeneous platform. However, the resource constraints are trivialized, e.g., end-to-end delay calculations, which require complex timing analysis especially if data age constraints are considered~\cite{mubeen2013support}. In addition AHP scales for at most 10 components. As opposed to the previously mentioned related work, we considered a system model with multi-rate tasks, and data age constraints on the end-to-end communication of the tasks. Moreover, we consider fault-tolerant application, i.e. by replicating software components, to maximize the reliability of the application, hence meeting reliability requirements, that is in parallel to satisfying the end-to-end timing requirements. On a different front, there exists research focusing on power and energy consumption in real-time distributed systems which consider dynamic voltage scaling ~\cite{bambagini2016energy} and task consolidation to minimizing computational nodes ~\cite{faragardi2013towards}~\cite{devadas2012interplay}. However, the former is applicable after software integration, that is during runtime, and the latter do not consider power consumption minimization as their objective function, though target reducing the number of computing nodes.
    \chapter{Conclusions and Future Work}
In this thesis, we have proposed formal methods and optimization techniques for the assured and efficient design of safety-critical embedded systems. We have developed a domain-specific requirements specification language, called \resa, which is tailored to embedded systems, and applied SAT-based and ontology-based techniques to improve quality of specifications. We have also proposed an approach for formally modeling hybrid Simulink models via the stochastic timed automata formalism. Our approach is based on an automated pattern-based, execution-order-preserving method to transform large-scale Simulink models into networks of stochastic timed automata. The automata are analyzed for function and end-to-end timing requirements via \uppaalsmc~\cite{Bulychev2012UPPAAL-SMC:Automata}. For resource-constrained design in general and low-battery embedded systems in particular, we have proposed efficient software allocation methods via integer linear programming and hybrid particle optimization on a network of heterogeneous computing nodes, with respect to power specification, failure rate and processor speed. During the allocation, the timing and the reliability constraints are satisfied by considering the worst-case response time, age delay, and exact reliability analysis in the context of fault tolerance ensured by node redundancy.

Our solutions are validated on various industrial automotive use cases and one benchmark. The \resa{} language is validated on the Adjustable Speed Limiter (ASL) use case, for which we are able to express around 90\% out of 300 requirements, which are mostly conditional statements. However, the 10\% of ASL requirements contain quantifiers, temporal and timed properties, which are difficult to handle via the SAT-based approach. The SAT-based approach as opposed to the ontology-based approach scales well, moreover, it is fully automated and is also integrated into the \eatop{} tool chain at VGTT. However, the analysis with the SAT-based approch is shallow due to abstraction of clauses whereas the ontology-based approach is rigorous as it operates at the lexical level but it scales less. 

Our proposed formal analysis of Simulink models is validated on the BBW system that contains 320 Simulink blocks, to which it scales well and partially on the ASL system, on which the transformation could not handle some of the Simulink blocks, hence the generated model is not immediately analyzable.  However, the limitations are due to the implementation of our tool SIMPPAAL that we also propose in this thesis, which can be improved and limitations removed by future work. 

Our proposed software allocation approaches are validated on the \autosar{} benchmark produced by Bosch, which consists of several synthetic applications. The results show that the ILP approach scales to a medium size of application software, i.e., not more than 15 software components and 60 chains. In contrast, the hybrid particle swarm optimization scales well to very large applications, consisting of 80 software components and 60 chains per application. Of the metaheuristic algorithms evaluated in our experiments, the hybrid \pso{} algorithms with hill climbing provided better quality solutions next to \ilp{} on small problems, and performed best on larger problems as compared to the classical PSO and DE algorithms, as expected. In particular the hybrid \pso{} with the stochastic hill climbing algorithm outperforms the rest.

As lines of future work, we foresee the following: (i) implementation of the ontology-based requirements analysis using open-source lexical databases, e.g., WordNet; (ii) generalization of the statistical model checking to any data-flow programming paradigms; (iii) extension of the software allocation problem to include dynamic synthesis of tasks; (iv) extension of the software allocation techniques to also address the dynamic power and dynamic configuration of software components. 

    

    %\chapter{Introduction}\label{chapter_introduction}
\lettrine{R}{eal-time} systems are usually characterized by timely computations, which are bounded by \textit{deadlines} besides correct results of the computations~\cite{Buttazzo2003}. They are applied in many \textit{safety-critical embedded} systems, which are specialized computer systems designed for safety-critical applications~\cite{WangJiacun2017RES}, e.g., brake-by-wire, flight control. For instance, the brake-by-wire system~\cite{bibid} applies torque proportional to the pressing of the brake pedal in order to slow down (or halt) the vehicle. Besides computing the correct torque, the system must act timely, i.e., not too soon or not too late so that a traffic accident is avoided. Therefore, safety-critical real-time systems should be analyzed for functional as well as timing correctness, and in fact, they should be analyzed rigorously according to functiona safety standards, e.g., ISO 26262 ``Road vehicles-Functional safety''~\cite{iso201126262} recommends the use of \textit{formal methods}, which are mathematical techniques and tools that enable unambiguous specification, modeling and rigorous analysis~\cite{o2017concise}, to develop safety-critical automotive systems. In this thesis, we apply various formal methods early, i.e., at the requirements specification and software design levels, to improve the quality assurance of developing safety-critical embedded systems.

Over the last decades, a lot of functionality that have been provided by mechanical and electical systems have moved to software functionality, e.g., the x-by-wire technology~\cite{bibid}, and several software functionality, which are driven by innovation and safety standards, are integrated into safety-critical systems to provide advanced, safe and reliable services, e.g., the advanced driver-assistance systems, vision system for autonomous and self-driving vehicles. In response, several computing architectures have moved from a federated architecture (i.e., dedicated hardware systems executing safety-critial software systems or applications) into consolidated (i.e., multiple safety-critial software applications co-hosted on the same hardware) and distributed architectures. In distributed computing~\cite{Kopetz2003Real-timeApplications}, the safety-critical software is mapped on multiple hardware systems to capitalize on the computational power provided by the distributed architecture, e.g., the brake-by-wire software executing on multiple electronic control units (ECU) that are connected through a network bus, e.g., CAN~\cite{}. The distributed software, unlike in the case of federated computing, it is exposed to a greater degree of permanent and transient faults, therefore, reliability of the software should be maximized to improve the overall dependability~\cite{} of the  system. However, such measures requires additional critical system resources besides computational resources, e.g., power and energy, which are constrained in battery-driven embedded systems. Thus, the distributed safety-critical software should be deployed efficiently on the distributed architecture while meeting the timing and reliability requirements of the software.


In this thesis, we propose a design approach which considers rigoruous analysis and efficiency of safety-critical embedded systems by applying formal methods and optimization techniques at the early stages of software development. Thus, the safety-critical  specifications should be unambiguous, comprehensible, etc~\cite{ieereqspecstandard}, and the software design should conform to the specifications. Since natural languag is inherently ambigous~\cite{ieereqspecstandard}, many functional safety standards, e.g., ISO 26262, recommends semi-formal and formal specification methods to specify safety-critical embedded systems requirements, e.g., template-based languages, controlled natural language, temporal logic. The template-based specification methods, e.g., requirements boilerplates~\cite{Hull2011RequirementsEngineering}, property-specification systems~\cite{Dwyer1999PatternsVerificationb}, etc., lack meta-model to effectively create templates, and is usually cumbersome to select the templates. The controlled natural languages, e.g., Attempto~\cite{attempto96}\cite{Fuchs2008AttemptoRepresentation}, etc., renders the syntax and semantics of the natural language and have formal semantics, however lacks support for embedded systems, hence are less effective. Rather, we propose a constrained natural language which is domain-specific and uses the notion of boilerplates to facilitate reuse. The specifications have semantics in Boolean logic and description logic to enable rigorous analysis via Boolean satisfiability~\cite{Malik2009BooleanSuccess} and ontology~\cite{f25ea3c6f4b743cd90c150926bbcf3db}, respectively.

The requirements specifications are used in subsequent system development including software design to verify the latter for correct functionality. The software design is usually modeled, simulated and analyzed before implementation. In this regard, Simulink is one of the most widely used development environment for multi-domain, multi-rate, discrete and continuous safety-critical systems in industry~\cite{JamesB.Dabney2003MasteringSimulink}. For this main reason, there is increasing interest in formal analysis of Simulink models~\cite{Manamcheri2011AModels}. Simulink Design Verifier\footnote{Simulink Design Verifier - https://se.mathworks.com/products/sldesignverifier.html}, which is based on the Prover\footnote{Prover - https://www.prover.com/software-solutions-rail-control/formal-verification/} model checker, is the de~facto tool in the Simulink environment to formally verify Simulink design models. However, it has limited functionality, e.g., it supports only discrete models, has issues with scalability due to state-space explosion, and lacks verification of timed properties~\cite{Leitner2008SimulinkStudy}.  In contrast, we propose a scalable, timed analysis via a statistical model checking~\cite{Legay2010StatisticalOverview}, which uses traces of executions and statistical analysis techniques, e.g., monte-carlo simulation.

The software design should be mapped to hardware effectively, that is satisfying the timing and reliability requirements of the distributed safety-critical software, but also efficiently to minimize the power consumption of the distributed system to facilitate accommodating complex software functionality. We assume the software is scheduled using a fixed-priority preemptive policy, which is quite common in industry, and posses end-to-end timing requirements, e.g., the time duration between the brake-pedal press and the slow-down (or halt) of the vehicle. Furthermore,  we consider fault tolerance as a means to maximize reliability of the distributed safety-critical software by mapping redundant software functionality on different computing units.  We propose  \textit{exact} and \textit{heuristic} optimization methods, which deliver optimal and near-optimal solutions, respectively, to efficiently map the distributed safety-critical software to a network of computing nodes. Specifically, we propose a formulation of integer-linear programming (ILP)~\cite{Mahmud5222}, which is solved using branch and bound. Furthermore, we propose a hybrid-particle optimization~\cite{Mirjalili2019ParticleOptimisation}, which is a meta-heuristic algorithm, to solve the shortcomings of the exact method for large-scale problems~\cite{Mahmud2019Power-awareOptimization} with trade-off over optimality.

The main contributions of the thesis are: 
\begin{enumerate*}[label=(\roman*)]
\item formal analysis of natural language requirements - we propose a fairly expressive, flexible yet structured and domain-specific constrained natural language called \textit{ReSA}~\cite{resatool}\cite{Mahmud2015ReSA:Systems}. The language has semantics in Boolean and description logic to support shallow and rigorous analysis, respectively. The Boolean specifications are checked for consistency using the Boolean satisfiability via the Z3 SMT solver.  The ReSA tool is integrated seamlessly into EATOP to complement the requirements modeling of EAST-ADL. The language and its tool support are validated on the adjustable speed limiter (ASL) use case, which is a vehicle speed control system provided by Volvo Group Trucks Technology (VGTT);
%Whereas the description logic is used to encode the specifications as ontology, subsequently is checked for consistency at the lexical level via a semantic reasoner (i.e., HermiT as used in Prot{\'e}g{\'e} tool), which is a software program that produces logical consequencies from asserted axioms and facts.

\item scalable and formal analysis of Simulink models - we propose a pattern-based, execution-order preserving automatic transformation of a Simulink model into a network stochastic timed automata that can be formally analyzed using the UPPAAL statistical model checker (SMC)~\cite{Filipovikj2016SimulinkSystems}. The statistical model checker analyzes a state-transition system by conducting statistical analysis on the collected traces of the system executions~ \cite{Bulychev2012UPPAAL-SMC:Automata}, effectively mitigating the state-space explosion of (exact) model checking \cite{Legay2010StatisticalOverview}. Our proposed technique is validated on the brake-by-wire (BBW) use case, which is an industrial prototype provide for academic uses from VGTT;

\item efficient allocation of distributed safety-critical software applications - we propose an integer-linear programming (ILP) model and hybrid particle swarm optimization algorithms to allocate AUTOSAR software applications on a network of heterogenuous computing nodes with respect to processor speed, failure rate and power consumption specifications. The  ILP problem is implemented using ILOG CPLEX, which is a toolset for modeling and solving optimization problems. The proposed allocation methods are validated on an automotive benchmark developed according to AUTOSAR.
\end{enumerate*}


%Our solutions are evaluated on industrial automotive use cases and on a realistic benchmark. The formal analysis of the natural language requirements specifications in ReSA and the formal analysis of Simulink models are evaluated on the adjustable speed-limiter (ASL) and brake-by-wire (BBW) systems provided by Volvo Group Trucks Technology (VGTT). ASL is a speed-limitation automotive function which controls the vehicle speed of Volvo trucks from speeding up, and is useful in roads where speed-limitation signs are in place. The ASL use case consists of around 300 requirements, which are specified in natural language, architectural models in EAST-ADL and Simulink models. The integrated software allocation is evaluated on the engine management system benchmark provided by Bosch~\cite{} provided for AUTOSAR applications. The benchmark consists of statistics of the schedulable objects, such as mean values, shares of timing specifications and activation mechanisms of the schedulable objects in the system.

\section{Thesis Outline Overview}
The thesis is divided into two parts. The first part is a summary of our research. It is organized as follows: in Chapter 2, we give the background on the logic-based reasoning using Boolean and description logic, Simulink, UPPAL statistical model checking, and the optimization techniques based on ILP and PSO. In Chapter 3, we explain the research problem and outline the research goals. The thesis contributions are discussed in Chapter 4, which states the peer-review papers mappings to the contributions. In Section 5, we provide the related work on requirements specification, formal analysis of Simulink models, and software allocation. In Chapter 6, we describe the research method applied to conduct the research. Finally, in Chapter 7, we conclude the thesis, and outline the limitation and discuss possible directions for future work.


    %\input{ProblemsAndSolutions}
    %\input{Example.tex}
\backmatter

    % References
    % No restriction is set to the reference styles
    % Save your references in References.bib
    %\nocite{*} % Remove this for your own citations
    \bibliographystyle{plain}
    \bibliography{references}

\end{document}